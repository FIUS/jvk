\excercise{Rekusion}
\begin{Infobox}[Rekursion]
Ein Kommando welches sich selbst aufruft wird als rekusiv bezeichnet. Die Fibbonacci Reihe zum beispiel (1,1,2,3,5,8,13,21,...) kann beschrieben werden durch $F(0)=1$ $F(1)=1$ $F(n)=F(n-2)+F(n-1)$. wobei F(n) die n-te Fibbonacci Zahl ist. bei $F(n)=F(n-2)+F(n-1)$ wird erneut mit einem kleineren Argument verwendet. Dadurch wird das Argument in allen Schritten kleiner bis es bei $F(0)$ oder $F(1)$ ankommt welche einen Festen Wert haben.
\end{Infobox}
\begin{enumerate}
\item versuche (rekursiv) ein Fibbonacci Kommando zu implementieren welches einen Integer n erhällt und dann $F(n)$ Münzen ablegt.
\item lasse Neo die ersten Sechs Elemente der Fibbonacci Folge nacheinander auf das Spielfeld legen.
\item (bonus) implementiere das Fibbonacci Kommando aus a) itterativ (ohne Rekursion).  
\end{enumerate}
