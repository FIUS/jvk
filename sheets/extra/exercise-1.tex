
\textbf{Hinweis:} Alle Aufgaben auf diesem Blatt sind Optional und dementsprechend schwerer als die vorherigen Aufgaben.\\
\excercise{Totoro im Labyrinth}	
	Totoro hat sich in den Büschen verlaufen und will nach Hause zu seinem Baum.
	Er erinnert sich, dass man aus jedem Labyrinth entkommen kann, wenn man immer an der rechten Wand entlangläuft.\\

		 Implementiere die Operation \lstinline{checkSideBush()} der Klasse \lstinline{MySmartTotoro}.
		Sie soll \lstinline{true} zurückgeben, wenn sich auf Totoros rechter Seite ein Busch befindet.
		Ansonsten soll sie \lstinline{false} zurückgeben.\\



		 Implementiere die Operation \lstinline{moveSmart()} der Klasse \lstinline{MySmartTotoro}.
		In dieser soll Totoro zuerst überprüfen, ob sich zu seiner rechten ein Busch befindet.
		Ist dies nicht der Fall, soll er sich nach rechts drehen und einen Schritt in diese Richtung gehen.
		Anderenfalls soll er überprüfen, ob er geradeaus laufen kann. Ist dies der Fall, soll er einen Schritt nach vorne machen.
		Sollte sich sowohl rechts, als auch geradeaus ein Busch befinden, soll Totoro sich nach links drehen und stehen bleiben.\\

		 Die Klasse \lstinline{OptionalLabyrinthTask} enthält ein Attribut vom Typ SmartTotoro.
		Weise diesem Attribut in der prepare-Operation ein neues MySmartTotoro-Objekt zu und
		platziere es auf Position \emph{[0, 0]}.
		Vervollständige weiterhin die solve-Operation mithilfe der Operationen,
		die du gerade geschrieben hast, sodass Totoro seinen Baum erreicht und darauf stehen bleibt.

		

	