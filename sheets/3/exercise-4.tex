% !TeX root = ./jvk-blatt3.tex
\excercise{Operation parameters}

\begin{enumerate}
	\item 
		Passe die Main Operation an die Aufgabe an.\\
		Verwende \lstinline{Sheet3Task4} und \lstinline{Sheet3Task4Verifier}.
	\item 
		Schreibe ein Kommando das Neo einen Schritt nach vorne gehen lässt.
	\item 
		Schreibe je ein Kommando das Neo 2,3 und 4 Schritte nach vorne gehen lässt.
		Du solltest jetzt vier Kommandos haben, mit denen du Neo 1, 2, 3 oder 4 Schritte gehen lassen kannst.
\end{enumerate}

\begin{Infobox}[Operations Parameter]
	Wenn du eine Operation schreiben möchtest die je nach Situation mit verschiedenen Variablen die selbe Operation ausführt, kannst du der Operation einen oder mehre Parameter übergeben. 
	Bisher haben wir einer Operation immer den Neo übergeben mit dem die Operation ausgeführt werden soll.
	Man kann aber jede Art von Objekt als Parameter übergeben.
	
	\begin{lstlisting}[xleftmargin=0.5cm]
public void printNumbers(Neo neo, Integer from, Integer to){
    for(Integer i=from; i <= to; i++){
		System.out.print(i);
		neo.turnClockWise();
    }
}
	\end{lstlisting}

	Diese Abfrage gibt die Zahlen von \lstinline{from} bis \lstinline{to} aus. 
	Z.B. \lstinline{printNumbers(2,7)} gibt \lstinline{234567} aus.
	Für jede Zahl die ausgegeben wird, dreht sich Neo ein mal nach rechts.
\end {Infobox}


\begin{enumerate}\setcounter{enumi}{3}
	\item
		Schreibe nun ein Kommando in dem Neo \lstinline{n} Schritte läuft.
		Dazu braucht das Kommando zwei Parameter.
		Einen Parameter für Neo und einen für die Anzahl der Schritte.

	\item
		Verwende das Kommando aus d) um Neo eine größer werdende Spirale laufen zu lassen. 
		Das heißt:
		
		\begin{itemize}
			\item[] Laufe \emph{einen} Schritt.
			\item[] Drehe dich nach rechts.
			\item[] Laufe \emph{zwei} Schritte.
			\item[] Drehe dich nach rechts.
			\item[] Laufe \emph{drei} Schritte.
			\item[] Drehe dich nach rechts.
			\item[] Laufe \emph{vier} Schritte.
			\item[] Drehe dich nach rechts.
			\item[] ...usw.
		\end{itemize}

		Du solltest mit der Spirale aufhören, wenn Neo 12 oder mehr Schritte am Stück gegangen ist.
\end{enumerate}

\newpage
