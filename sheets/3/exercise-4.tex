% !TeX root = ./jvk-blatt3.tex
\excercise{Operation parameters}

\begin{enumerate}
	\item 
		Passe die Main an die Aufgabe an. Verwende \lstinline{Sheet3Task4} und \lstinline{Sheet3Task4Verifier}.
	\item 
		Schreibe ein Kommando die Neo einen Schritt nach vorne gehen lässt.
	\item 
		- Schreibe ein Kommando das Neo 2 Schritte nach vorne gehen lässt.\\
		- Schreibe ein Kommando das Neo 3 Schritte nach vorne gehen lässt.\\
		- Schreibe ein Kommando das Neo 4 Schritte nach vorne gehen lässt.
\end{enumerate}

\begin{Infobox}[Operations Parameter]
	Wenn du eine Operation schreiben möchtest die je nach Situation mit verschiedenen Variablen die selbe Operation ausführt, kannst du der Operation einen oder mehre Parameter übergeben. 
	Bisher haben wir der Methode immer den Neo übergeben der alle Operationen ausführen sollte, man kann aber jede art von Objekt als Parameter übergeben.
	
	\begin{lstlisting}[xleftmargin=0.5cm]
public void printNumbers(Integer from, Integer to){
    for(int i=from; i <= to; i++){
        System.out.print(i);
    }
}
	\end{lstlisting}

	Diese Abfrage gibt die Zahlen von\lstinline{from} bis \lstinline{to} aus. 
	Z.B. \lstinline{printNumbers(2,7)} gibt \lstinline{234567} aus.
\end {Infobox}


\begin{enumerate}\setcounter{enumi}{3}
	\item
		Schreibe nun ein Kommando in dem Neo \lstinline{n} Schritte läuft.

	\item
		Verwende das Kommando aus d) um Neo eine größer werdende Spirale laufen zu lassen. Das heißt:
		- Laufe einen Schritt.
		- drehe dich nach rechts
		- Laufe zwei Schritte
		- drehe dich nach rechts
		- Laufe drei Schritte
		- drehe dich nach rechts
		- Laufe vier Schritte
		- drehe dich nach rechts
		- ...usw
\end{enumerate}

\newpage
