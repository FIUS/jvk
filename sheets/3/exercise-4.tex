% !TeX root = ./jvk-blatt3.tex
\excercise{Operationsparameter}

\begin{enumerate}
	\item
	Instanziiere die Simulation wie bekannt (\lstinline{Sheet3Task4} und \lstinline{Sheet3Task4Verifier}) und mache dich mit dieser vertraut.
\item
Schreibe ähnlich zu den Methoden aus der letzten Aufgabe in der \lstinline{Sheet3Task4}-Klasse ein Kommando, das Totoro einen Schritt nach vorne gehen lässt.
\item
Schreibe je ein Kommando das Totoro 2,3 und 4 Schritte nach vorne gehen lässt.
Du solltest jetzt vier Kommandos haben, mit denen du Totoro 1, 2, 3 oder 4 Schritte gehen lassen kannst.
\end{enumerate}

\begin{Infobox}[Operationsparameter]
Wenn du eine Operation schreiben möchtest, die je nach Situation mit verschiedenen Variablen dieselbe Operation ausführt, kannst du der Operation einen oder mehre Parameter übergeben.
Bisher haben wir einer Operation immer Totoro übergeben, mit dem die Operation ausgeführt werden sollte.\newline

Man kann aber beliebig viele Typen als Parameter an eine Funktion übergeben:

	\begin{lstlisting}[xleftmargin=0.5cm]
public void printNumbers(Totoro totoro, int from, int to){
    for(int i = from; i <= to; i++){
		System.out.print(i);
		totoro.turnClockWise();
    }
}
	\end{lstlisting}

	Diese Abfrage gibt die Zahlen von \lstinline{from} bis \lstinline{to} aus.
	Z.B. \lstinline{printNumbers(2,7)} gibt \lstinline{234567} aus.
	Für jede Zahl die ausgegeben wird, dreht sich Totoro ein mal nach rechts.
\end {Infobox}


\begin{enumerate}\setcounter{enumi}{3}
	\item
		Schreibe nun ein Kommando, das Totoro $n$ Schritte laufen lässt.

	\item
		Verwende das Kommando aus d) um Totoro eine größer werdende Spirale laufen zu lassen.
		Das heißt:

		\begin{itemize}
			\item[] Laufe \emph{einen} Schritt.
			\item[] Drehe dich nach rechts.
			\item[] Laufe \emph{zwei} Schritte.
			\item[] Drehe dich nach rechts.
			\item[] Laufe \emph{drei} Schritte.
			\item[] Drehe dich nach rechts.
			\item[] Laufe \emph{vier} Schritte.
			\item[] Drehe dich nach rechts.
			\item[] ...usw.
		\end{itemize}

		Du solltest mit der Spirale aufhören, wenn Totoro 12 oder mehr Schritte am Stück gegangen ist.
\end{enumerate}

\newpage
