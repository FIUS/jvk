    \excercise{Anwendung von Schleifen und If-Anweisungen}
            \subexcercise Morpheus musste untertauchen. Aber damit Neo ihn finden kann, hat er einen Pfad aus Pillen gelegt. Immer wenn Neo auf einem Feld steht, auf dem eine rote Pille liegt, muss er nach rechts abbiegen.

            \textbf{Extra für diese Aufgabe} stellen wir eine spezielle Operation \texttt{Neo.peakPill()} zur Verfügung, die die Pille zurück gibt, die auf dem selben Feld liegt, auf dem Neo gerade steht. Die Operation gibt null zurück, wenn Neo auf einem Feld steht, auf dem es keine Pille gibt.

            Außerdem kann man die Pille mit der Query \texttt{getColor()} fragen, welche Farbe sie hat.
            Die Query gibt \texttt{Color.RED} zurück, wenn die Pille rot ist und \texttt{Color.BLUE}, wenn die Pille blau ist.

            Neo muss solange suchen, bis er auf eine Telefonzelle trifft.

            \subexcercise Die Agenten sind Morpheus auf der Spur. Er muss den Weg komplizierter machen. Zusätzlich zu den roten Pillen gibt es jetzt auch blaue Pillen. Wenn man auf einem Feld steht, das eine blaue Pille enthält, muss Neo nach links abbiegen.

            \textbf{Diese Aufgabe} enthält wieder die Operation \texttt{Neo.peakPill()}. Außerdem bauen wir das Level so, dass es kein Feld gibt, auf dem sowohl eine blaue, als auch eine rote Pille liegt.
