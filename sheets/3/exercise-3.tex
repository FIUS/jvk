\excercise{Operationen }

\begin{enumerate}
	\item
	Instanziiere die Simulation wie bekannt (\lstinline{Sheet3Task3} und \lstinline{Sheet3Task3Verifier}) und mache dich mit dieser vertraut.
	\item
		Lass Neo das folgende Muster laufen:
		\begin{itemize}
			\item[] laufe 2 Schritte
			\item[] …drehe dich nach rechts
			\item[] …laufe einen Schritt
			\item[] …drehe dich nach links
			\item[] …laufe 3 Schritte
			\item[] …drehe dich nach links
			\item[] …laufe 2 Schritte
			\item[] …drehe dich nach rechts
			\item[] …laufe 2 Schritte
			\item[] …drehe dich nach rechts
			\item[] …laufe einen Schritt
			\item[] …drehe dich nach links.
		\end{itemize}
	\item
		Laufe das Muster aus b) mithilfe einer For-Schleife 3 Mal.
\end{enumerate}


\begin{Infobox}[Operation um Code mehrfach zu verwenden]
	Wenn man denselben Code an mehreren Stellen verwenden möchte ist es nicht sinnvoll diesen mehrmals zu kopieren.
Bei wenig Code denkt man vielleicht bei einer Änderung daran, dass man ihn kopiert hat und passt die andere Stelle an.
Wenn der Code allerdings länger wird und man ihn an mehrere Stellen kopiert hat, ist es sehr schwer den Überblick zu behalten und es kommt zu ungewollten Inkonsistenzen bei der Ausführung.\newline

Deshalb sollte man Code, der an mehreren Stellen verwendet wird, in Operationen auslagern.
So kann man dann die Operation an mehreren Stellen aufrufen und bei Bedarf für Änderungen einfach den Code in der Methode anpassen, um Inkonsistenzen zu vermeiden.
Als netten Nebeneffekt erhält man zudem noch besser verständlichen Code, wenn man die Methodennamen gut wählt.

	\begin{lstlisting}[xleftmargin=0.5cm]
public class DemoTask implements Task {

    @Override
    public void run(Simulation sim) {
        Neo neo = new Neo();
        // Anfang für Beispiel weggelassen

        neo.collectCoin();
        this.moveTwiceAndTurn(neo);
        neo.collectCoin();
        this.moveTwiceAndTurn(neo);
    }

    /**
	 * This method moves the neo parameter twice and then turns him clockwise.
	 */
    public void moveTwiceAndTurn(Neo neo) {
        neo.move();
        neo.move();
        neo.turnClockWise();
    }

}
	\end{lstlisting}

	Auf Aufgabenblatt 2 hast du schon eine Operation in der Neo-Klasse implementiert.
Da die Operation im Beispiel oben ein sogenanntes \textit{Kommando} ist, hat sie ebenfalls keinen Rückgabewert.
Deshalb ist hier vor dem Operationsnamen \lstinline{moveTwiceAndTurn} \lstinline{void} angegeben.\newline

Anders als bei dem Kommando in der Klasse \lstinline{Neo} können wir diesmal nicht mit \lstinline{this} auf die Operationen von einem Objekt der Klasse \lstinline{Neo} zugreifen.
In einer Klasse \lstinline{Task} haben wir mit \lstinline{this} nämlich nur Zugriff auf die Operationen und Attribute der \lstinline{Task}-Klasse selbst.
In dem Beispiel oben könnten wir also \lstinline{this.run(/*...*/)} oder \lstinline{this.moveTwiceAndTurn(/*...*/)} aufrufen, da das Methoden sind, die in dieser Klasse definiert wurden.\newline

Da wir die Operationen von Neo nicht über \lstinline{this} aufrufen können, müssen wir der neuen Operation eine Variable vom Typ Neo als Parameter übergeben.
Parameter werden in Aufgabe 4 nochmal genauer erklärt.\newline

\textbf{Achtung:} Wenn in der \lstinline{moveTwiceAndTurn}-Methode wieder \lstinline{this.moveTwiceAndTurn(...)} aufgerufen wird, entsteht eine sogenannte endlose Rekursion.
Die Methode würde sich dadurch theoretisch unendlich oft immer wieder selber aufrufen, was aufgrund von technischen Details nach einigen (tausenden?) Aufrufen automatisch abgebrochen werden würde.
Rekursion wird auf dem Extra Blatt und der PSE Vorlesung noch genauer erklärt. Als netten Nebeneffekt erhält man zudem noch besser verständlichen Code, wenn man die Methodennamen gut wählt.

\end{Infobox}


\begin{enumerate}\setcounter{enumi}{3}
	\item
	Lagere nun den Code aus b) in die gegebene Operation \lstinline{movePattern} aus.

	\item
	Ersetze jetzt den Code in der Schleife aus c) durch die Operation \lstinline{movePattern}.
	
	\item
	In der Klasse \lstinline{Sheet3Task3} findest du auch die folgende Operation.
	
	\begin{lstlisting}
	private void dropFourCoinsAndTurnLeft(Neo neo) {
	//write the Code for f) here
	//make neo drop four coins
	//make neo turn Left
	}
	\end{lstlisting}
	
	Neo soll in dieser Operation 4 Münzen fallen lassen und sich nach links drehen.
	Ersetze den Kommentar in der Operation durch die passenden Kommandos.
	
	\item
	Führe jetzt in der Schleife aus c) nach \lstinline{movePattern} einmal \lstinline{dropFourCoinsAndTurnLeft} aus.
	Der Schleifenkörper sollte beide Operationen enthalten.
	
	\item
	Nun möchten wir eine eigene Operation schreiben.
	Kopiere dazu erst mal die \lstinline{movePattern} Operation und ändere den Namen in etwas anderes (z.B. \lstinline{moveStraight}).
	Ersetze nun in der Schleife aus c) den Aufruf von \lstinline{movePattern} durch einen Aufruf der neuen Operation.
	
	Du solltest jetzt in der Klasse \lstinline{Sheet3Task3} zwei Operationen haben, die sich nur im Namen unterscheiden.
	
	\item
	Schreibe die Operation aus h) so um, dass Neo sich nur durch geradeaus laufen, nach Ausführung der Operation, an derselben Stelle befindet wie nach dem Ausführen der Operation \lstinline{movePattern}.
	Wenn Neo an der gleichen Position startet, dann sollte er, egal ob \lstinline{movePattern} oder die neue Operation ausgeführt wurde, an der gleichen Position stehenbleiben.
	
	Teste deine Implementierung und überprüfe, ob die Münzen an den gleichen Stellen abgelegt werden, wenn du in der Schleife \lstinline{movePattern} durch deine neue Operation ersetzt hast.
\end{enumerate}
\newpage
