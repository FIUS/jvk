\excercise{Methode}
\subsection*{a)}
	Passe die Main Methode an die Aufgabe an. Verwende \lstinline{Sheet3Task3} und \lstinline{Sheet3Task3Verifier}.
\subsection*{b)}
	Lass Neo das folgende Muster laufen:\\
	- laufe 2 Schritte\\
	- drehe dich nach rechts\\
	- laufe einen Schritt\\
	- drehe dich nach links\\
	- laufe 3 Schritte\\
	- drehe dich nach links\\
	- laufe 2 Schritte\\
	- drehe dich nach rechts\\
	- laufe 2 Schritte\\
	- drehe dich nach rechts\\
	- laufe einen Schritt\\
	- drehe dich nach links.
\subsection*{c)}
	Laufe das Muster aus b) mithilfe einer for-Schleife 3 Mal.
\subsection*{d)}
	\begin{Infobox}[Operation]
		Wenn man den selben Code an mehreren Stellen verwenden möchte ist es nicht sinnvoll diesen mehrmals zu kopieren. Stattdessen kann man den Code in eine Operation auslagern. Dann kann man den gesammten Code in nur einer Zeile  aufrufen.
				\begin{lstlisting}
public class DemoTask implements Task {
    
    @Override
    public void run(Simulation sim) {
        PlayfieldModifier pm = new PlayfieldModifier(sim.getPlayfield());
        Neo neo = new Neo();
        pm.placeEntityAt(neo, new Position(0, 0));
     
        moveThreeStepsBackward(neo); //neo moves back three Steps
        neo.turnClockWise(); //neo turns right
        moveThreeStepsBackward(neo); //neo moves back three Steps
	
        
    }
    
    public void moveThreeStepsBackward(Neo neo) {
        //turn around
        neo.turnClockWise();
        neo.turnClockWise();
        //move three times
        neo.move();
        neo.move();
        neo.move();
        //turn around
        neo.turnClockWise();
        neo.turnClockWise();
    }
    
}
		\end{lstlisting}
void heißt, dass diese Operation keinen Rückgabewert hat. 
Wir werden vorerst nur void Operationen verwenden. 
Das \lstinline{(Neo neo)} heißt, dass wir den Neo übergeben müssen der sich bewegen soll. 
Mehr dazu in Aufgabe 4. 
Das stichwort \lstinline{public} könnt ihr vorerst ignorieren.
 	\end{Infobox}
Lagere nun den Code aus b) in die gegebene Operation  \lstinline{movePatern} aus.
\subsection*{e)}
Ersetze nun den Code in der Schleife aus c) durch die Operation  \lstinline{movePatern}.
\subsection*{f)}
Schreibe die folgende Operation in Sheet3Task3.
\begin{lstlisting}
private void dropFourCoinsAndTurnLeft(Neo neo) {
    //write the Code for f) here
    //make neo drop four coins
    //make neo turn Left
}
\end{lstlisting}
Neo soll in dieser Operation 4 Münzen fallen lassen und sich nach links drehen. Schreibe den passenden Code in die Operation.
\subsection*{g)}
führe nun in der Schleife aus c) nach \lstinline{movePatern} ein mal \lstinline{dropFourCoinsAndTurnLeft} aus.
\subsection*{h)}
Nun möchten wir eine eigene Methode schreiben. Kopiere dazu erst mal die movePatern Operation und ändere den Namen in etwas anderes.
Ersetze nun in der Schleife aus c) \lstinline{movePatern} durch die neue Methode.
\subsection*{i)}
Schreibe die Operation aus h) um sodass Neo sich nur durch geradeauslaufen nach ausführung der Operation an der selben Stelle befindet wie nach ausführen der Operation  \lstinline{movePatern}.
\newpage