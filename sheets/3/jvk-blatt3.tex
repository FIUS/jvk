\documentclass{../vorkurs}

\def\sheetNumber{3}

\preamble{
\begin{center}
        \begin{tabular}{ l | c | c | l }
                Text & Math. Zeichen & Java & Bemerkung\\
            \hline
                  gr\"oßer als & $>$ & $>$ & \\
                  kleiner als & $<$ & $<$ & \\
                  gleich & $=$ & $==$ & einfaches $=$ ist in Java eine Zuweisung\\
                  ungleich & $\neq$ & $!=$ & \\
                  gr\"oßer gleich & $\geq$ & $>=$ & wie man es spricht \glqq größer gleich\grqq{} \\
                  kleiner gleich & $\leq$ & $<=$ &  \\
        \end{tabular} \\
        \vspace{1cm}
        \begin{tabular}{ l | l | l | l}
            Text & log. Zeichen & Java Zeichen &Bemerkung \\
            \hline
            und  &$\land$& $\&\&$& \minibox{Achtung kein einfaches \& in Java verwenden,\\ das macht etwas anderes (aber ähnliches)} \\
            oder  &$\lor$& $||$& Kein einfaches $|$ verwenden (analog zu und) \\
            & &  \\

            nicht & $!$&$\neg$ &\\
    \end{tabular}\\
\end{center}
\subsubsection*{Aufgabe 0: Vererbung}
Neo hat im Lotto gewonnen und hat jetzt viel Geld. Da er von Keanu Reevs gespielt wird und daher alle Menschen liebt bzw. sowieso schon zu viel Geld auf dem Konto hat, verschenkt er seinen Gewinn. Schreibe die Klasse RichNeo die Neo dirket mit 100000 Coins im Wallet erzeugt. Außerdem lässt RichNeo bei jedem Schritt 100 Coins auf dem Boden zurück.

\emph{Zusatz:} Lass RichNeo auf eine Wanderung gehen: RichNeo soll so lange laufen bis er nur noch die Hälfte seines Gelds hat.

\emph{Zusatz:} Weil wir im letzten Moment die Reihenfolge geändert haben ist die Nummer für diesen Taks 3.5.

Denke daran, in der Klasse \texttt{Task3\_5.java} die Klasse von Neo auf RichNeo zu ändern, sobald du die Klasse \texttt{RichNeo.java} erstellt hast.
}
\makedocument
