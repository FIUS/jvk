\excercise{Inputtest and Exeptions}

\begin{enumerate}
	\item
		Passe die Main Operation an die Aufgabe an.\\
		Verwende den selben Task und Verifier wie in Aufgabe 4.

	\item
		Probier die Operation aus Aufgabe 4 d) mit unterschiedlichen werten aus: (10, 1, 0, -1, \lstinline{Integer.MAX_VALUE}) \\
		Was passiert?
\end{enumerate}

	
\begin{Infobox}[Return]
	Mit dem Schlüsselwort \lstinline{return} kannst du den Rückgabewert einer Operation festlegen.
	Der Rückgabewert muss nach dem Schlüsselwort \lstinline{return} und vor dem \lstinline{;} stehen.

	Return ist immer die letzte Anweisung einer Operation die ausgeführt wird.
	Du kannst mit \lstinline{return} also die Operation beenden.
	Das funktioniert sogar in Kommandos die ja keinen Rückgabewert haben.

	Wenn nach dem \lstinline{return} noch Anweisungen stehen die ganz sicher nicht mehr erreicht werden können, dann wird Java das Programm nicht mehr ausführen wollen.
	Ihr könnt die Zeilen nach dem Return (bis zum Ende des Code-Blocks der Operation) auskommentieren damit Java das Programm wieder ausführt.

	Beispiel für \lstinline{return} in einer Abfrage:
	
	\begin{lstlisting}[xleftmargin=0.5cm]
public Integer getAnswer(){
    return 42;
}
	\end{lstlisting}

	Wenn man \lstinline{getAnswer()} aufruft bekommt man den Wert 42 zurück.

	Beispiel für \lstinline{return} in einem Kommando:

	\begin{lstlisting}[xleftmargin=0.5cm]
public void printNumbers(){
    for(int i=0; i < 10; i++){
        System.out.print(i);
        if ( i == 5){
            return;
        }
    }
}
	\end{lstlisting}

	Wenn man nun \lstinline{printNumbers();} aufruft dann ist die Ausgabe in der Konsole \lstinline{012345}, da die Operation bei i=5 beendet wird.
\end{Infobox}


\begin{enumerate}\setcounter{enumi}{2}
	\item
		Wir wollen nun nicht, dass Neo in dieser Operation nicht mehr als 50 Schritte laufen kann. 
		Wenn Neo mehr als 50 Schritte laufen soll, dann soll er sich gar nicht bewegen.
		Füge ein am Anfang des Kommandos aus Aufgabe 4 d) ein IF-Statement ein, in dem du prüfst ob der Parameter für die Anzahl der Schritte größer als 50 ist.
		Wenn ja, dann nutze \lstinline{return;} um die Ausführung der Operation direkt abzubrechen.

	\item
		Probiere nun erneut unterschiedliche Werte, insbesondere die Werte aus b), aus.
		Das Verhalten der Operation sollte sich geändert haben.
\end{enumerate}

\medskip % for better wrapping in following infobox

\begin{Infobox}[Throwing Exeptions]
	Wir möchten nicht, dass ungültige Usereingaben einfach Ignoriert werden. 
	Darum sollte man bei ungültigen Eingaben eine Exeption werfen.
	Für eine Erklärung von Exceptions kannst du dir Nochmal Aufgabe 2 auf Blatt 2 ansehen.

	Um eine exeption zu werfen verwendet man das Schlüsselwort \lstinline{throw}.

	\begin{lstlisting}[xleftmargin=0.5cm]
public void printNumbersTo(Integer to) {
    if (i < 0) {
        throw new IllegalArgumentException("negative Number");	
    }
    for(int i=0; i <= to; i++) {
        System.out.print(i);
    }
}
	\end{lstlisting}
	
	Diese Operation wirft bei negativen Werten eine \lstinline{IllegalArgumentException}.
\end{Infobox}


\begin{enumerate}\setcounter{enumi}{4}
	\item
		Ändere nun die Operation so, dass zu große Zahlen die Operation nicht mehr mit einem \lstinline{return;} abbrechen lassen.
		Stattdessen soll eine \lstinline{IllegalArgumentException} mit einer passenden Nachricht geworfen werden.
		Du kannst die Zeile mit dem \lstinline{return;} einfach durch die Zeile \lstinline{throw new IllegalArgumentException("");} ersetzen.

		Wähle eine hiflreiche Nachricht für deine \lstinline{IllegalArgumentException}.

	\item
		Jetzt bekommt man eine Rote Fehlermeldung wenn man versucht die Operation mit einem zu großen Parameter aufzurufen.
		Wir bekommen also den Fehler deutlich mit.

		Noch besser wäre es, wenn wir den Fehler gar nicht erst verursachen würden.
		Dazu müssen wir wissen, dass diese Operation eine Exception wirft wenn der Parameter zu groß ist.
		Das kann man im Javadoc Kommentar der Operation dokumentieren.
		
		Darum erweitern wir jetzt den Javadoc Kommentar der Operation um zu erklären wann die Exeption geworfen wird. 
		Verwende dafür \lstinline{@throws} in dem Javadoc Kommentar.

		Hier ist nochmal ein Beispiel wie man \lstinline{@throws} in einem Javadoc Kommentar verwendet.

		\begin{lstlisting}
/**
 * prints the numbers from 0 to 'to' in consecutive order.
 * 
 * @param to the number to which to print.
 * @throws IllegalArgumentException
 *         if 'to' is negative
 */
public void printNumbersTo(int to){
    // inhalt wie im Beispiel oben
}
		\end{lstlisting}
\end{enumerate}
\newpage
