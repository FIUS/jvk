% !TeX root = ../jvk-blatt3.tex

\vkchapter{IF und Conditionals}

In diesem Kapitel werden If-Verzweigungen behandelt sowie logische Vergleichsoperationen.

\begin{Infobox}[Bool'sche Werte]
    Um if-Verzweigung verstehen zu können, müssen wir uns zuerst Bool'sche Werte anschauen. Bool'sche Werte sind Variablen oder auch Rückgabewerte von Funktionen, welche nur den Wert \lstinline{true} oder \lstinline{false} (also wahr oder falsch) annehmen können. \\
    
    Es ist wichtig zu erwähnen, dass in Java der primitive Datentyp \lstinline{boolean} und das Objekt \lstinline{Boolean} dieselben Werte speichern können, aber nicht derselbe Typ sind.
    Den genauen Unterschied zwischen den beiden Typen werdet ihr noch in der PSE (Programmierung und Softwareentwicklung) kennenlernen.
    
    Die Funktion \lstinline{neo.canMove()} zum Beispiel gibt den bool'schen Wert \lstinline{true} zurück, wenn Neo eine Schritt in seine Blickrichtung gehen kann, also auf diesem Feld keine Hindernisse sind.
    Die \lstinline{canMove}-Methode hat also den Rückgabewert \lstinline{boolean}.
    
\end{Infobox}

\begin{Infobox}[IF-Verzweigung]
    Nun betrachten wir ein beispielhaftes IF-Statement:

    \begin{lstlisting}[breaklines=true, numbers=none]
        if (neo.canMove()) {
            neo.move();
        } 
    \end{lstlisting}

    Das beispielhafte IF-Statement bedeutet informal, dass falls \lstinline{neo.canMove()} den bool'schen Wert \lstinline{true} hat, wird \lstinline{neo.move()} ausgeführt. Falls \lstinline{neo.canMove()} den bool'schen Wert \lstinline{false} hat wird der Code in den geschweiften Klammern übersprungen.

    \begin{lstlisting}[breaklines=true, numbers=none]
    if (neo.canMove()) {
        neo.move();
    } else {
        System.out.println("Morpheus: There is a difference between knowing the path and walking the path")
    }
    \end{lstlisting}

    Nun wurde noch ein ELSE-Statement angehängt. Dadurch dient der Code als Verzweigung mit der Bedingung, dass falls etwas gilt mache etwas, falls nicht mache irgendwas anderes. In unserem Beispiel, falls \lstinline{neo.canMove()} den bool'schen Wert \lstinline{false} hat wird auf der Konsole der Text ''Morpheus: There is a difference between knowing the path and walking the path'' ausgegeben.\\

    Für sowohl das IF- als auch das ELSE-Statement gilt, dass in den geschweiften Klammern beliebig viel Code stehen kann, nicht nur eine Zeile wie in unserem Beispiel.
\end{Infobox}

\addexcercise

\begin{Infobox}[If-Statements 2]

	If-Statements und die Vergleichsoperatoren (wie z.B. \lstinline{<=}) kennst du ja schon.
	Diese können wie schonmal erwähnt zusammen mit primitiven Datentypen wie Ganzzahlen (\lstinline{int}s) verwendet werden.\newline

	Hier ein Beispiel:

	\begin{lstlisting}[numbers=none]
	int x = 2;
	if ( x < 3 ) {
		System.out.println("x ist kleiner als 3!");
	}
	\end{lstlisting}

	Das kann man zum Beispiel dazu verwenden, um zu schauen wie viele Münzen Neo hat.
	In Java kannst du die folgenden Vergleichsoperatoren verwenden:

	\begin{center}
		\begin{tabular}{ c | c | c | l }
			Natürliche Sprache & Mathematisches Zeichen & Java Operator \\
			\hline
			größer als & $>$ & \texttt{>} \\
			kleiner als & $<$ & \texttt{<} \\
			gleich & $=$ & \texttt{==}\\
			ungleich & $\neq$ & \texttt{!=} \\
			größer gleich & $\geq$ & \texttt{>=} \\
			kleiner gleich & $\leq$ & \texttt{<=} \\
        \end{tabular} \\
	\end{center}

	Damit können wir nun prüfen, ob sich auf Neo's Feld genau 3 Münzen befinden:

	\begin{lstlisting}[numbers=none]
if ( neo.getCurrentlyCollectableCoins().size() == 3 ){
	System.out.println("neo hat genau 3 Münzen!");
}
	\end{lstlisting}
\end{Infobox}

\begin{Infobox}[\lstinline{==} und die \lstinline{equals()}-Methode]
	\textbf{Vorsicht:} Mit dem Vergleichsoperator \enquote{\lstinline{==}} sollte man in der Regel nur primitive Datentypen (= etwa Werte vom Typ \lstinline{int}, Fließkommazahlen wie \lstinline{double} und ein paar andere Typen die wir der Einfachheit halber hier weglassen) vergleichen!\newline

	Zum Vergleichen von zwei Objekten sollte immer die \lstinline{equals}-Methode, die auf jedem Objekt definiert ist, verwendet werden:

	\begin{lstlisting}[numbers=none]
if (neo1.getPosition().equals(new Position(1, 1))){
	System.out.println("Neo is on Position (1, 1)");
}
	\end{lstlisting}

	Mit dem folgenden Codebeispiel kannst du in der Konsole sehen, dass \lstinline{==} und \lstinline{.equals} nicht zu dem gleichen Ergebnis kommen müssen, wenn man zwei Objekte vergleichen will.

	\begin{lstlisting}[numbers=none]
System.out.println(new Position(1,1) == new Position(1,1)); // false
System.out.println(new Position(1,1).equals(new Position(1,1))); // true
	\end{lstlisting}

\end{Infobox}

\begin{Infobox}[Logische Operatoren]

	Wenn mehrere Bedingungen (also bool'sche Werte) für ein Ereignis eine Rolle spielen, muss man diese logisch verknüpfen.
	Es folgt eine Auflistung der häufigsten Operatoren:

	\begin{center}
		\begin{tabular}{ c | c | c | l}
			Natürliche Sprache & logische Operatoren & Java Operatoren & Bemerkung \\
			\hline
			und  & $\wedge$ & \texttt{\&\&} & \minibox{Das einfache \texttt{\&} hat eine\\etwas andere Funktion} \\
			oder & $\vee$ & \texttt{||} & Dasselbe gilt für \texttt{|} \\
			nicht & $\neg$ & \texttt{!} &\\
		\end{tabular}\\
	\end{center}

	Hier ein paar Beispiele der Anwendung dieser in Java-Verzweigungen.
	Natürlich kann man die logischen Operatoren aber auch in while- oder for-Schleifeköpfen, oder sonst wo verwenden:

	\begin{lstlisting}[numbers=none]
if (neo.canMove() && neo.getCurrentlyCollectableCoins().size() == 7) {
	//neo hat genau 7 Münzen UND kann nach vorne gehen.
}
	\end{lstlisting}

	\begin{lstlisting}[numbers=none]
if (!neo.canMove() && neo.getCurrentlyCollectableCoins().size() == 7) {
	//neo hat genau 7 Münzen UND kann sich NICHT nach vorne gehen.
}
	\end{lstlisting}

	\begin{lstlisting}[numbers=none]
if (neo.canMove() || neo.getCurrentlyCollectableCoins().size() == 7) {
	//neo hat genau 7 Münzen ODER kann nach vorne gehen.
}
	\end{lstlisting}

	Zu beachten ist außerdem, dass das \q{oder} \lstinline{||} kein ausschließendes \q{oder} ist, daher ist \lstinline{((2 < 3) || (42 != 4))} eine wahre Aussage.

\end{Infobox}

\addexcercise
