%!TEX root = ./jvk-blatt3.tex

\excercise{Verzweigung}

In dieser Aufgabe wollen wir uns Conditions (Verzweigungen) näher anschauen.
% Chapter11Task1 ist alte Sheet2Task4
\begin{enumerate}                           
    \item Instanziiere die Simulation wie bekannt (\lstinline{Chapter11Task1} und \lstinline{Chapter11Task1Verifier}) und mache dich mit dieser vertraut.
        Was passiert bei der Ausführung?
    \item Betrachte die Operation \lstinline{movement} der Klasse \lstinline{Chapter11Task1}.\\
        Füge das obige IF-Statement an einer geeigneten Stelle ein und führe die Simulation erneut aus.
        Dabei sollten die dann noch vorhandenen \lstinline{.move()} Kommandos in den ersten Code-Block vom IF verschoben werden.
        Der \lstinline{else} Code-Block soll erstmal leer bleiben.
    \item Schließe nun das Fenster und öffne es erneut.
        Lösche die Wand in dem Spielfenster.
        Dazu musst du oben rechts das rote Minus auswählen und dann auf die Wand klicken.
        Starte dann die Simulation.\\
        Neo sollte sich jetzt anders verhalten, obwohl du den Code nicht geändert hast!
    \item Jetzt wollen wir den Code so anpassen, dass Neo die Telefonstation auch erreicht wenn eine Wand im Weg ist.

        Du musst also einmal den Fall betrachten, dass keine Wand im Weg ist und einmal den Fall, dass eine Wand im Weg ist.\\

        Überprüfe deinen Code, indem du einmal die Simulation startest, ohne die Wand zu löschen und einmal, wenn du die Wand davor gelöscht hast.
        In beiden Fällen sollte Neo die Telefonzelle erreichen.

\end{enumerate}
