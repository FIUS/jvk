    \excercise{For-Schleifen}
\subsection*{a)}
	Passe die Main Methode an die Aufgabe an. Verwende \lstinline{Sheet3Task1} und \lstinline{Sheet3Task1Verifier}.
\subsection*{b)}
	Probiere Neo mithilfe einer While-Schleife zehn Schritte laufen zu lassen.\\
	Hinweiß: Wenn du nicht weißt wie du Neo genau 10 Schritte laufen lässt kannst du die Simulation mithilfe der Pause Taste stoppen nachdem Neo 10 Schritte gelaufen ist.

\begin{Infobox}[For-Schleife]
		Mit einer For-Schleife kann man genau bestimmen wie of der Code in der Schleife ausgeführt wird.
		Anstatt dieselbe Operation n-mal untereinander zu schreiben um sie n-mal auszuführen, kann man den Code ein mal in eine For-Schleife die n-mal ausgeführt wird schreiben.
		Dazu verwenden wir eine for-Schleife. 
		\begin{lstlisting}
//here neo moves five times
this.move();
this.move();
this.move();
this.move();
this.move();
		\end{lstlisting}
		ist equivalent zu:
		\begin{lstlisting}
//this moves neo five times
for (int i = 0; i < 5; i++) {
	this.move();
}
		\end{lstlisting}
 \lstinline{int i = 0} setzt die Variable i anfangs auf 0.  \lstinline{i++} heißt, dass I jeden Schleifendurchlauf hochgezählt wird. Es wird jeden Schleifendurchlauf überprüft, ob  \lstinline{i < 5} gilt. Falls ja wird noch ein Schleifendurchlauf ausgeführt. falls nicht ist ist die Schleife beedet. Die obige Schleife wird insgesammt 5 mal ausgeführt.
 \end{Infobox}

\subsection*{c)}
Laufe nun mithilfe einer for-Schleife 10 Schritte.

\subsection*{d)}
Lass Neo danach folgendes tun:\\
	- drehe dich nach rechts\\
	- laufe 2 Schritte\\
	- drehe dich nach links\\ 
	- laufe 20 Schitte\\
	- hebe eine Münze auf\\
	- drehe dich nach rechts\\
	- laufe 12 Schritte\\
	- lasse die Münze fallen\\
	- laufe 4 Schritte

\subsection*{e)}
Lass Neo danach folgendes tun:\\
- drehe dich nach Links
- Lass Neo genau 12 Münzen aufheben. \\
- gehe 10 Schritte nach vorne.\\
- lass alle Münzen fallen.\\
- hebe genau 15 Münzen auf.\\
- gehe 5 Schritte nach vorne.\\
- lass eine Münze fallen.

\subsection*{f)}
Hebe alle Münzen auf dem aktuellen Feld auf. Laufe nun 5 Schritte und lege jeden Schritt so viele Münzen ab wie die schon Schritte gelaufen bist. Verwende für diese Aufgabe neoF.

\subsection*{g)}
 Laufe 10 Schritte und hebe dabei maximal 5 münzen pro Feld auf. Verwende für diese Aufgabe neoG.

\subsection*{h)}
Laufe 10 schritte und hebe dabei alle Münzen auf den Feldern auf Verwende für diese Aufgabe neoH.
\newpage
