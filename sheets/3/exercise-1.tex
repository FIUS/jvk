\excercise{For-Schleifen}

\begin{enumerate}
	\item Passe die Main Methode an die Aufgabe an. 
		Verwende \lstinline{Sheet3Task1} und \lstinline{Sheet3Task1Verifier}.
	\item
	Probiere Neo mithilfe einer While-Schleife zehn Schritte laufen zu lassen.\\
	Hinweiß: Wenn du nicht weißt wie du Neo genau 10 Schritte laufen lässt kannst du die Simulation mithilfe der Pause Taste stoppen nachdem Neo 10 Schritte gelaufen ist.
\end{enumerate}


\begin{Infobox}[For-Schleife]
	Mit einer For-Schleife kann man genau bestimmen wie of der Code in der Schleife ausgeführt wird.
	Anstatt dieselbe Operation n-mal untereinander zu schreiben um sie n-mal auszuführen, kann man den Code ein mal in eine For-Schleife die n-mal ausgeführt wird schreiben.
	Dazu verwenden wir eine for-Schleife. 

	\begin{lstlisting}
//here neo moves five times
neo.move();
neo.move();
neo.move();
neo.move();
neo.move();
	\end{lstlisting}
	
	ist equivalent zu:

	\begin{lstlisting}
//this moves neo five times
for (int i = 0; i < 5; i++) {
	neo.move();
}
	\end{lstlisting}

	\lstinline{int i = 0} setzt die Variable \lstinline{i} anfangs auf \lstinline{0}. 
	\lstinline{i++} heißt, dass \lstinline{i} jeden Schleifendurchlauf hochgezählt wird. 
	Es wird jeden Schleifendurchlauf überprüft, ob \lstinline{i < 5} gilt. 
	Falls ja wird noch ein Schleifendurchlauf ausgeführt, falls nicht ist ist die Schleife beedet. 
	Die obige Schleife wird insgesammt 5 mal ausgeführt.
\end{Infobox}


\begin{enumerate}\setcounter{enumi}{2}
	\item 
		Laufe nun mithilfe einer for-Schleife 10 Schritte ohne die Simulation zu pausieren

	\item
		% TODO: use Latex Lists
		Lass Neo danach folgendes tun:\\
			- drehe dich nach rechts\\
			- laufe 2 Schritte\\
			- drehe dich nach links\\ 
			- laufe 20 Schitte\\
			- hebe eine Münze auf\\
			- drehe dich nach rechts\\
			- laufe 12 Schritte\\
			- lasse die Münze fallen\\
			- laufe 4 Schritte


	\item
		% TODO: use Latex Lists
		Lass Neo danach folgendes tun:\\
		- drehe dich nach Links
		- Lass Neo genau 12 Münzen aufheben. \\
		- gehe 10 Schritte nach vorne.\\
		- lass alle Münzen fallen.\\
		- hebe genau 15 Münzen auf.\\
		- gehe 5 Schritte nach vorne.\\
		- lass eine Münze fallen.

	\item
		Verwende für diese Teilaufgabe den Spieler/Neo in der Variable \lstinline{neoF}.\\
		Hebe alle Münzen auf dem aktuellen Feld auf. \\
		Laufe nun 5 Schritte mit einer For-Schleife und lege jeden Schritt so viele Münzen ab wie die schon Schritte gelaufen bist.

		Hinweis: Dafür musst du eine For-Schleife in der For-Schleife verwenden. 
		Die Anzahl der Schritte die Du schon gelaufen bist ist in der (lauf-)Variable der äußeren For-Schleife.


	\item
		Verwende für diese Teilaufgabe den Spieler/Neo in der Variable \lstinline{neoG}.\\
		Laufe 10 Schritte und hebe dabei maximal 5 münzen pro Feld auf.

		Hinweis: In dieser Aufgabe musst du if-Statments verwenden.

	\item
		Verwende für diese Teilaufgabe den Spieler/Neo in der Variable \lstinline{neoH}.\\
		Laufe 10 Schritte und hebe dabei alle Münzen auf den Feldern auf.
\end{enumerate}
\newpage
