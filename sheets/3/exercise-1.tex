\excercise{For-Schleifen }

\begin{enumerate}
	\item Instanziiere die Simulation wie bekannt (\lstinline{Sheet3Task1} und \lstinline{Sheet3Task1Verifier}) und mache dich mit dieser vertraut.
	\item Probiere Totoro mithilfe einer While-Schleife 10 Schritte laufen zu lassen.\\

\end{enumerate}


\begin{Infobox}[For-Schleife]
	In Java kann man neben der While-Schleife auch eine For-Schleife benutzen.
	Mit einer For-Schleife kann man genau bestimmen wie oft der Code in der Schleife ausgeführt wird.\newline

	Anstatt eine Sequenz von Operationen $n$-mal untereinander zu schreiben, um sie $n$-mal auszuführen, kann man den repetitiven Code einmal in den Rumpf eine For-Schleife schreiben.
	Im \enquote{Kopf} der For-Schleife kann man dann angeben, wie oft der Schleifenrumpf ausgeführt werden soll.\newline

	Hier ein Beispiel:

	\begin{lstlisting}[numbers=none]
	// Move totoro five times
	totoro.move();
	totoro.move();
	totoro.move();
	totoro.move();
	totoro.move();
	// entspricht
	for (int i = 0; i < 5; i++) {
		totoro.move();
	}
	\end{lstlisting}

	Wie die While-Schleife hat auch die For-Schleife einen Schleifenrumpf, der mehrfach ausgeführt wird.
Der Inhalt der runden Klammern \lstinline{(int i = 0; i < 5; i++)} unterscheidet sich aber deutlich von dem einer While-Schleife.\newline

In den runden Klammern einer For-Schleife stehen insgesamt drei Anweisungen, die jeweils durch ein Semikolon ''\lstinline{;}'' voneinander getrennt sind.

Die erste Anweisung \lstinline{int i = 0} setzt die Zählvariable \lstinline{i} der For-Schleife auf den Anfangswert \lstinline{0}.
Der Name \enquote{\lstinline{i}} ist hierbei typisch für die Zählvariable einer for-Schleife.
Die erste Anweisung wird nur einmal vor Beginn der Schleife ausgeführt.\newline

Die zweite Anweisung (\lstinline{i < 5} in unserem Beispiel) ist vergleichbar mit der bool'schen Bedingung einer While-Schleife.
Sie wird jedes Mal überprüft, bevor der Schleifenrumpf ausgeführt wird.
Nur wenn sie den Wert \lstinline{true} zurückgibt, wird der Code-Block der Schleife ausgeführt.\newline

Der Ausdruck \lstinline{i < 5} überprüft hier, ob der Wert der Variable \texttt{i} größer (echt) kleiner als die Ganzzahl \texttt{5} ist.
Bei \texttt{<} handelt es sich also wie schon bei \texttt{+},\texttt{-},\texttt{/} und \texttt{*} um eine Operator - genauer einen Vergleichsoperator - der in den meisten Programmiersprachen vorhanden ist.
Weiter Vergleichsoperatoren sind \texttt{>} für \enquote{Ist $\langle$links$\rangle$ (echt) kleiner als $\langle$rechts$\rangle$?}, \texttt{<=}, \texttt{>=}, \texttt{!=} und \texttt{==}.
Genaueres erfährst du hierzu noch in Aufgabe 2.\newline

Die dritte Anweisung \lstinline{i++} wird nach jedem Durchlauf des Codes im Schleifenrumpf ausgeführt, bevor die Bedingung erneut überprüft wird.
\lstinline{i++} heißt, dass \lstinline{i} jeden Schleifendurchlauf um \lstinline{1} erhöht wird.
Das \lstinline{i++} ist also nichts anderes als eine Abkürzung für das Kommando \lstinline{i = i + 1}.\newline

Die obige Schleife wird 5 Mal ausgeführt.
Beachte, dass \lstinline{i} mit dem Wert \lstinline{0} beginnt und die Schleifenbedingung auf \lstinline{i < 5} und nicht \lstinline{i <= 5} (<= bedeuted i < 5 oder i == 5 gilt) prüft.
Im letzten Schleifendurchlauf hat \lstinline{i} den Wert \lstinline{4}, wird dann nach der Ausführung des Rumpfes um eins erhöht, wodurch die Bedingung zu \lstinline{flase} ausgewertet wird und der Rumpf nicht weiter ausgeführt wird.
\end{Infobox}


\begin{enumerate}\setcounter{enumi}{2}
	\item
	Laufe nun mithilfe einer For-Schleife 10 Schritte ohne die Simulation zu pausieren.

	\item
	Laufe mit Totoro jeweils 3, 7, 14 und 22 Schritte, indem du die Schleifenbedingung anpasst.
	Am Ende sollte Totoro auf einem Nusshaufen stehen.
	
	\item
	Hebe jetzt mit Totoro jeweils 2, 5, 16 und 20 Nüsse vom Feld \texttt{(22,0)} auf.
	Dazu musst du eine weitere For-Schleife verwenden, in der du auch den Schleifenrumpf anpasst.
	
	\item
	Verwende für diese Teilaufgabe den Totoro in der Variable \lstinline{totoroF}.
	
	Hebe alle Nüsse auf Totoro's Startposition auf.
	Laufe jetzt 5 Schritte mit einer For-Schleife und lege bei jedem Schritt so viele Nüsse ab, wie du schon Schritte gelaufen bist.
	
	\textbf{Hinweis:} Dafür musst du eine For-Schleife in der For-Schleife verwenden.
	Du kannst die Variable \lstinline{i} der äußeren For-Schleife in der Schleifenbedingung (an der 2. Stelle im Schleifenkopf) der inneren For-Schleife benutzen.
	Überlege dir, welcher Vergleichsoperator hier am besten passt.
	
	\item
	Verwende für diese Teilaufgabe den Totoro in der Variable \lstinline{totoroG}.
	
	Laufe 10 Schritte und hebe dabei maximal 5 Nüsse pro Feld auf.
	
	\item
	Verwende für diese Teilaufgabe den Totoro in der Variable \lstinline{totoroH}.
	
	Laufe 10 Schritte und hebe dabei alle Nüsse auf den Feldern auf.
	
	\textbf{Hinweis:} Hier musst du eine For- und eine While-Schleife verwenden.
\end{enumerate}
\newpage
