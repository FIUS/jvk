    \excercise{For-Schleifen}
\subsection*{a)}
	Passe die Main Methode an die Aufgabe an.
\subsection*{b)}

	Probiere Neo mithilfe einer While-Schleife genau zehn Schritte laufen zu lassen.

\begin{Infobox}[For-Schleife]
		Wir wollen nun anstatt die selbe Operation n-mal untereinander schreiben zu müssen um sie genau n-mal zu wiederholen, diesen Prozess vereinfachen.
		Dazu verwenden wir eine for-Schleife. 
		\begin{lstlisting}
public void moveFiveTimes() {
	this.move();
	this.move();
	this.move();
	this.move();
	this.move();
}
		\end{lstlisting}
		ist equivalent zu:
		\begin{lstlisting}
public void moveFiveTimes() {
	for (int i = 0; i < 5; i++) {
		this.move();
	}
}
		\end{lstlisting}
Eine For-Schleife funktioniert (zumindest in Java) wie eine While-Schleife mit einer zusätzlichen Variable die sich jeden Schleifendurchlauf verändert.
Die obige if Schleife ist z.B. äquivalent zu:
		\begin{lstlisting}
public void moveFiveTimes() {
	int i = 0;
	while (i < 5) {
		this.move();
		i++;
	}
}
		\end{lstlisting}
 \end{Infobox}

\subsection*{c)}
Laufe nun mithilfe einer for-Schleife 10 Schritte.

\subsection*{d)}
Lass Neo folgendes tun:\\
	- drehe dich nach rechts\\
	- laufe 5 Schritte\\
	- drehe dich nach links\\ 
	- laufe 20 Schitte\\
	- hebe eine Münze auf\\
	- drehe dich nach rechts\\
	- laufe 56 Schritte\\
	- lasse die Münze fallen

\subsection*{e)}
Lass Neo folgendes tun:\\
- Lass Neo genau 72 Münzen aufheben. \\
- gehe 10 Schritte nach vorne.\\
- lass alle Münzen fallen.\\
- hebe genau 42 Münzen auf.\\
- gehe 5 Schritte nach vorne.\\
- lass eine Münze fallen.

\subsection*{f)}
Hebe alle Münzen auf dem aktuellen Feld auf. Laufe nun 5 Schritte und lege jeden Schritt so viele Münzen ab wie die schon Schritte gelaufen bist.

\subsection*{g)}
 Laufe 10 Schritte und hebe dabei maximal 5 münzen pro Feld auf.

\subsection*{h)}
Laufe 10 schritte und hebe dabei alle Münzen auf den Feldern auf
\newpage
