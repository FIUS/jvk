\excercise{For-Schleifen}

\begin{enumerate}
	\item Passe die Main Operation an die Aufgabe an. 
		Verwende \lstinline{Sheet3Task1} und \lstinline{Sheet3Task1Verifier}.
	\item Probiere Neo mithilfe einer While-Schleife 10 Schritte laufen zu lassen.\\
	
		Hinweis: Wenn du nicht weißt, wie du Neo genau 10 Schritte laufen lässt, kannst du die Simulation mithilfe der Pause Taste stoppen nachdem Neo 10 Schritte gelaufen ist.
\end{enumerate}


\begin{Infobox}[For-Schleife]
	In Java kann man neben der While-Schleife auch eine For-Schleife benutzen.
	Mit einer For-Schleife kann man genau bestimmen wie oft der Code in der Schleife ausgeführt wird.
	Anstatt dieselbe Operation n-mal untereinander zu schreiben, um sie n-mal auszuführen, kann man den Code einmal in eine For-Schleife schreiben, die n-mal ausgeführt wird.
	Dazu verwenden wir eine For-Schleife. 

	\begin{lstlisting}[numbers=none]
//here neo moves five times
neo.move();
neo.move();
neo.move();
neo.move();
neo.move();
	\end{lstlisting}
	
	ist equivalent zu:

	\begin{lstlisting}[numbers=none]
//this moves neo five times
for (Integer i = 0; i < 5; i++) {
	neo.move();
}
	\end{lstlisting}

	Wie die While-Schleife hat auch die For-Schleife einen Schleifenrumpf, der mehrfach ausgeführt wird.
	Der Inhalt der runden Klammern \lstinline{()} unterscheidet sich aber deutlich von dem einer While-Schleife.

	In den runden Klammern einer For-Schleife stehen insgesamt drei Anweisungen, die jeweils durch ein Semikolon \lstinline{;} voneinander getrennt sind.
	Die erste Anweisung \lstinline{Integer i = 0} setzt die Variable \lstinline{i} auf den Anfangswert \lstinline{0}.
	Der Variablenname \lstinline{i} wird gerne für Schleifen verwendet und steht häufig für Index.
	Die erste Anweisung wird nur einmal vor Beginn der Schleife ausgeführt.
	
	Die zweite Anweisung \lstinline{i < 5} ist vergleichbar mit der Bedingung einer While-Schleife.
	Sie wird jedes mal überprüft, bevor der Schleifenrumpf ausgeführt wird.
	Nur wenn sie \lstinline{true} ist wird der Code-Block der Schleife ausgeführt.

	Die dritte Anweisung \lstinline{i++} wird nach jedem Schleifendurchlauf ausgeführt, bevor die Bedingung erneut überprüft wird.
	\lstinline{i++} heißt, dass \lstinline{i} jeden Schleifendurchlauf um \lstinline{1} erhöht wird. 
	
	Die obige Schleife wird 5 Mal ausgeführt.
	Beachte, dass \lstinline{i} mit dem Wert \lstinline{0} beginnt und die Schleifenbedingung auf \lstinline{i < 5} und nicht \lstinline{i <= 5} prüft.
	Im letzten Schleifendurchlauf hat \lstinline{i} den Wert \lstinline{4}.
\end{Infobox}


\begin{enumerate}\setcounter{enumi}{2}
	\item 
		Laufe nun mithilfe einer For-Schleife 10 Schritte ohne die Simulation zu pausieren.

		Hinweis: Dafür musst du die Schleifenbedingung, also die zweite Anweisung, der For-Schleife aus dem Beispiel anpassen.

	\item
		Laufe mit Neo jeweils 3, 7, 14 und 22 Schritte indem du die Schleifenbedingung anpasst.
		Am Ende sollte Neo auf einem Münzhaufen stehen.

	\item
		Hebe jetzt mit Neo jeweils 2, 5, 16 und 20 Münzen auf.
		Dazu musst du eine weitere For-Schleife benutzen, in der du auch den Schleifenrumpf anpasst.

	\item
		Verwende für diese Teilaufgabe den Spieler/Neo in der Variable \lstinline{neoF}.

		Hebe alle Münzen auf dem aktuellen Feld auf. 
		Laufe nun 5 Schritte mit einer For-Schleife und lege jeden Schritt so viele Münzen ab, wie du schon Schritte gelaufen bist.

		Hinweis: Dafür musst du eine For-Schleife in der For-Schleife verwenden. 
		Dafür musst du in der inneren For-Schleife einen anderen Variablennamen als \lstinline{i} verwenden.
		Du kannst die Variable \lstinline{i} der äußeren For-Schleife in der Schleifenbedingung der inneren For-Schleife benutzen.

	\item
		Verwende für diese Teilaufgabe den Spieler/Neo in der Variable \lstinline{neoG}.

		Laufe 10 Schritte und hebe dabei maximal 5 Münzen pro Feld auf.

		Hinweis: In dieser Aufgabe musst du if-Statements verwenden.

		\begin{lstlisting}
if (neoG.canCollectCoin()) {
    neoG.collectCoin();
}
		\end{lstlisting}

	\item
		Verwende für diese Teilaufgabe den Spieler/Neo in der Variable \lstinline{neoH}.

		Laufe 10 Schritte und hebe dabei alle Münzen auf den Feldern auf.

		Hinweis: Hier musst du eine For- und eine While-Schleife verwenden.
\end{enumerate}
\newpage
