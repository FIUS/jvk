\excercise{If-Verzweigungen 2}

\begin{enumerate}
	\item
	Instanziiere die Simulation wie bekannt (\lstinline{Sheet3Task2} und \lstinline{Sheet3Task2Verifier}) und mache dich mit dieser vertraut.

	\item
		Mit dem folgenden Codebeispiel kannst du die Anzahl der Nüsse in dem Feld unter Totoro abfragen und auf der Konsole ausgeben.

		\begin{lstlisting}
int nutsUnderTotoro = totoro.getCurrentlyCollectableNuts().size();
System.out.println(nutsUnderTotoro);
		\end{lstlisting}

		Laufe mit Totoro bis zum ersten Feld mit einer Nuss und gib die Anzahl der Nüsse unter Totoro auf der Konsole aus.\newline

		\textbf{Wichtig}: Mit der Operation \lstinline{totoro.getCurrentlyCollectableNuts();} bekommt man eine Liste aller Nuss-Objekte, die auf Totoro's aktuellem Feld liegen.
		Du kannst die den Datentyp List wie eine Sequenz (Aneinanderreihung) von Objekten desselben Typs - in diesem Fall des Typs \lstinline{Nut} - vorstellen.
		Eine Liste hat eine Länge die angibt wie viele Elemente in ihr enthalten sind.
		Diese Länge kann mit der \lstinline{size()}-Methode auf Listen-Objekten abgefragt werden.
\end{enumerate}


\begin{Infobox}[If-Statements 2]

	If-Statements und die Vergleichsoperatoren (wie z.B. \lstinline{<=}) kennst du ja schon.
Diese können wie schonmal erwähnt zusammen mit primitiven Datentypen wie Ganzzahlen (\lstinline{int}s) verwendet werden.\newline

Hier ein Beispiel:

\begin{lstlisting}[numbers=none]
int x = 2;
if ( x < 3 ) {
System.out.println("x ist kleiner als 3!");
}
\end{lstlisting}

Das kann man zum Beispiel dazu verwenden, um zu schauen, wie viele Nüsse Totoro hat.
In Java kannst du die folgenden Vergleichsoperatoren verwenden:
	\begin{center}
		\begin{tabular}{ c | c | c | l }
			Natürliche Sprache & Mathematisches Zeichen & Java Operator \\
			\hline
			größer als & $>$ & \texttt{>} \\
			kleiner als & $<$ & \texttt{<} \\
			gleich & $=$ & \texttt{==}\\
			ungleich & $\neq$ & \texttt{!=} \\
			größer gleich & $\geq$ & \texttt{>=} \\
			kleiner gleich & $\leq$ & \texttt{<=} \\
        \end{tabular} \\
	\end{center}

	Damit können wir nun prüfen, ob sich auf Totoro's Feld genau 3 Nüsse befinden:

	\begin{lstlisting}[numbers=none]
if ( totoro.getCurrentlyCollectableNuts().size() == 3 ){
	System.out.println("Totoro hat genau 3 Nüsse!");
}
	\end{lstlisting}
\end{Infobox}


\begin{Infobox}[\lstinline{==} und die \lstinline{equals()}-Methode]
	\textbf{Vorsicht:} Mit dem Vergleichsoperator \enquote{\lstinline{==}} sollte man in der Regel nur primitive Datentypen (= etwa Werte vom Typ \lstinline{int}, Fließkommazahlen wie \lstinline{double} und ein paar andere Typen, die wir der Einfachheit halber hier weglassen) vergleichen!\newline

	Zum Vergleichen von zwei Objekten sollte immer die \lstinline{equals}-Methode, die auf jedem Objekt definiert ist, verwendet werden:
	
	\begin{lstlisting}[numbers=none]
	if (totoro.getPosition().equals(new Position(1, 1))){
	System.out.println("Totoro is on Position (1, 1)");
	}
	\end{lstlisting}
	
	Mit dem folgenden Codebeispiel kannst du in der Konsole sehen, dass \lstinline{==} und \lstinline{.equals} nicht zu dem gleichen Ergebnis kommen müssen, wenn man zwei Objekte vergleichen will.

	\begin{lstlisting}[numbers=none]
System.out.println(new Position(1,1) == new Position(1,1)); // false
System.out.println(new Position(1,1).equals(new Position(1,1))); // true
	\end{lstlisting}

\end{Infobox}


\begin{enumerate}\setcounter{enumi}{2}
	\item
	 	Nun wollen wir daran arbeiten möglichst viele Nüsse auf dem Spielfeld einzusammeln.

		Dazu wollen wir Totoro erstmals so lange geradeaus laufen lasse, bis sich unter ihm eine oder mehrere Nüsse befinden.
		Wenn Totoro auf eine Nuss trifft, sollst du erst mal anhalten.
\end{enumerate}


\begin{Infobox}[Endlosschleifen und \lstinline{break}]
	Eine While-Schleife, in der die Schleifenbedingung \lstinline{true} ist, nennt man eine Endlosschleife.
	Das heißt, dass die Schleife niemals aufhören wird den Code im Schleifenrumpf auszuführen, wenn das nicht durch eine Exception oder den Stop der Ausführung des Programms verhindert wird!
	Deshalb kann es passieren, dass du den Stopp Knopf in der Eclipse Konsole brauchst, um die Schleife wieder abzubrechen.\newline

	Endlosschleifen möchte man in der Regel wann immer möglich vermeiden.
	Aber man kann auch eine \lstinline{while(true)}-Schleife planmäßig im Quellcode und ohne das Auslösen von Exceptions beenden.
	Dafür braucht man das Schlüsselwort \lstinline{break}.\newline

	Wenn du im Rumpf einer Schleife \lstinline{break;} aufrufst, dann wird die Schleife sofort unterbrochen.
	Auch die Schleifenbedingung wird dann nicht mehr überprüft und der Code nach der Schleife wird weiter ausgeführt.
	Das funktioniert auch in For-Schleifen, ist aber hier eher untypisch, da man dort meistens die Zählvariablen zur Begrenzung der Wiederholungen benutzt.

	\begin{lstlisting}[numbers=none]
	while(true){
		// do something
		if (hadEnough) {
			break;
		}
	}
	\end{lstlisting}

\end{Infobox}


\begin{enumerate}\setcounter{enumi}{3}
	\item
	Mit dem neuen Wissen wollen wir nun an unserem Code aus Teilaufgabe c) weiterarbeiten.

	Wenn Totoro also auf einem Feld mit genau einer Nuss steht, soll er die Nuss aufsammeln, sich dann nach rechts drehen und einen weiteren Schritt machen.
	Wenn sich mehr als eine Nuss unter Totoro befinden, soll er die Schleife erstmals mit einem \lstinline{break} verlassen.
	Dann soll der wieder anfangen zu laufen, bis unter ihm wieder eine oder mehrere Nüsse auftauchen.
	
	\item
	Wenn sich jetzt mehr als eine Nuss unter Totoro befinden sollten, dann soll Totoro genau eine Nuss aufsammeln, sich nach links drehen und dann wieder einen Schritt nach vorne machen.
	
	Momentan soll Totoro noch nicht alle Nüsse aufgesammelt bekommen, ohne gegen einen Busch zu laufen.
	\end{enumerate}
	
	\begin{Infobox}[Logische Operatoren]
	
	Wenn mehrere Bedingungen (also bool'sche Werte) für ein Ereignis eine Rolle spielen, muss man diese logisch verknüpfen.
	Es folgt eine Auflistung der häufigsten Operatoren:

	\begin{center}
		\begin{tabular}{ c | c | c | l}
			Natürliche Sprache & logische Operatoren & Java Operatoren & Bemerkung \\
			\hline
			und  & $\wedge$ & \texttt{\&\&} & \minibox{Das einfache \texttt{\&} hat eine\\etwas andere Funktion} \\
			oder & $\vee$ & \texttt{||} & Dasselbe gilt für \texttt{|} \\
			nicht & $\neg$ & \texttt{!} &\\
		\end{tabular}\\
	\end{center}

	Hier ein paar Beispiele der Anwendung dieser in Java-Verzweigungen.
	Natürlich kann man die logischen Operatoren aber auch in while- oder for-Schleifeköpfen, oder sonst wo verwenden:

	\begin{lstlisting}[numbers=none]
if (totoro.canMove() && totoro.getCurrentlyCollectableNuts().size() == 7) {
	//totoro hat genau 7 Nüsse UND kann nach vorne gehen.
}
	\end{lstlisting}

	\begin{lstlisting}[numbers=none]
if (!totoro.canMove() && totoro.getCurrentlyCollectableNuts().size() == 7) {
	//totoro hat genau 7 Nüsse UND kann sich NICHT nach vorne gehen.
}
	\end{lstlisting}

	\begin{lstlisting}[numbers=none]
if (totoro.canMove() || totoro.getCurrentlyCollectableNuts().size() == 7) {
	//totoro hat genau 7 Nüsse ODER kann nach vorne gehen.
}
	\end{lstlisting}

	Zu beachten ist außerdem, dass das \q{oder} \lstinline{||} kein ausschließendes \q{oder} ist, daher ist \lstinline{((2 < 3) || (42 != 4))} eine wahre Aussage.

\end{Infobox}


\begin{enumerate}\setcounter{enumi}{5}
	\item
	Bevor Totoro gegen einen Busch läuft, soll er sich umdrehen.
	Dafür musst du vor jedem \lstinline{move()} Kommando prüfen, ob Totoro sich überhaupt nach vorne bewegen kann.
	
	\item
	Jetzt wollen wir, dass Totoro auch irgendwann aufhört.
	Er soll aufhören zu laufen oder Nüsse einzusammeln, wenn er schon 20 Nüsse gesammelt hat oder wenn er auf einem Feld mit exakt 9 Nüssen ankommt.
	Mit \lstinline{totoro.getNutsInPocket()} kannst du die Anzahl der Nüsse in Totoro's Inventory abfragen.
	
	\textbf{Tipp:} Hierfür kannst du entweder das Argument der Endlosschleife ändern oder \lstinline{break} Verwenden.
	Falls du die Entscheidung, was Totoro auf einem Feld machen soll, mit einem \lstinline{if-else}-Verzweigung implementiert hast, kannst du hier ein \lstinline{else if(...)} einführen.
\end{enumerate}
\newpage
