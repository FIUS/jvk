    \excercise{Schleifen}
            \subexcercise Schreibe deine \texttt{turnLeft()} so um, dass sie eine for-Schleife verwendet.

            \subexcercise Neo muss so schnell wie möglich die nächstgelegene Telefonzelle erreichen, um die Matrix zu verlassen. Leider weiß er nicht genau, wie weit diese entfernt ist. Schreibe hierzu eine Operation, die Neo so lange geradeaus laufen lässt, bis die Operation \texttt{Neo.isOnPhoneBooth()} den Wert \texttt{true} zurück gibt. Verwende in deiner Lösung eine while-Schleife.

            \subexcercise(\textbf{Bonus}) Schreibe die Operation aus \textit{(b)} so um, dass sie eine do-while-Schleife verwendet.

            \subexcercise Dieses Mal haben die Agenten alle Telefonzellen in Neos Nähe zerstört. Welche Auswirkung hat das auf deinen Code aus \textit{(b)}? Nutze die Operation \texttt{Neo.canMove()}, um den Fehler zu beheben.

            \textbf{Für diese Aufgabe} müsst ihr \texttt{this.boothsDestroyed} auf \texttt{true} setzen.

