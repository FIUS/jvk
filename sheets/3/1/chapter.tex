% !TeX root = ../jvk-blatt3.tex

\vkchapter{Exceptions}

In diesem Kapitel werden Exceptions erklärt. Desweiteren werdet ihr lernen Stacktraces zu lesen und auch eigene Exceptions zu werfen. 

\begin{Infobox}[Exception]
    Eine Exception ist ein Fehler, der den normalen Ablauf stört. Diese treten dann auf, wenn die Korrektheit des Programmes nicht mehr gewährleistet werden kann. Als Beispiel wird
    \begin{lstlisting}[breaklines=true, numbers=none]
        Integer a = 1;
        Integer b = 0;
        System.out.println(a/b);
    \end{lstlisting}
    folgende Fehlermeldung: \lstinline{Exception in thread "main" java.lang.ArithmeticException: / by zero at Main.main Main.java:9} ausgeben, weil eine Division durch null in der Mathematik nicht definiert ist.

    Desweiteren kann ein Programmierer auch dem Programm sagen, dass eine Exception gewerfen wird, wenn die Eingabe nicht korrekt ist.
    Eine Exception wird mit \lstinline{throw} new Exception() geworfen. 

\end{Infobox}

\begin{Infobox}[Stacktrace]
    Als ein Stacktrace wird die Ausgabe einer Konsole bezeichnet, wenn eine Exception geworfen wird. 
    Auf dem Stacktrace werden Addressen hinterlegt, die im Programm aufgerufen wurden. Desweiteren wird auch dargelegt, um welche Art einer Exception es sich handelt.
\end{Infobox}

\addexcercise

\addexcercise