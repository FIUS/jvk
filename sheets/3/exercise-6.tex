\excercise{What is \lstinline{this}?}

% TODO Aufgabe verbesser!

\begin{enumerate}
	\item 
		Passe die Main Operation an die Aufgabe an.\\
		Verwende \lstinline{Sheet3Task6} und \lstinline{Sheet3Task6Verifier}.
		
	\item 
		Versuche die Operation aus Aufgabe 4 d) aufzurufen.

		Hinweis: Dein Versuch darf auch schiefgehen.
		Es ist zwar möglich, aber nicht auf direktem Weg.
		Du kannst also nicht in der Klasse \lstinline{Sheet3Task6} einfach den Namen einer Operation aus einer anderen Klasse schreiben und erwarten, dass Java schon das richtige macht.
		
	\item 
		Kopiere die komplette Operation aus Aufgabe 4 d) in die Neo Klasse und versuche erneut diese aufzurufen. 
		Achte darauf, dass die Operation \lstinline{public} ist.
		Rufe nun diese Operation in \lstinline{Sheet3Task6} auf mit \lstinline{neo.<dein Operationsname>(neo, 5)}.
	
	\item
		Bei dem Operationsaufruf in c) kommt die Variable \lstinline{neo} zwei mal vor.
		Solche dopplungen kann man fast immer vermeiden.
		
		Gehe wieder in die Neo Klasse und lösche den Parameter \lstinline{(Neo neo)}.
		Die runden Klammern must du dabei behalten.

		Ersetze alle Vorkommen in der Operation von \lstinline{neo.} mit \lstinline{this.}.

		Jetzt musst du noch deinen Aufruf der Operation anpassen, da die Operation keinen Parameter mehr benötigt.
\end{enumerate}


\begin{Infobox}[\lstinline{this}]
	In Blatt 2 hast du erfahren, dass Java den Wert von \lstinline{this} automatisch festlegt.
	Jetzt können wir erklären, wie das passiert.
	Dafür nehmen wir die folgenden Zwei Klassen:

	\begin{lstlisting}[xleftmargin=0.5cm]
public class Neo extends Human {

    public void turnAround() {
        this.turnClockWise();
        this.turnClockWise();
    }
}
	\end{lstlisting}

	\begin{lstlisting}[xleftmargin=0.5cm]
public class Sheet3Task6 implements Task {
    
    public void run(Simulation sim) {
        Neo neo = new Neo();
        // other stuff
        
        neo.turnAround();
    }
}
	\end{lstlisting}

	In der \lstinline{run(/*...*/)} Operation der Klasse \lstinline{Sheet3Task6} wird in Zeile 7 \lstinline{neo.turnAround();} aufgerufen.
	In der Variable \lstinline{neo} ist das in Zeile 4 erstellte Objekt der Klasse \lstinline{Neo} gespeichert.
	In Zeile 7 wird also auf diesem Neo Objekt die Operation \lstinline{turnAround()} aufgerufen.

	Die Operation \lstinline{turnAround()} ist in der Klasse des Neo Objekts, also in der Klasse \lstinline{Neo} in Zeile 3 definiert.
	Java setzt für diesen einen Aufruf der Operation \lstinline{turnAround()} auf dem in \lstinline{Sheet3Task6} Zeile 4 erzeugten Neo Objekt \lstinline{this = Sheet3Task6::run:neo}.
	(Die Notation \lstinline{Sheet3Task6::run:neo} ist frei erfunden und soll nur deutlich machen, dass es sich um das Objekt aus der Variable \lstinline{neo} aus der Operation \lstinline{run} der Klasse{Sheet3Task6} handelt.)

	In den meisten Fällen kann man vereinfacht sagen: das \lstinline{this} in einer Operation ist das, was beim Aufruf der Operation vor dem Punkt steht.
	Allerdings muss man beachten, das Variablen selber keine Objekte sind, sondern nur ein Name (oder eine Adresse) mit dem man ein Objekt mehrfach benutzen kann.
	Der Unterschied von Variablen und Objekten wird auch in PSE nochmal genauer erklärt.

\end{Infobox}


\begin{enumerate}\setcounter{enumi}{4}
	\item \optional Welche Operationen kannst du mit \lstinline{this.} aufrufen und welche mit \lstinline{playerA.}? 
		Kannst du unterschiedliche Operationen aufrufen, wenn ja welche?

	\item 
		Gehe zu Aufgabe 4 zurück und verwende ebenfalls die Operation die auf Neo definiert ist.
	\item \optional Versuche nochmal b) zu lösen.

		Hinweis: Du brauchst ein Objekt der Klasse um einer Operation der Klasse aufrufen zu können.
\end{enumerate}
