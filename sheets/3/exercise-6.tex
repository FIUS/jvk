\excercise{Collections}
\subsection*{a)}
	Passe die Main Methode an die Aufgabe an. Verwende \lstinline{Sheet3Task6} und \lstinline{Sheet3Task6Verifier}.
\subsection*{b)}
	Versuche die Operation aus Aufgabe 4 d) aufzurufen.
\subsection*{c)}
	Kopiere die komplette Operation aus Aufgabe 4 d) in die Neo Klasse und versuche erneut diese aufzurufen. Achte darauf, dass die Operation \lstinline{public} ist.
	rufe nun diese Operation in \lstinline{Sheet3Task6} auf mit \lstinline{neo.[dein Operationsname](neo, 5)}
\subsection*{d)} Bei dem Operationsaufruf in c) kommt neo zwei mal vor. Dies kann man vermeiden. Gehe wieder in die Neo Klasse und lösche den Parameter (Neo neo). Ersetze alle Vorkommen von \lstinline{neo} mit \lstinline{this}.
\subsection*{e)(Optional)}Welche Methoden kannst du mit this. aufrufen und welche mit playerA. gibt es unterschiede, wenn ja welche?
\subsection*{f)}Gehe zu Aufgabe 4 zurück und verwende ebenfalls die Methode die auf Neo definiert ist.