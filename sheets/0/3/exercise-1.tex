% !TeX root = ../jvk-blatt0.tex

\excercise{IntelliJ installieren}
\label{ex1}
Gehe zuerst auf die Website \href{intellijurl}{jetbrains.com/idea/download/} und lade dir die IntelliJ IDEA Community Edition herunter. Auf dem Bild siehst du die Website und den zu klickenden ''download'' Button für Windows. Für Mac und Linux wird das ''.exe'' durch den jeweils passenden Dateityp ersetzt, dies sollte euer Internet Browser automatisch tun.
\begin{center}
    \includegraphics[width=\linewidth]{./3/IntelliJ download site.PNG}
\end{center}
\textbf{Für Windows und Mac:}\newline
Starte nun das Programm das du gerade installiert hast. Es wird sich ein Fenster öffnen indem du den Ort wo su IntelliJ hin installierst, ob du einen Desktop Shortcut zu IntelliJ willst, usw. wählen kannst. Lade in diesem Fenster IntelliJ mit den von dir gewünschten Einstellungen runter.\\\\

\textbf{Für Linux:}\newline
Schaue dir an wie man IntelliJ für deine Distribution installiert.

