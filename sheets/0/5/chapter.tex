% !TeX root = ../jvk-blatt0.tex

\vkchapter{Simulator Tutorial}
In dieser Aufgabe werden wir dir einige der wichtigsten Elemente der UI von dem Simulator vorstellen.
\begin{center}
    \includegraphics[width=\linewidth]{./5/Simulator Erklärung.PNG}
\end{center}
\begin{itemize}
    \item \textbf{run:} startet die Simulation
    \item \textbf{stop:} stopt die Simulation
    \item \textbf{Schrittweise run:} lässt die Simulation einen Schritt weiter laufen
    \item \textbf{Geschwindigkeit:} Regler für die Simulationsgeschwindigkeit
    \item \textbf{Spielfeld:} Spielfeld auf dem sich die Objekte befinden
    \item \textbf{Objekte auf Spielfeld:} sortierte Liste aller Objekte auf dem Spielfeld
    \item \textbf{Konsole:} Konsolenausgabe des Programmes
\end{itemize}
\newpage
\begin{center}
    \includegraphics[width=\linewidth]{./5/Simulator Erklärung 2.PNG}
\end{center}
\begin{itemize}
    \item \textbf{Objekte:} Objekte die sich auf dem Spielfeld befinden
    \item \textbf{manuell Objekt erstellen/löschen:} man kann Objekte durch klicken auf die gewünschte Position auf das Spielfeld platzieren/löschen
    \item \textbf{Objektauswahl:} Objekt zum erstellen/löschen wählen
    \item \textbf{Verifier:} Prüft ob Aufgaben korrekt gelöst wurden
    \item \textbf{Refresh Button:} Überprüft nochmal ob Aufgaben korrekt gelöst wurden (\textit{Hinweis:} der Refresh Button sollte nach dem Lösen jeder Teilaufgabe betätigt werden)
\end{itemize}
\textit{Hinweis:} nachdem das Simulationsfenster geschlossen wird muss man immer auch noch in IntelliJ das Programm stoppen.
