% !TeX root = ../jvk-blatt0.tex

\vkchapter{IntelliJ Tutorial}
In dieser Aufgabe werden wir dir einige der wichtigsten Elemente der UI von IntelliJ vorstellen.
\begin{center}
    \includegraphics[width=\linewidth]{./4/IntelliJ Erklärung.PNG}
\end{center}
\begin{itemize}
    \item \textbf{run Button:} durch drücken auf diesen Button wird das Programm gestartet
    \item \textbf{run Einstellungen:} dort kann eingestellt werden, was wie gestartet wird
    \item \textbf{Einstellungen:} durch über dieses Feld hovern öffnet sich ein auf dem folgendem Bild sichtbarer Teil
    \item \textbf{Klassen:} hier steht der Code drinnen
    \item \textbf{Kommentar:} erklärt den Code der unter ihm steht
    \item \textbf{Package Explorer:} dort sind alle Dateien aus dem Projekt sicht- und öffenbar
\end{itemize}
\newpage
Auf dem zweiten Bild sieht man den Stand von IntelliJ nachdem man über das Einstellungen Feld hovert und ein Programm gestartet, aber noch nicht beendet hat.
\begin{center}
    \includegraphics[width=\linewidth]{./4/IntelliJ Erklärung 2.PNG}
\end{center}
\begin{itemize}
    \item \textbf{stop Button:} beendet das Programm
    \item \textbf{run Button:} startet das Programm neue
    \item \textbf{speichern:} speichert das Projekt
    \item \textbf{Konsole:} beinhaltet die Ausgabe von Code (ist für uns meistens nicht wichtig, da wir im Simulationsfenster arbeiten)
    \item \textbf{Einstellungen:} hier kann man viele Dinge einstellen, aber auch z.B. kopieren und einfügen Befehle ausführen
\end{itemize}
\newpage
\section*{Nützliche Shortcuts}
\begin{itemize}
    \item \textbf{Code umformatieren:} Strg+Alt+L (sorgt dafür, dass Einrückungen usw. im Code ordentlich und einheitlich sind)
    \item \textbf{Speichern:} Strg+S
    \item \textbf{Kopieren:} Strg+C 
    \item \textbf{Einfügen:} Strg+V
    \item \textbf{Ausschneiden:} Strg+X
    \item \textbf{Rückgängig machen:} Strg+Z
    \item \textbf{Autocompletion für Operationen auf Objekten:} Strg+Leer
\end{itemize}