    \excercise{Neo im Labyrinth - Teil 2 (Polymorphie)}

    Nachdem Neo aus dem Labyrinth entkommen ist, fragt er sich, ob er nicht schneller gewesen wäre, wenn er der linken Wand gefolgt wäre.

        \subexcercise Implementiere die Operationen \lstinline{checkSideWall()} und \lstinline{moveSmart()} der Klasse \lstinline{MyLefthandedSmartNeo} entsprechend.
        \subexcercise Wie musst du die \lstinline{Solution4_2} Klasse verändern, damit Neo an der linken Wand entlang läuft?

        \subexcercise Schau dir noch einmal alle Neo-Klassen und ihre Vererbung an.
        Objekte welcher Klassen kannst du dem Attribut \lstinline{neo} der Klasse \lstinline{Task4_2} zuweisen?
        Was bewirkt der Aufruf von \lstinline{neo.checkSideWall()} jeweils?


\titledquestion{Escape}
Hilf Neo aus der Matrix zu entkommen. Als erstes muss Neo eine Pille einsammeln. Wenn diese Pille blau ist, soll die Münzenstapel von niedrig (links) nach hoch (rechts) sortieren. Wenn die gefundene Pille aber rot ist soll er es genau anders herum machen.
Wenn er die Aufgabe erledigt hatt wird das Telefon klingeln und Neo kann die Matrix verlassen.

\titledquestion{}
Jeder Neo muss eine die logische Formel
$$
(A \land B) \lor ((\neg D) \land C \lor B)
$$
lösen.
A, B, C ,D entsprechen den Feldern x=1 bis x=4. Wenn eine Münze auf einem Feld liegt besitzt das Feld den Wert True, ansonsten False.

Trage das Ergebnis in x=6 ein (Münze oder keine Münze)
