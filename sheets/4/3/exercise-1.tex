\excercise{Verebung}
\begin{enumerate}
	\item
	Instanziiere die Simulation (\lstinline{Sheet3Task6} und  \lstinline{Sheet3Task6Verifier}) und mache dich mit dieser vertraut.
\end{enumerate}

\begin{Infobox}[Vererbung]


Bei der Vererbung spricht man von sogenannten Elternklassen (manchmal auch Superklassen genannt) und ihren Kindklassen (Subklassen).
Die Kindklassen \enquote{erben} manche der Funktionen und Attribute der Elternklasse und können diese somit genauso verwenden wie ihre \enquote{Eltern}.\newline

Kindklassen können die geerbten Funktionen aus der Elternklasse auch überschreiben und somit zum Beispiel in einer Subklasse \lstinline{SpeedyNeo} von \lstinline{Neo} die \lstinline{move}-Operation überscheiben um den \lstinline{SpeedyNeo} zwei statt einen Schritt pro Ausführung der Methode laufen zu lassen.

Im Code wird Vererbung durch das Schlüsselwort ''extends'' erzeugt:

\begin{lstlisting}[xleftmargin=0.5cm]
public class GreedyNeo extends Neo {
	// Neue Methoden oder Methoden, die GreedyNeo von Neo
	// überschreiben soll...
}
\end{lstlisting}

Das Beispiel sagt also aus, dass \lstinline{SpeedyNeo} alle Funktionen von Neo verwenden kann.\newline
Kindklassen können Funktionen ihrer Elternklasse mit der Annotation \lstinline{@Override} über der Neu-Definition der Methode in der Kindklasse überschreiben.

\begin{lstlisting}[xleftmargin=0.5cm]
public class Neo {
	public void collectCoin() {
		// ...
	}
}

public class GreedyNeo extends Neo {
	@Override
	public void collectCoin() {
		// ...
	}
}
\end{lstlisting}
Die ursprüngliche Funktion der Elternklasse die von der Kindklasse überschrieben wurde kann mit dem Schlüsselwort \lstinline{super} aufgerufen werden:
\begin{lstlisting}[xleftmargin=0.5cm]
public class FastNeo extends Neo {
	//fast Neo moves two spaces every time move() is called
	@Override
	public void move(){
		super.move();
		super.move();
	}
}
\end{lstlisting}
\end{Infobox}


\begin{enumerate}\setcounter{enumi}{1}
\item Schreibe nun die Klasse \lstinline{GreedyNeo} aus dem \texttt{de.unistuttgart.informatik.fius.jvk.provided.}\\\texttt{entity}-Paket so um, dass sie eine Kindklasse der Klasse Neo ist.

Initialisiere danach ein \lstinline{GreedyNeo} in \lstinline{Sheet3Task6} und schau dir die verfügbaren Operationen an.

\item Überschreibe nun die \lstinline{collectCoin()} Funktion der \lstinline{GreedyNeo} Klasse, sodass \lstinline{GreedyNeo} immer alle Münzen auf einem Feld einsammelt, wenn die Funktion aufgerufen wird.
Teste diese Funktionalität indem du in \lstinline{Sheet3Task6} 5 Münzen auf \lstinline{GreedyNeo} Feld platzierst und einmal \lstinline{collectCoin()} ausführst.

\textbf{Tipp:} schaue die dazu die \lstinline{collectCoin()} Funktion der \lstinline{Neo} Klasse genauer an.

\item Nun wollen wir Morpheus in die Matrix laden.

Außerdem wollen wir Morpheus' \lstinline{collectCoin()} Funktion so abändern, dass er einfach nichts macht, falls sich keine Münze unter ihm befindet.
Spezieller soll dabei auch keine Fehlermeldung auf der Konsole erscheinen.
Gehe dazu analog zu der letzten Teilaufgabe vor.

Teste diese Funktionalität, indem du in \lstinline{Sheet3Task6} keine Münzen auf \lstinline{Morpheus}' Feld platzierst und einmal \lstinline{collectCoin()} ausführst.

\item Morpheus soll in seiner \lstinline{move()} Funktion nun automatisch Wände erkennen und nicht einfach in sie hineinzulaufen.
Stattdessen soll Morpheus prüfen, ob das Feld links, rechts oder hinter ihm frei ist und in die entsprechende Richtung weiterlaufen.

Teste diese Funktionalität, indem du in der\lstinline{Sheet3Task6}-Klasse eine Wand auf dem Feld vor dem \lstinline{Morpheus}-Objekt platzierst und ein paar angrenzende Felder blockierst.
Was passiert, wenn Du Morpheus komplett einschließt und die \lstinline{move}-Operation ein paar Male ausführst?

\item \optional Dir ist vielleicht aufgefallen, dass Morpheus in der Simulation anders aussieht als \lstinline{Neo}, \lstinline{GreedyNeo} allerdings aussieht wie \lstinline{Neo}.
Finde den Codeteil der dieses Verhalten bewirkt.

Im Projekt ist noch eine Mario Textur hinterlegt.
Versuche die Textur von GreedyNeo durch die von Mario zu ersetzten.

\end{enumerate}
