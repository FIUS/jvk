\excercise{Schleifen und IF}

\begin{enumerate}
	\item
	Instanziiere die Simulation wie bekannt (\lstinline{Sheet3Task2} und \lstinline{Sheet3Task2Verifier}) und mache dich mit dieser vertraut.

	\item
		Mit dem folgenden Codebeispiel kannst du die Anzahl der Münzen in dem Feld unter Neo abfragen und auf der Konsole ausgeben.

		\begin{lstlisting}
int coinsUnderNeo = neo.getCurrentlyCollectableCoins().size();
System.out.println(coinsUnderNeo);
		\end{lstlisting}

		Laufe mit Neo bis zum ersten Feld mit einer Münze und gib die Anzahl der Münzen unter Neo auf der Konsole aus.\newline

		\textbf{Wichtig}: Mit der Operation \lstinline{neo.getCurrentlyCollectableCoins();} bekommt man eine Liste aller Münzen-Objekte, die auf Neo's aktuellem Feld liegen.
		Du kannst die den Datentyp List wie eine Sequenz (Aneinanderreihung) von Objekten desselben Typs - in diesem Fall des Typs \lstinline{Coin} - vorstellen.
		Eine Liste hat eine Länge die angibt wie viele Elemente in ihr enthalten sind.
		Diese Länge kann mit der \lstinline{size()}-Methode auf Listen-Objekten abgefragt werden.

	\item
	 	Nun wollen wir daran arbeiten möglichst viele Münzen auf dem Spielfeld einzusammeln.

		Dazu wollen wir Neo erstmals so lange geradeaus laufen lasse, bis sich unter ihm eine oder mehrere Münzen befinden.
		Wenn Neo auf eine Münze trifft, sollst du erst mal anhalten.

	\item
		Mit dem neuen Wissen wollen wir nun an unserem Code aus Teilaufgabe c) weiterarbeiten.

		Wenn Neo also auf einem Feld mit genau einer Münzen steht, soll er die Münze aufsammeln, sich dann nach rechts drehen und einen weiteren Schritt machen.
		Wenn sich mehr als eine Münze unter Neo befinden, soll er die Schleife erstmals mit einem \lstinline{break} verlassen.
		Dann soll der wieder anfangen zu laufen, bis unter ihm wieder eine oder mehrere Münzen auftauchen.

	\item
		Wenn sich nun mehr als eine Münze unter Neo befinden sollten, dann soll Neo genau eine Münze aufsammeln, sich nach links drehen und dann wieder einen Schritt nach vorne machen.

		Momentan soll Neo noch nicht alle Münzen aufgesammelt bekommen, ohne gegen eine Wand zu laufen.

	\item
		Bevor Neo gegen eine Wand läuft soll er sich umdrehen.
		Dafür musst du vor jedem \lstinline{move()} Kommando prüfen, ob Neo sich überhaupt nach vorne bewegen kann.

	\item
		Jetzt wollen wir, dass Neo auch irgendwann aufhört.
		Er soll aufhören zu laufen oder Münzen einzusammeln, wenn er schon 20 Münzen gesammelt hat oder wenn er auf einem Feld mit exakt 9 Münzen ankommt.
		Mit \lstinline{neo.getCoinsInWallet()} kannst du die Anzahl der Münzen in Neo's Inventory abfragen.

		\textbf{Tipp:} Hierfür kannst du entweder das Argument der Endlosschleife ändern oder \lstinline{break} Verwenden.
		Falls du die Entscheidung, was Neo auf einem Feld machen soll, mit einem \lstinline{if-else}-Verzweigung implementiert hast, kannst du hier ein \lstinline{else if(...)} einführen.
\end{enumerate}
