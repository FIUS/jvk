% !TeX root = ./jvk-blatt1.tex

\excercise{Eine API verwenden / Doku lesen}
\label{ex4}

\begin{enumerate}
    \item 
    Passe die Main Operation an die Aufgabe an.\\
		Verwende \lstinline{Sheet4Task1} und \lstinline{Sheet4Task1Verifier}.
\end{enumerate}



\begin{enumerate} \setcounter{enumi}{1}
    

    \item Nachdem wir auf Blatt 1 die Operation \lstinline{placeEntityAt()} aus der Klasse \lstinline{PlayfieldModifier} kennengelernt haben, beschäftigen wir uns nun etwas genauer mit dem \lstinline{PlayfieldModifier}.
        Insbesondere wollen wir nun herausfinden, welche anderen Operationen von dieser Klasse bereitgestellt werden. \\
        Es gibt noch zwei weitere Operationen, welche wir zum Platzieren von Münzen (\lstinline{Coin}), Wänden (\lstinline{Wall}) und Spielfiguren verwenden können.\\

        Versuche entweder mit Autocomplete von Eclipse oder mit der Dokumentation auf \url{\javadocRoot} herauszufinden, wie diese heißen.\\

        Falls du noch mehr Infos brauchst über Operationen o.ä. findest du die Dokumentation der \newline \lstinline{PlayfieldModifier} Klasse unter dem Link \url{\javadocRoot de.unistuttgart.informatik.fius.icge.simulation/de/unistuttgart/informatik/fius/icge/simulation/tools/PlayfieldModifier.html}.

        Am besten überfliegst du sie einfach kurz.

\end{enumerate}



\begin{enumerate} \setcounter{enumi}{2}
    \item Jetzt, da dir weitere Operationen zur Verfügung stehen, wollen wir einfach 20 Münzen in einer Zelle platzieren.

        Dazu musst du das Kommando vom \lstinline{PlayfieldModifier} benutzen, welches mehrere Entities auf demselben Feld platziert.

        Du sollst also nicht 19mal den Code kopieren um eine Münze zu erzeugen!

        Für den ersten Parameter (\lstinline{entityFactory}) kannst du \lstinline{new CoinFactory()} verwenden.
        Wenn dir nicht klar ist, wie dieses Kommando genau funktioniert, dann ließ dir nochmal die Dokumentation der Kommandos der Klasse \lstinline{PlayfieldModifier} durch.
\end{enumerate}




\begin{enumerate} \setcounter{enumi}{3}
    \item Platziere nun 3 horizontale Reihen mit jeweils 7 Münzen länge.
        Diese sollen jeweils ein Feld Abstand zueinander haben.

        Nutze dafür die Klasse \lstinline{Line}.\\
        Sie befindet sich im Paket \texttt{de.unistuttgart.informatik.fius.jvk.provided.shapes}.\\

        Die Line Klasse erwartet eine Startposition und auch eine Endposition.\\
        Ebenso solltest du in der Dokumentation von der \lstinline{PlayfieldModifier} Klasse nachschauen welche Operation du jetzt benutzen musst um mehrere Entitäten an mehreren Positionen zu erzeugen.\\

        \textbf{Übringens:} In dem Paket \texttt{de.unistuttgart.informatik.fius.jvk.provided.shapes} findest du die anderen Shapes aus unserer Shape Sammlung.

    \begin{lstlisting}
// Beispiel: eine Reihe von der Position (0,0) bis (5,0)
Position start = new Position(0,0);
Position end = new Position(5,0);

Line myLine = new Line(start, end);
    \end{lstlisting}

    %TODO f) optional andere Shapes
\end{enumerate}

\newpage
