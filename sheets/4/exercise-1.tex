    \excercise{Neo im Labyrinth}

    Die Agenten haben es geschafft, Neo in einem Labyrinth einzusperren.
    Um zu entkommen, muss er die einzige Telefonzelle finden.
    Neo erinnert sich, dass man aus jedem Labyrinth entkommen kann, wenn man immer an der rechten Wand entlangläuft.

        \subexcercise Implementiere die Operation \lstinline{checkSideWall()} der Klasse \lstinline{MySmartNeo}.
        Sie soll \lstinline{true} zurückgeben, wenn sich auf Neos rechter Seite eine Wand befindet.
        Ansonsten soll sie \lstinline{false} zurückgeben.


        \subexcercise Implementiere die Operation \lstinline{moveSmart()} der Klasse \lstinline{MySmartNeo}.
        In dieser soll Neo zuerst überprüfen, ob sich zu seiner rechten eine Wand befindet.
        Ist dies nicht der Fall, soll er sich nach rechts drehen und einen Schritt in diese Richtung gehen.
        Anderenfalls soll er überprüfen, ob er geradeaus laufen kann. Ist dies der Fall, soll er einen Schritt nach vorne machen.
        Sollte sich sowohl rechts, als auch geradeaus eine Wand befinden, soll Neo sich nach links drehen und stehen bleiben.


        \subexcercise Die Klasse \lstinline{Solution4_2} enthält ein Attribut vom Typ SmartNeo.
        Weise diesem Attribut in der prepare-Operation ein neues MySmartNeo-Objekt zu und
        platziere es auf Position \emph{[0, 0]}.
        Vervollständige weiterhin die solve-Operation mithilfe der Operationen,
        die du gerade geschrieben hast, sodass Neo das Telefon erreicht und darauf stehen bleibt.


