%!TEX root = ./jvk-blatt4.tex
\excercise{break und continue}

\begin{Infobox}[Endlosschleifen und \lstinline{break}]
	Eine While-Schleife, in der die Schleifenbedingung \lstinline{true} ist, nennt man eine Endlosschleife.
	Das heißt, dass die Schleife niemals aufhören wird den Code im Schleifenrumpf auszuführen, wenn das nicht durch eine Exception oder den Stop der Ausführung des Programms verhindert wird!
	Deshalb kann es passieren, dass du den Stopp Knopf in der IntelliJ brauchst, um die Schleife wieder abzubrechen.\newline

	Endlosschleifen möchte man in der Regel wann immer möglich vermeiden.
	Aber man kann auch eine \lstinline{while(true)}-Schleife planmäßig im Quellcode und ohne das Auslösen von Exceptions beenden.
	Dafür braucht man das Schlüsselwort \lstinline{break}.\newline

	Wenn du im Rumpf einer Schleife \lstinline{break;} aufrufst, dann wird die Schleife sofort unterbrochen.
	Auch die Schleifenbedingung wird dann nicht mehr überprüft und der Code nach der Schleife wird weiter ausgeführt.

	\begin{lstlisting}[numbers=none]
	while(condition){
		// do something
		if (hadEnough) {
			break;
		}
	}
	\end{lstlisting}
\end{Infobox}

\begin{enumerate} \setcounter{enumi}{0}
    \item gegeben ist eine Endlosschleife, schreibe deinen Code in den Schleifenkörper, sodass neoA bis zur gegenüberliegenden Wand läuft und vor ihr stehen bleibt. Dann soll neo eine Münze auf das Feld legen.
    \item Schreibe eine neue Endlosschleife unter den bissherigen Code und lasse neoB laufen bis er an der genenüberliegenden Wand ankommt oder bis er auf einem Feld mit genau zwei Münzen steht, dann nimm eine der Münzen auf.
\end{enumerate}

\begin{Infobox}[\lstinline{continue}]
	Wenn du im Rumpf einer Schleife \lstinline{continue;} aufrufst, dann geht die Schleife sofort in den nächsten Schleifendurchlauf.
	Die Schleifenbedingung wird erneut überprüft und die Schleife wird von anfang an ausgeführt falls die Schleifenbedingung wahr ist.

	\begin{lstlisting}[numbers=none]
	while(condition){
		// do something
		if (want_to_skip) {
			continue;
		}
		// code that is skipped if continued
	}
	\end{lstlisting}

\end{Infobox}
