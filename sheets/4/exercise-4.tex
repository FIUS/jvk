\excercise{Programme}

\begin{enumerate}
	\item Passe die Main Operation an die Aufgabe an.
		Verwende \lstinline{Sheet4Task4}, auch in dieser Aufgabe gibt es keinen Verifier.
		Diese Aufgabe ist mehr als ''Ideensammlung'' zu verstehen und wird nicht überprüft.
		Spiele mit den Konzepten, die eingeführt wurden.
	\item Erzeuge ein Programm, das Neo geradeaus laufen lässt, bis er keine Schritte mehr machen kann.
\end{enumerate}

\begin{Infobox}[Programme]
	Unsere Sammlung hat nicht nur verschiedene Sorten von Shapes, sondern auch Programme.
	Diese Programme erlauben es, Verhalten als ein Objekt darzustellen.
	Um ein Programm zu erstellen, solltest du eine neue Java-Datei anlegen und darin das Verhalten wie bekannt definieren.

	\begin{lstlisting}[title=SicherLaufenProgram.java, numbers=none,xleftmargin=0.5cm]
public class SicherLaufenProgram implements Program<Neo> {
	@Override
	public void run(Neo neo) {
		// neo läuft nur, wenn das auch geht:
		if (neo.canMove()) {
			neo.move();
		}
	}
}
	\end{lstlisting}

	Um ein Programm zu verwenden, musst du wie gewohnt ein Objekt erzeugen und kannst dann die Operation \lstinline{run} mit dem Objekt aufrufen, die sich wie im Programm angegeben verhalten soll:

	\begin{lstlisting}[numbers=none,xleftmargin=0.5cm]
	Neo neo; // definiert und auf dem Playfield verfügbar
	SicherLaufenProgram laufen = new SicherLaufenProgram();
	laufen.run(neo);
	\end{lstlisting}
\end{Infobox}

\begin{Infobox}[Generische Objekte]
	Beim definieren von \lstinline{SicherLaufenProgramm} könnte dir aufgefallen sein, dass \mbox{\lstinline{Program<Neo>}} etwas neues ist.
	Java bietet die Möglichkeit eine Art ''Typvariable'' zu definieren.
	Die Definition dieser Typvariablen kannst du dir in der Klasse \lstinline{Program} anschauen.

	Wenn du nun ein spezielles Programm definieren willst, kannst du den Typ festlegen.
	Java ersetzt dann die Typvariable mit dem Typ, den du angegeben hast.
\end{Infobox}

\begin{enumerate}\setcounter{enumi}{2}
	\item Löse einige der Aufgaben auf vorangegangenen Arbeitsblättern mit Programmen.
	
	\textbf{Tipp:} Du kannst auch Programme in anderen Programmen verwenden.
	\item Schreibe ein Programm, das eine vertikale Wand umgeht und dann auf derselben y-Koordinate landet.
	\item \optional \emph{Schwer!} Kannst du dieses Programm ohne eine Schleife schreiben?
\end{enumerate}
