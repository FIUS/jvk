\begin{questions}

    \titledquestion{Neo im Labyrinth}

    Die Agenten haben es geschafft, Neo in einem Labyrinth einzusperren.
    Um zu entkommen, muss er die einzige Telefonzelle finden.
    Neo erinnert sich, dass man aus jedem Labyrinth entkommen kann, wenn man immer an der rechten Wand entlangläuft.

    \begin{parts}
        \part{} Implementiere die Operation \lstinline{checkSideWall()} der Klasse \lstinline{MySmartNeo}.
        Sie soll \lstinline{true} zurückgeben, wenn sich auf Neos rechter Seite eine Wand befindet.
        Ansonsten soll sie \lstinline{false} zurückgeben.

        \begin{solution}
        \begin{lstlisting}[language=Java]
private boolean checkWall() {
    boolean result;
    this.turnClockwise()
    result = this.canMove();
    this.turnCounterClockwise();
    return result;
}
        \end{lstlisting}
        \end{solution}

        \part{} Implementiere die Operation \lstinline{moveSmart()} der Klasse \lstinline{MySmartNeo}.
        In dieser soll Neo zuerst überprüfen, ob sich zu seiner rechten eine Wand befindet.
        Ist dies nicht der Fall, soll er sich nach rechts drehen und einen Schritt in diese Richtung gehen.
        Anderenfalls soll er überprüfen, ob er geradeaus laufen kann. Ist dies der Fall, soll er einen Schritt nach vorne machen.
        Sollte sich sowohl rechts, als auch geradeaus eine Wand befinden, soll Neo sich nach links drehen und stehen bleiben.

        \begin{solution}
        \begin{lstlisting}[language=Java]
public void moveSmart() {
    if(this.checkSideWall()) {
        this.turnClockWise();
        this.move();
    } else if(this.canMove()) {
        this.move();
    } else {
        this.turnCounterClockwise();
    }
}
        \end{lstlisting}
        \end{solution}

        \part{} Die Klasse \lstinline{Solution4_2} enthält ein Attribut vom Typ SmartNeo.
        Weise diesem Attribut in der prepare-Operation ein neues MySmartNeo-Objekt zu und
        platziere es auf Position \emph{[0, 0]}.
        Vervollständige weiterhin die solve-Operation mithilfe der Operationen,
        die du gerade geschrieben hast, sodass Neo das Telefon erreicht und darauf stehen bleibt.


        \begin{solution}
        \begin{lstlisting}[language=Java]
@Override
public void prepare() {
    this.neo = new MySmartNeo();
}

@Override
public void solve() {
    while(notOnPhone) {
        this.neo.moveSmart();
    }
}
        \end{lstlisting}
        \end{solution}
    \end{parts}

    \titledquestion{Neo im Labyrinth - Teil 2 (Polymorphie)}

    Nachdem Neo aus dem Labyrinth entkommen ist, fragt er sich, ob er nicht schneller gewesen wäre, wenn er der linken Wand gefolgt wäre.

    \begin{parts}
        \part{} Implementiere die Operationen \lstinline{checkSideWall()} und \lstinline{moveSmart()} der Klasse \lstinline{MyLefthandedSmartNeo} entsprechend.
        \part Wie musst du die \lstinline{Solution4_2} Klasse verändern, damit Neo an der linken Wand entlang läuft?
        \begin{solution}
        \begin{lstlisting}[language=Java]
@Override
public void prepare() {
    this.neo = new MyLefthandedSmartNeo();
}
        \end{lstlisting}
        \end{solution}

        \part{} Schau dir noch einmal alle Neo-Klassen und ihre Vererbung an.
        Objekte welcher Klassen kannst du dem Attribut \lstinline{neo} der Klasse \lstinline{Task4_2} zuweisen?
        Was bewirkt der Aufruf von \lstinline{neo.checkSideWall()} jeweils?

        \begin{solution}
            \begin{enumerate}
                \item SmartNeo $\rightarrow{}$ Wirft eine Exception
                \item MySmartNeo $\rightarrow{}$ Überprüft die rechte Wand
                \item MyLefthandedSmartNeo $\rightarrow{}$ Überprüft die linke Wand
            \end{enumerate}
        \end{solution}
\end{parts}

\titledquestion{Escape}
Hilf Neo aus der Matrix zu entkommen. Als erstes muss Neo eine Pille einsammeln. Wenn diese Pille blau ist, soll die Münzenstapel von niedrig (links) nach hoch (rechts) sortieren. Wenn die gefundene Pille aber rot ist soll er es genau anders herum machen.
Wenn er die Aufgabe erledigt hatt wird das Telefon klingeln und Neo kann die Matrix verlassen.

\titledquestion{}
Jeder Neo muss eine die logische Formel
$$
(A \land B) \lor ((\neg D) \land C \lor B)
$$
lösen.
A, B, C ,D entsprechen den Feldern x=1 bis x=4. Wenn eine Münze auf einem Feld liegt besitzt das Feld den Wert True, ansonsten False.

Trage das Ergebnis in x=6 ein (Münze oder keine Münze)
\end{questions}
