\excercise{Objekte und Klassen}

\begin{Infobox}[HowTo: Klassen und Objekte auseinander halten]
    Klassen kann man sich wie Baupläne für Objekte vorstellen, 
    während Objekte nach dem in den dazugehörigen Klassen definierten Bauplänen gebaute Instanzen sind.\\

    \textbf{Beispiel:} Wir haben eine Klasse Student.

    Die Klasse definiert dabei, dass ein Student folgende Eigenschaften und Operationen haben soll,  befüllt aber die Eigenschaften noch nicht mit Werten:

    \begin{itemize}
        \item name
        \item age
        \item getName()
        \item tellAboutMyself()
        \item learn()
        \item eat()
    \end{itemize}

    Ein Objekt wäre dabei ein spezifischer Student, also z.B. du.\\
Dabei hast du einen Namen und ein Alter.
Ebenso kannst du folgende Operationen ausführen wie lernen, essen und über dich was erzählen.\\
Es kann aber auch insbesondere mehrere Objekte von einer Klasse geben, z.B. ist dein:e Nebensitzer:in  demnach auch ein (anderes) Objekt der Klasse Student.\\

In Java könnte das dann so aussehen:
    \begin{lstlisting}[numbers=left,xleftmargin=2em,frame=single,framexleftmargin=1.5em]
class Student {

        private String name;
        private int age;
        // ...

        public String getName() {
            return this.name;
        }

        public void tellAboutMyself() {
            System.out.println("Hello I'm " + this.getName() + ". I'm " + age + " old!";
        }

        public void learn() {
            // Some Code for learning
        }

        public void eat(Food food) {
            // Some Code to eat the food
        }
}

    \end{lstlisting}
\end{Infobox}

\begin{Infobox}[HowTo: Operationen von Objekten]
    Klassen verfügen über Operationen, die von bzw. auf ihren Objekten ausgeführt werden können.
Als Beispiel betrachten wir ein Auto. Es kann fahren, bremsen, abbiegen und den Motor an-/abschalten.\\

Dabei können Operationen auch sogenannte Parameter besitzen wie z.B. abbiegen(links) oder fahren(50) (für 50 km/h).
Wie man genau bestimmt, was diese Parameter bedeuten erklären wir noch.
\end{Infobox}

\begin{Infobox}[Info: Objekte vs. reale Objekte]
Wichtig zu verstehen ist, dass Klassen und Objekte keine physischen Objekte repräsentieren müssen.\\
Man kann auch z.B. eine Klasse für Useraccounts oder eine Klasse für Personengruppen haben.\\
Auch werden Klassen verwendet, um technische Teile des Programms zu repräsentieren, wie z.B. das Fenster, welches angezeigt wird beim Starten der Simulation, oder ein Paket, das durchs Internet geschickt wird.
\end{Infobox}

\begin{enumerate}
\item Instanziiere die Simulation wie bekannt (Sheet2Task7 und Sheet2Task7Verifier) und mache dich mit dieser vertraut.
\item Dir wird auffallen, dass in dieser Aufgabe mehr als ein Totoro auf dem Spielfeld erscheinen.
 Was unterscheidet die Totoros?
\item Versuche nun alle Attribute aufzulisten, welche verschiedene Totoros voneinander unterscheidet.
\end{enumerate}
