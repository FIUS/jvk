\excercise{Exceptions - Mit dem Kopf durch die Wand}
Tank und Neo wollen nun in einer neuen Simulationsumgebung trainieren. 
Dabei soll Neo in einem potenziell gefährlichen Szenario so schnell wie möglich zur nächsten Telefonstation laufen.
\begin{enumerate}[label=\alph*)]
    \item Instanziiere die Simulation wie bekannt und mache dich mit dieser vertraut.
    \item Als Operator hat Tank die Möglichkeit Hindernisse zwischen Neo und der Telefonstation einzufügen.
    Betrachte die Klasse \textit{Sheet2Task2} mit der Methode \textit{buildEnvironment}. Überlege dir wie man an Tank's Stelle 
    ein solches Hindernis an einer geeigneten Stelle platzieren kann.\par
    Was passiert, wenn man nun die Simulation wie zuvor ausführt?
    \item Im Folgenden betrachten wir beispielhaft den Stacktrace einer Exception:
    \definecolor{ForestGreen}{rgb}{0.0, 0.5, 0.0}
    \begin{lstlisting}[keywords={}, breaklines=true, escapeinside={(*@}{@*)}]
de.unistuttgart.informatik.fius.icge.simulation.exception.IllegalMoveException: Solid Entity in the way
at de.unistuttgart.informatik.fius.icge.simulation.entity.MovableEntity.internalMove(MovableEntity.java:(*@{\color{ForestGreen}linenumber1}@*))
at de.unistuttgart.informatik.fius.icge.simulation.entity.MovableEntity.move(MovableEntity.java:(*@{\color{ForestGreen}linenumber2}@*))
at de.unistuttgart.informatik.fius.jvk.tasks.Sheet2Task2.movement(Sheet2Task2.java:(*@{\color{ForestGreen}linenumber3}@*))
at de.unistuttgart.informatik.fius.jvk.tasks.Sheet2Task2.run(Sheet2Task2.java:(*@{\color{ForestGreen}linenumber4}@*))
at de.unistuttgart.informatik.fius.icge.simulation.internal.tasks.StandardTaskRunner.executeTask(StandardTaskRunner.java:(*@{\color{ForestGreen}linenumber5}@*))
at java.base/java.util.concurrent.CompletableFuture$AsyncSupply.run(CompletableFuture.java:(*@{\color{ForestGreen}linenumber6}@*))
at java.base/java.util.concurrent.ThreadPoolExecutor.runWorker(ThreadPoolExecutor.java:(*@{\color{ForestGreen}linenumber7}@*))
at java.base/java.util.concurrent.ThreadPoolExecutor$Worker.run(ThreadPoolExecutor.java:(*@{\color{ForestGreen}linenumber8}@*))
at java.base/java.lang.Thread.run(Thread.java:(*@{\color{ForestGreen}linenumber9}@*))
    \end{lstlisting}
    Dieser soll nun genauer analysiert werden:
    \begin{enumerate}
        \item[i)] Welche Exception wurde geworfen und in welcher Klasse ist diese aufgetreten?
        \item[ii)] Was für eine Struktur besitzt der Stacktrace? Welche Rolle spielt die Reihenfolge der Zeilen 2-10? 
        \item[iii)] In dem obigen Stacktrace sind die dazugehörigen Zeilennummern verloren gegangen. Ersetze diese!  
    \end{enumerate}
    \item Jetzt soll die Exception behandelt werden. Dazu betrachten wir die Methode \textit{movement} (erneut in der Klasse 
    \textit{Sheet2Task2}).
    \begin{enumerate}
        \item[i)] Welche ''linenumber'' aus dem obigen Stacktrace ändert sich, wenn man zwischen den \textit{move()} Aufrufen jeweils
        Leerzeilen einfügt?  
        \item[ii)] Kommentiere in dem markierten Block zunächst so viele Zeilen aus, sodass keine Exception mehr auftritt. 
        Spielt die Reihenfolge der auskommentierten Zeilen eine Rolle?
    \end{enumerate}
    \item  Nun sollen keine Zeilen gelöscht oder auskommentiert werden, durch welche Methode muss \textit{move()} ersetzt werden, damit 
    denoch keine Exception auftritt?
    \item Überlege dir möglichst viele verschiedene Methoden, wie man Neo trotz Hindernis zu der Telefonstation laufen lassen kann. 
    Implementiere diese!
\end{enumerate}
 
