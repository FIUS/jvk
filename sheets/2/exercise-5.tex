    \excercise{Attribute}
    Attribute sind Variablen die Informationen über den Zustand eines Objekts halten. Sie können Zahlen, Strings, Daten, oder auch nicht primitive Datentypen sein. Sie werden innerhalb der Klasse deklariert.
    \subexcercise Deklariere die folgende Attribute in MyNeo:
    \begin{itemize}
    \item age: Ganzzahl
    \item address: Text
    \item amountSteps: Ganzzahl
    \end{itemize}
    \subexcercise getter-Query gibt den Wert für die Attribute zurück
    \newline Erstelle für alle Attribute aus a) eine getter-Query
    \subexcercise Mit einer Setter-Operation werden die Attribute aktualisiert.
    Erstelle für jeweilige Attribut aus a) eine setter-Command
    \subexcercise countSteps beschreibt die Anzahl der Schritte die schon Neo gelaufen ist. Überschreibe nun das move-Command, so dass der Wert von     countSteps nach einem Schritt aktualisiert wird.
    \subexcercise Lösche alle zuvor erstellten setter und getter. Klicke nun mit der rechten Maustaste innerhalb der Klasse wähle \fbox{Source} $\to$ \fbox{Generate Setter and Getter...} $\to$ \fbox{hake alle Attribute oben an} $\to$ \fbox{Generate}
    \subexcercise Schaue dir noch einmal den Code der vorherigen Aufgabe an.
    Verändere ihn so, dass sowohl \lstinline{richNeo} als auch \lstinline{poorNeo} Attribute der Klasse \lstinline{Solution2_4} sind.

    Verschiebe ausßerdem die Anweisungen zum Erzeugen der Objekte in die prepare-Operation.
