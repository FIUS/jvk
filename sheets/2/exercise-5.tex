%!TEX root = ./jvk-blatt2.tex

\excercise{While Schleife}

Ziel dieser Aufgabe ist es, die Syntax und die Funktion von While-Schleifen zu verstehen.

\begin{enumerate}
    \item Instanziiere die Simulation wie bekannt (\lstinline{Sheet2Task5} und \lstinline{Sheet2Task5Verifier}) und mache dich mit dieser vertraut.
\end{enumerate}


\begin{Infobox}[While Schleife]
    Wir betrachten eine beispielhafte While Schleife:

    \begin{lstlisting}[breaklines=true, numbers=none]
while(totoro.canMove()) {
    totoro.move();
}
    \end{lstlisting}

    Ähnlich wie bei dem If-Statement ist auch bei der While Schleife in den geschweiften Klammern der Code-Block (Schleifen inneres, oft auch: Schleifenrumpf, Englisch: body).\\

    Das Verhalten von While Schleifen lässt sich einfach darstellen:
    \begin{enumerate}
        \item[1:] Der Computer prüft als Erstes, ob die Bedienung \lstinline{totoro.canMove()} erfüllt ist, also ob sich kein Hindernis im Weg befindet. Wenn ein Hindernis im Weg ist, springe zu Punkt 4.
        \item[2:] Das Schleifeninnere \lstinline{totoro.move()} wird ausgeführt.
        \item[3:] Springe zurück zu Punkt 1
        \item[4:] Verlasse die Schleife
    \end{enumerate}

    Ebenso sind folgende Beispiele zum Aufheben bzw. Ablegen aller (möglichen) Nüsse hilfreich:

    \begin{lstlisting}[breaklines=true, numbers=none]
// Aufheben aller möglichen Nüsse
while(totoro.canCollectNut()){
    totoro.collectNut();
}

// Ablegen aller möglichen Nüsse
while(totoro.canDropNut()){
    totoro.dropNut();
}
    \end{lstlisting}

    Dabei prüfen wir mit der Schleifenbedingung jeweils, ob es noch weitere Nüsse gibt, die wir aufheben (oder ablegen) können, bevor wir sie aufheben (oder ablegen).
\end{Infobox}


\begin{enumerate} \setcounter{enumi}{1}
    \item Hebe alle Nüsse auf dem aktuellen Feld auf.\\
    Dafür kannst du eine der Schleifen von oben verwenden.
    
    \textbf{Hinweis}: Bei dieser Aufgabe ist das Spielfeld und die Anzahl der Nüsse jedes Mal anders.
    \item Platziere jetzt alle Nüsse auf dem rechten mittleren Spielfeld an dem Busch.
    Hier kannst du dich wieder an den Beispielen oben orientieren.
    \item Zeichne mit den aufgehobenen Nüssen eine Linie auf der mittleren Spur, beginnend auf dem Feld neben Totoro.
    Auf jedem Feld in der mittleren Reihe sollte jetzt eine Nuss liegen.\\
    Benutze dazu auch eine While-Schleife, anstatt deinen Code mehrmals zu kopieren.
    
    \textbf{Hinweis}: Hier kannst du nicht direkt eine Schleife von den angegebenen Beispielen kopieren.
    Du musst noch den Code-Block der Schleife anpassen, damit mehr Kommandos bei jedem Schleifendurchlauf ausgeführt werden.
    \item Platziere nun statt einer Nuss auf jedem Feld der Linie drei Nüsse.
    
    \textbf{Hinweis}: Hier sollte es ausreichen den Code-Block der Schleife aus der Teilaufgabe davor anzupassen.
    \item Platziere auf jedem zweiten Feld eine Nuss.\\
    Also die Linie soll abwechselnd eine Nuss und ein leeres Feld nacheinander haben.\\
    
    \textbf{Hinweis}: Hier kannst du entweder die Abfolge direkt kodieren oder mit Variablen vom Typ \lstinline{boolean} und if-Verzweigungen aus Aufgabe 4 arbeiten.
    \item \optional Lege auf jedem von Totoro aus erreichbaren Feld genau eine Nuss ab.
    Beachte hierbei, dass die Büsche auf der oberen und unteren Spur zufällig platziert werden.
    
    \textbf{Hinweis}: Hier musst du neben einer While Schleife auch IF-Statements verwenden.\\
    Schau dir dafür nochmal Aufgabe 4 an.
\end{enumerate}
