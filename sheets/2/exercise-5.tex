%!TEX root = ./jvk-blatt2.tex

\excercise{While i do stuff on repeat}

Ziel dieser Aufgabe ist es, die Syntax und die Funktion von While-Schleifen genauer zu verstehen.

\begin{enumerate}
    \item Instanziiere die Simulation wie bekannt und mache dich mit dieser vertraut.
\end{enumerate}


\begin{Infobox}[While Schleife]
    Wir betrachten eine beispielhafte While Schleife:

    \begin{lstlisting}[breaklines=true, numbers=none]
while(player.canMove()) {
    player.move();
}
    \end{lstlisting}

    Ähnlich wie bei dem If-Statement ist auch bei der While Schleife in den geschweiften Klammern der Code-Block (Schleifen inneres, oft auch: Schleifenrumpf, Englisch: body).\\

    Das Verhalten von While Schleifen lässt sich einfach darstellen:
    \begin{enumerate}
        \item[1:] Der Computer prüft als Erstes, ob die Bedienung \lstinline{palyer.canMove()} erfüllt ist, also das sich kein Hindernis im Weg befindet, wenn NEIN springe zu Punkt 4.
        \item[2:] Das Schleifeninnere \lstinline{player.move()} wird ausgeführt.
        \item[3:] Springe zurück zu Punkt 1
        \item[4:] Verlasse die Schleife
    \end{enumerate}

    Was passiert, wenn von Anfang an ein Hindernis vor dem Spieler sich befindet?\\

    Ebenso sind folgende Beispiele zum Aufheben bzw Ablegen aller (möglichen) Münzen hilfreich:

    \begin{lstlisting}[breaklines=true, numbers=none]
// Aufheben aller möglichen Münzen
while(player.canCollectCoin()){
    player.collectCoin();
}

// Ablegen aller möglichen Münzen
while(player.canDropCoin()){
    player.dropCoin();
}
    \end{lstlisting}

    Dabei prüfen wir mit der Schleifenbedingung jeweils, ob es noch weitere Münzen gibt, die wir aufheben (oder ablegen) können, bevor wir sie aufheben (oder ablegen).
\end{Infobox}


\begin{enumerate} \setcounter{enumi}{1}
    \item Öffne die Klasse \lstinline{Sheet2Task5}.
        Hebe alle Münzen auf.\\
        Dafür kannst du eine der Schleifen von oben verwenden.

    \textbf{Hinweis}: Bei dieser Aufgabe ist das Spielfeld und die Anzahl der Münzen jedes mal anders.
    \item Platziere jetzt alle Münzen auf dem rechten mittleren Spielfeld an der Wand. 
        Hier kannst du dich wieder an den Beispielen oben orientieren.
    \item Zeichne mit den aufgehobenen Münzen eine Linie auf der mittleren Spur. 
        Auf jedem Feld in der mittleren Reihe sollte jetzt eine Münze liegen.\\
        Benutze dazu auch eine While Schleife, anstatt deinen Code mehrmals zu kopieren.

    \textbf{Hinweis}: Hier kannst du nicht direkt eine Schleife von den angegebenen Beispielen kopieren.
        Du musst noch den Code-Block der Schleife anpassen, damit mehr Kommandos bei jedem Schleifendurchlauf ausgeführt werden.
    \item Platziere nun statt einer Münze auf jedem Feld der Linie drei Münzen.

    \textbf{Hinweis}: Hier sollte es ausreichen den Code-Block der Schleife aus der Teilaufgabe davor anzupassen.
    \item Platziere auf jedem zweiten Feld eine Münze.\\
        Also die Linie soll abwechselnd eine Münze und ein leeres Feld nacheinander haben.
    \item \optional Lege auf jedem von Neo aus erreichbaren Feld genau eine Münze ab. 
        Beachte hierbei, dass die Wände auf der oberen und unteren Spur zufällig platziert werden.

    \textbf{Hinweis}: Hier musst du neben einer While Schleife auch IF-Statements verwenden.\\
        Schau dir dafür nochmal Aufgabe 3 an.
\end{enumerate}
