%!TEX root = ./jvk-blatt2.tex

\excercise{While i do stuff on repeat}

Ziel dieser Aufgabe ist es, die Syntax und die Funktion der While-Schleife genauer zu verstehen.

\begin{enumerate}
    \item Instanziiere die Simulation wie bekannt und mache dich mit dieser vertraut.
\end{enumerate}


\begin{Infobox}[While Schleife]
    Wir betrachten eine beispielhafte While Schleife.

    \begin{lstlisting}[breaklines=true, numbers=none]
while(player.canMove()){
    player.move();
}
    \end{lstlisting}

    While Schleifen prüfen jeweils vor Ausführung des Schleifeninneren, ob die gegebene Bedinung noch erfüllt ist. 
    In unserem Beispiel wird also zunächst geprüft ob \textit{i \textless 3} mit dem Initialwert \textit{i = 0} gilt. 
    Da dies erfüllt ist, wird der aktuelle Wert von \textit{i} ausgegeben und inkrementiert. 
    Nach der dritten Iteration ist der Wert von \textit{i} gleich 3, wodurch die obige Bedinung nicht mehr erfüllt ist und die Schleife nicht weiter ausgeführt wird. 
    
    Eine für uns relevantere Funktion wäre das Aufheben von einer unbekannten Anzahl an Münzen. 
    Diese kann mittels einer While Schleife wie folgt realisiert werden:

    \begin{lstlisting}[breaklines=true, numbers=none]
while(player.canCollectCoin()){
    player.collectCoin();
}
    \end{lstlisting}

    Dabei prüfen wir mit der Schleifenbedinung jeweils, ob es noch weitere Münzen gibt die wir aufheben können und heben diese ggfs. in dem Schleifenrumpf auf.
\end{Infobox}


\begin{enumerate} \setcounter{enumi}{1}
    \item Betrachte das Kommando \textit{run} in der Klasse \textit{Sheet2Task5}. 
        Übernehme die obige While Schleife um alle Münzen auf dem Feld vor Neo aufzuheben.
    \item Platziere alle Münzen auf dem rechten mittleren Spielfeld an der Wand. 
        Implementiere das Verhalten in dem \textit{run} Kommando.
        Hier können die Abfragen \lstinline{canDropCoin()} und \lstinline{dropCoin()} genutzt werden.
    \item Zeichne mit den aufgehobenen Münzen eine Linie auf der mittleren Spur. 
        Implementiere das Verhalten in dem \textit{run} Kommando so, dass auf jedem Feld der Spur genau eine Münze liegt.
    \item Platziere nun auf jedem Feld der selben Spur exakt 3 Münzen.
    \item Hier soll die selbe Spur wie in Teilaufgabe \textit{d)} gezeichnet werden, nur dass die geraden Felder übersprungen werden.
    \item Optional: Lege auf jedem von Neo aus erreichbaren Feld genau eine Münze ab. 
        Beachte hierbei, dass die Wände auf der oberen und unteren Spur zufällig platziert werden.
\end{enumerate}
