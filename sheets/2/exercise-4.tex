%!TEX root = ./jvk-blatt2.tex

\excercise{Verzweigung}

In dieser Aufgabe wollen wir uns Conditions (Verzweigungen) näher anschauen.

\begin{enumerate}                           
    \item Instanziiere die Simulation wie bekannt (\lstinline{Sheet2Task4} und \lstinline{Sheet2Task4Verifier}) und mache dich mit dieser vertraut.
        Was passiert bei der Ausführung?
\end{enumerate}

\begin{Infobox}[Bool'sche Werte]
    Um if-Verzweigung verstehen zu können, müssen wir uns zuerst Bool'sche Werte anschauen. Bool'sche Werte sind Variablen oder auch Rückgabewerte von Funktionen, welche nur den Wert \lstinline{true} oder \lstinline{false} (also wahr oder falsch) annehmen können. \\
    Im Quellcode hat jeder Wert, ob Rückgabe einer Funktion oder eben Variablen, einen Typ.
    Dieser Typ bestimmt, welche Werte für diesen Wert zulässig sind.
    Beispielsweise gibt es einfachen - primitive - Werte wie zum Beispiel Ganzzahlen oder Zeichenketten, die wir bisher sogar schon verwendet haben.
    Eine Zeichenkette (in Programmiersprachen in der Regel als \lstinline{String} bezeichnet) kann beispielsweise den Inhalt \lstinline{''Hello World''} haben.
    Diesen Wert allerdings in einer Variable, die nur Ganzzahlen speichert (= \lstinline{int} für Integer) zu speichern, macht allerdings weniger Sinn.
    
    Genau wie \lstinline{int} gibt es noch ein paar andere primitive Typen, welche häufig verwendet werden.
    Am häufigsten kommen dabei wohl bool'sche Werte mit dem zugehörige primitive Typ \lstinline{boolean} zur Verwendung.
    Da es sich um einen primitiven Typ handelt, wird er genau wie \lstinline{int} in klein geschrieben.\\
    
    Es ist wichtig zu erwähnen, dass in Java \lstinline{int} und \lstinline{Integer} dieselben Werte speichern können, aber nicht derselbe Typ sind.
    Den genauen Unterschied zwischen den beiden Typen werdet ihr noch in der PSE (Programmierung und Softwareentwicklung) kennenlernen.
    
    Die Funktion \lstinline{neo.canMove()} zum Beispiel gibt den bool'schen Wert \lstinline{true} zurück wenn Neo eine Schritt in seine Blickrichtung gehen kann, also auf diesem Feld sind keine Hindernisse sind.
    Die \lstinline{canMove}-Methode hat also den Rückgabewert \lstinline{boolean}.
    
\end{Infobox}

\begin{Infobox}[IF-Verzweigung]
    Nun betrachten wir ein beispielhaftes IF-Statement:

    \begin{lstlisting}[breaklines=true, numbers=none]
        if (neo.canMove()) {
            neo.move();
        } 
    \end{lstlisting}

    Das beispielhafte IF-Statement bedeutet informal, dass falls \lstinline{neo.canMove()} den bool'schen Wert \lstinline{true} hat, wird \lstinline{neo.move()} ausgeführt. Falls \lstinline{neo.canMove()} den bool'schen Wert \lstinline{false} hat wird der Code in den geschweiften Klammern übersprungen.

    \begin{lstlisting}[breaklines=true, numbers=none]
if (neo.canMove()) {
    neo.move();
} else {
    System.out.println("Morpheus: There is a difference between knowing the path and walking the path")
}
    \end{lstlisting}

    Nun wurde noch ein ELSE-Statement angehängt. Dadurch dient der Code als Verzweigung mit der Bedingung, dass falls etwas gilt mache etwas, falls nicht mache irgendwas anderes. In unserem Beispiel, falls \lstinline{neo.canMove()} den bool'schen Wert \lstinline{false} hat wird auf der Konsole der Text ''Morpheus: There is a difference between knowing the path and walking the path'' ausgegeben.\\

    Für sowohl das IF- als auch das ELSE-Statement gilt, dass in den geschweiften Klammern beliebig viel Code stehen kann, nicht nur eine Zeile wie in unserem Beispiel.
\end{Infobox}


\begin{enumerate} \setcounter{enumi}{1}
    \item Betrachte die Operation \lstinline{movement} der Klasse \lstinline{Sheet2Task4}.\\
        Füge das obige IF-Statement an einer geeigneten Stelle ein und führe die Simulation erneut aus.
        Dabei sollten die dann noch vorhandenen \lstinline{.move()} Kommandos in den ersten Code-Block vom IF verschoben werden.
        Der \lstinline{else} Code-Block soll erstmal leer bleiben.
    \item Schließe nun das Fenster und öffne es erneut.
        Lösche die Wand in dem Spielfenster.
        Dazu musst du oben rechts das rote Minus auswählen und dann auf die Wand klicken.
        Starte dann die Simulation.\\
        Neo sollte sich jetzt anders verhalten, obwohl du den Code nicht geändert hast!
    \item Jetzt wollen wir den Code so anpassen, dass Neo die Telefonstation auch erreicht wenn eine Wand im Weg ist.

        Du musst also einmal den Fall betrachten, dass keine Wand im Weg ist und einmal den Fall, dass eine Wand im Weg ist.\\

        Überprüfe deinen Code indem du einmal die Simulation startest ohne die Wand zu löschen und einmal wenn du die Wand davor gelöscht hast.
        In beiden Fällen sollte Neo die Telefonzelle erreichen.

\end{enumerate}
