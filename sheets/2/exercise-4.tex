% !TEX root = ./jvk-blatt2.tex

\excercise{I have the largest pockets}

In dieser Aufgabe wollen wir eine neue Funktion unserer Spieler einführen. 
Wer die Doku aufmerksam gelesen hat kennt sie vielleicht schon: 
Unesere Spieler können Goldmünzen aufheben und niederlegen. 
Dafür verwenden wir \lstinline{collectCoin()} und \lstinline{dropCoin()}.

\begin{enumerate}
    \item[a)] Wie immer muss der Task und Verifier gesetzt werden (Sheet2Task4 und Sheet2Task4VerifierB für Aufgabe b) bzw. Sheet2Task4VerifierC für Aufgabe c))
    \item[b)] Auf dem Spielfeld siehst du viele Goldmünzen.
        Wenn dein Spieler auf einem Feld mit Münzen steht kann er mit \lstinline{collectCoin()} eine aufheben.
        Räume damit das ganze Feld auf.
        
        \textbf{Tipp}: Verwende \lstinline{move()} und \lstinline{turnRight()} um deinen Spieler zu bewegen.

    \item[c)] Nun verwende die \lstinline{dropCoin()} Operation um alle Münzen die dein Spieler aufgesammelt hat auf das Feld (2,2) zu legen.
    \item[d)] Welche Bedingungen gibt es, die bestehen müssen, das ein Spieler die \lstinline{collectCoin()} Methode sinnvoll verwenden kann? 
\end{enumerate}
