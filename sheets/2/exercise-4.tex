    \excercise{Variablen und Objekterzeugung}
    Zunächst wollen wir uns nun einen eigenen Neo erstellen. Dies können wir tun, indem wir die folgende Zeile in eine Operation einfügen:
    %TODO: refactor class name MyNeo in this listing and following exercise ; ?--Lion
    \begin{lstlisting}
MyNeo neo = new MyNeo();
    \end{lstlisting} Hierbei ist \texttt{neo} der Name einer Variable vom Typ MyNeo. Wir erzeugen ein neues Objekt mittels dem Schlüsselwort \lstinline{new} und dem \emph{Konstruktor} der Klasse \texttt{MyNeo} und weisen dieses direkt der Variable '\texttt{neo}' zu.

    Nachdem wir die Variable \lstinline{neo} definiert und initialisiert haben, können wir sie verwenden, zum Beispiel indem wir ihn wie folgt an Position \((0,0)\) auf das Spielfeld setzen:
    \begin{lstlisting}
sim.getPlayfield().addEntity(new Position(0, 0), neo);
    \end{lstlisting}
    Dafür fragen wir mit \texttt{getPlayfield()} erst das Spielfeld von der Simulation ab und sagen dann mit \texttt{addEntitiy(...)}, dass es neo an Position \((0,0)\) hinzufügen soll
    \subexcercise Erstelle zwei neue Objekte vom Typ MyNeo stelle sie nebeneinander auf das Spielfeld. Drehe beide um \(360^\circ\).

    Nun wollen wir einen Neo erstellen, der schon von Anfang an einige Münzen hat. Dies können wir erreichen, indem wir dem Konstruktor der Klasse \lstinline{MyNeo} die Anzahl der Münzen übergeben, die er zu Beginn haben soll. Hierfür schreiben wir diese \emph{Argumente} wie bei Operationenaufrufen in die Klammern hinter '\lstinline{new MyNeo}'.

    \subexcercise Erstelle ein Objekt \lstinline{richNeo} vom Typ \lstinline{MyNeo}, das schon zu Beginn \(1\,000\) Münzen hat.
    \subexcercise Erstelle nun noch ein Objekt \lstinline{poorNeo} vom Typ \lstinline{MyNeo},
    das zu Beginn keine Münzen hat.
    Lasse \lstinline{richNeo} {\(2\)} Münzen ablegen
    und sammle diese dann mit \lstinline{MyNeo} ein.
    %TODO Konstruktor selber schreiben?


