%!TEX root = ./jvk-blatt2.tex

\excercise{Conditions}

In dieser Aufgabe wollen wir uns Conditions näher anschauen.

\begin{enumerate}                           
    \item Instanziiere die Simulation wie bekannt (\lstinline{Sheet2Task4} und \lstinline{Sheet2Task4Verifier}) und mache dich mit dieser vertraut.
        Was passiert bei der Ausführung?
\end{enumerate}

\begin{Infobox}[Boolean Werte]
    Um if Conditions verstehen zu können, müssen wir uns zuerst Boolean Werte anschauen. Boolean Werte sind Variablen, welche nur den Wert \lstinline{true} oder \lstinline{false} (also wahr oder falsch) annehmen können. \\

    Die Funktion \lstinline{player.canMove()} zum Beispiel gibt den boolschen Wert \lstinline{true} zurück wenn der Player (z.B. Neo) eine Schritt in seine Blickrichtung gehen kann, also auf diesem Feld sind keine Hindernisse. 
    
\end{Infobox}

\begin{Infobox}[IF-Condition]
    Nun betrachten wir ein beispielhaftes IF-Statement:

    \begin{lstlisting}[breaklines=true, numbers=none]
        if (player.canMove()) {
            player.move();
        } 
    \end{lstlisting}

    Das beispilehafte IF-Statement bedeuted informal, dass falls player.canMove() den boolschen Wert \lstinline{true} hat, wird player.move() ausgeführt. Falls player.canMove() den boolschen Wert \lstinline{false} hat wird der Code in den eckigen Klammern übersprungen.

    \begin{lstlisting}[breaklines=true, numbers=none]
if (player.canMove()) {
    player.move();
} else {
    System.out.println("Morpheus: There is a difference between knowing the path and walking the path")
}
    \end{lstlisting}

    Nun wurde noch ein ELSE-Statement angehängt. Dadurch dient der Code als Verzweigung mit der Bedingung, falls etwas gilt mache etwas, falls nicht mache irgendwas anderes. In unserem Beispiel, falls player.canMove() den boolschen Wert \lstinline{false} hat wird auf der Konsole der Text "Morpheus: There is a difference between knowing the path and walking the path" ausgegeben.\\

    Für sowohl das IF- als auch das ELSE-Statement gilt, dass in den eckigen Klammern belibig viel Code stehen kann, nicht nur eine Zeile wie in unserem Beispiel.
\end{Infobox}


\begin{enumerate} \setcounter{enumi}{1}
    \item Betrachte die Operation \lstinline{movement} der Klasse \lstinline{Sheet2Task4}.\\
        Füge das obige IF-Statement an einer geeigneten Stelle ein und führe die Simulation erneut aus.
        Dabei sollten die dann noch vorhandenen \lstinline{.move()} Kommandos in den ersten Code-Block vom IF verschoben werden.
        Der \lstinline{else} Code-Block soll erstmal leer bleiben.
    \item Schließe nun das Fenster und öffne es erneut.
        Lösche die Wand in dem Spielfenster.
        Dazu musst du oben rechts das rote Minus auswählen und dann auf die Wand klicken.
        Starte dann die Simulation.\\
        Neo sollte sich jetzt anders verhalten, obwohl du den Code nicht geändert hast!
    \item Jetzt wollen wir den Code so anpassen, dass Neo die Telefonstation auch erreicht wenn eine Wand im Weg ist.

        Du musst also einmal den Fall betrachten, dass keine Wand im Weg ist und einmal den Fall, dass eine Wand im Weg ist.\\

        Überprüfe deinen Code indem du einmal die Simulation startest ohne die Wand zu löschen und einmal wenn du die Wand davor gelöscht hast.
        In beiden Fällen sollte Neo die Telefonzelle erreichen.

\end{enumerate}
