\excercise{Wenn ich mich doch nur nach links drehen könnte}

\begin{enumerate}[label=\alph*)]
    \item Instanziiere den Task und den Verifier. Dabei soll wie in der \textit{Aufgabe 5b)} der SubTask jeweils angegeben werden.
    \item Lasse Neo im Uhrzeigersinn an der Wand entlang laufen, bis er wieder an seinem Startpunkt angekommen ist.
    Implementiere das Verhalten in der Methode \textit{subTaskB()} in der Klasse \textit{Sheet3Task6}.
    \item Lasse nun Neo gegen den Uhrzeigersinn an der Wand entlang laufen, bis er wieder an seinem Startpunkt angekommen ist.
    Implementiere das Verhalten in der Methode \textit{subTaskC()} in der Klasse \textit{Sheet3Task6}.
    Was fällt dir im Vergleich zu vorhin auf?
\end{enumerate}
\begin{Infobox}[Methodensyntax]
    Wir wollen uns nun die Methodensyntax etwas genauer anschauen. Dafür betrachten wir beispielhaft den folgenden Programmcode:
    \begin{lstlisting}[breaklines=true, numbers=none]
//Class A
public void foo(){
    B b = new B();
    System.out.println(b.bar(5, 7));
}

//Class B
private int b = 9;
public boolean bar(int a, int b){
    if(a > b)
        return true;
    return power(a) > this.b;
}

private int power(int a){
    return a*a;
}
\end{lstlisting}
Bei der Programmausführung gehen wir davon aus dass die Methode \textit{foo} ausgeführt wird. Wir unterteilen die Erklärung der Syntax in folgende Blöcke:
\begin{enumerate}[label=\roman*)]
    \item 
    Grundlegender Sytaxaufbau: Der Aufbau einer \textit{einfachen} Methode folgt zunächst einmal dem Schema 
    \begin{lstlisting}
Modifier Datentyp Methodenname (Par a, Par b, ...){
    //do stuff
    return ...;
}
    \end{lstlisting}
    Welche Methoden sind also in dem obigen Codebeispiel gegeben und wie wurden diese benannt?
    \item Modifier: Hier unterscheiden wir zunächst zwischen den \textit{public} und \textit{private} Modifiern. Der \textit{public} Modifier erlaubt es, dass die Methode aus einer
    anderen beliebigen Klasse aufgerufen werden kann. Im Gegensatz dazu kann eine Methode mit dem \textit{private} Modifier nur innerhalb der selben Klasse aufgerufen werden.
    Somit ist es beispielsweise möglich, dass die Methode \textit{foo} in der Klasse A die Methode \textit{bar} von der Klasse B aufruft, während Klasse A jedoch die \textit{power} Methode
    der Klasse B nicht sehen kann.
    \item Datentyp: Dieser gibt an, welche Art von Rückgabe die Methode liefert. Die Rückgabe selber wird durch das Keyword \textit{return} signalisiert.
    \item \textit{return}:
    Nachdem \textit{return} aufgerufen wurde, wird sofort zu der aufrufenden Methode zurückgesprungen.\\
    Beispiel: Wenn die Methode \textit{bar} mit Parametern \textit{a = 5, b = 2} aufgerufen wird, ist die IF-Condition erfüllt und es wird sofort \textit{true} zurückgegeben.
    Dadurch wird die darunterliegende Zeile nicht ausgeführt.
\end{enumerate}

\end{Infobox}
\begin{enumerate}[label=\alph*)] \setcounter{enumi}{3}
    \item Anstatt Neo \textit{manuell} drei Mal im Uhrzeigersinn drehen zu lassen, wollen wir nun die fehlende Methode \textit{turnCounterClockwise()}
    implementieren. Finde dafür zunächst die richtige Klasse und die dazugehörige Stelle im Code, in welcher die Methode eingefügt werden soll (diese Stelle ist auch durch ein Kommentar vorgegeben). \par
    Implementiere anschließend die fehlende Methode \textit{turnCounterClockwise()}. Im Folgenden wird die Syntax der Methode beispielshaft vorgegeben:
    \begin{lstlisting}
public void turnCounterClockwise() {
    //TODO
}
    \end{lstlisting}
    \item Teste deine Implementierung aus der Teilaufgabe \textit{f)} indem du deinen Code in der Methode \textit{subTaskC()} anpasst.
\end{enumerate}
\begin{Infobox}[Javadoc]
    Hier wollen wir nun erstmalig Javadoc für eine unserer Methoden schreiben. Dafür betrachten wir vorerst die Syntax an einem Beispiel:
    \begin{lstlisting}[numbers=none]
    /**
    * Computes the greatest integer value of the logarithm 
    * of a given natural number and a given base. 
    * 
    * The computed value corresponds to the logarithm of the
    * given base rounded downwards. In order to determine
    * a unique solution, a base greater than 1 is required.
    * Furthermore, the logarithm of 0 is not defined for any 
    * natural number. In this case an exception is thrown.
    * 
    * @param n the natural number of which the logarithm is computed
    * @param b the base of the computed logarithm
    * @throws IllegalArgumentException if n is not a natural
    * number greater than 0 or b is a base for which an unique
    * solution cannot be computed.
    * @return the logarithm of the given base rounded downwards
    */
    public static int roundedLogarithm(int n, int b){
        if(n <= 0 || b <= 1)
            throw new IllegalArgumentException();
        
        int logValue = -1;
        int power = 1;
        while(power <= n){
            power = power * b;
            logValue = logValue + 1;
        }

        return logValue;
    }
    \end{lstlisting}
    Dabei kann ein großteil der Berechnung innerhalb der Methode ignoriert werden und ist nur zur Vollständigkeit mit angegeben. \par
    Der generelle Aufbau der Methodendokumentation startet hier mit einer kurzen Zusammenfassung 
    gefolgt von einer ausführlichen Beschreibung der Methodenfunktion. Für den Leser ist hierbei die 
    konkrete Umsetzung innerhalb der Methode eher uninteressant, weshalb man sich in der Dokumentation
    auf die Verwendung und das Verhalten der Methode beschränkt. Es sollte hierbei insbesondere darauf 
    eingegangen werden, welche Vorbedingungen von der Methode benötigt werden und mit welchem Ergebnis 
    der Benutzer anschließend rechnen darf. \par 
    Nun gibt es noch ein paar interessante Keywords, die im Folgenden noch kurz erklärt werden:
    \begin{enumerate}[label=\roman*)]
        \item \textit{\texttt{@param}}: Gefolgt von einem Parametername (bspw. \textit{\texttt{@param n}}) wird dem
        Leser die Bedeutung des Parameters erklärt.
        \item \textit{\texttt{@return}}: Beschreibt den zurückgegebene Wert unter Annahme, dass die Eingabe korrekt ist.
        \item \textit{\texttt{@throws}}: Gefolgt von der geworfenen Exception (bspw. \textit{\texttt{@throws IllegalArgumentException}})
        wird hier beschrieben wann diese Exception geworfen wird.
    \end{enumerate}
\end{Infobox}
\begin{enumerate}[label=\alph*)] \setcounter{enumi}{5}
    \item Nun wollen wir natürlich dafür sorgen, dass unsere Nachfolger auch verstehen was in der von uns implementierten Methode von 
    Teilaufgabe \textit{f)} passiert. Schreibe Javadoc für diese Methode.
    \item Die Dokumentation von Methoden soll überraschendes Verhalten bei der Ausführung verhindern. 
    Wenn du Neo um 180° umdrehen willst kannst du dafür jetzt turnClockwise und deine neue Methode benutzen. 
    Verhalten sich die Beiden Methoden dabei gleich?
    Wenn nein: Kannst du das unterschiedliche Verhalten vorhersagen indem du nur die Dokumentation der beiden Methoden vergleichst?
    Falls nicht war deine Dokumentation noch nicht genau genug du solltest sie nochmal verbessern.
\end{enumerate}
