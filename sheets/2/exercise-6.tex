\excercise{Wenn ich mich doch nur nach links drehen könnte}

\begin{enumerate}[label=\alph*)]
    \item Instanziiere den Task und den Verifier. Dabei soll wie in der \textit{Aufgabe 5b)} der SubTask jeweils angegeben werden.
    \item Lasse Neo im Uhrzeigersinn an der Wand entlang laufen, bis er wieder an seinem Startpunkt angekommen ist.
    Implementiere das Verhalten in der Methode \textit{subTaskB()} in der Klasse \textit{Sheet3Task6}.
    \item Lasse nun Neo gegen den Uhrzeigersinn an der Wand entlang laufen, bis er wieder an seinem Startpunkt angekommen ist.
    Implementiere das Verhalten in der Methode \textit{subTaskC()} in der Klasse \textit{Sheet3Task6}.
    Was fällt dir im Vergleich zu vorhin auf?
\end{enumerate}
\begin{Infobox}
    Wir wollen uns nun die Methodensyntex etwas genauer anschauen. Dafür betrachten wir beispielhaft den folgenden Programmcode:
    \begin{lstlisting}[breaklines=true, numbers=none]
//Class A
public void foo(){
    B b = new B();
    System.out.println(b.bar(5, 7));
}

//Class B
private int b = 9;
public boolean bar(int a, int b){
    if(a > b)
        return true;
    return power(a) > this.b;
}

private int power(int a){
    return a*a;
}
\end{lstlisting}
Bei der Programmausführung gehen wir davon aus dass die Methode \textit{foo} ausgeführt wird. Wir unterteilen die Erklärung der Syntax in folgende Blöcke:
\begin{enumerate}[label=\roman*)]
    \item 
    Grundlegender Sytaxaufbau: Der Aufbau einer \textit{einfachen} Methode folgt zunächst einmal dem Schema 
    \begin{lstlisting}
Modifier Datentyp Methodenname (Par a, Par b, ...){
    //do stuff
    return ...;
}
    \end{lstlisting}
    Welche Methoden sind also in dem obigen Codebeispiel gegeben und wie wurden diese benannt?
    \item Modifier: Hier unterscheiden wir zunächst zwischen den \textit{public} und \textit{private} Modifiern. Der \textit{public} Modifier erlaubt es, dass die Methode aus einer
    anderen beliebigen Klasse aufgerufen werden kann. Im Gegensatz dazu kann eine Methode mit dem \textit{private} Modifier nur innerhalb der selben Klasse aufgerufen werden.
    Somit ist es beispielsweise möglich, dass die Methode \textit{foo} in der Klasse A die Methode \textit{bar} von der Klasse B aufruft, während Klasse A jedoch die \textit{power} Methode
    der Klasse B nicht sehen kann.
    \item Datentyp: Dieser gibt an, welche Art von Rückgabe die Methode liefert. Die Rückgabe selber wird durch das Keyword \textit{return} signalisiert.
    \item \textit{return}:
    Nachdem \textit{return} aufgerufen wurde, wird sofort zu der aufrufenden Methode zurückgesprungen.\\
    Beispiel: Wenn die Methode \textit{bar} mit Parametern \textit{a = 5, b = 2} aufgerufen wird, ist die IF-Condition erfüllt und es wird sofort \textit{true} zurückgegeben.
    Dadurch wird die darunterliegende Zeile nicht ausgeführt.
\end{enumerate}

\end{Infobox}
\begin{enumerate}[label=\alph*)] \setcounter{enumi}{3}
    \item Anstatt Neo \textit{manuell} drei Mal im Uhrzeigersinn drehen zu lassen, wollen wir nun die fehlende Methode \textit{turnCounterClockwise()}
    implementieren. Finde dafür zunächst die richtige Klasse und die dazugehörige Methode. \par
    Implementiere anschließend die fehlende Methode \textit{turnCounterClockwise()}. Im Folgenden wird die Syntax beispielshaft vorgegeben:
    \begin{lstlisting}
//Class Neo
public void turnCounterClockwise() {
    //TODO
}
    \end{lstlisting}
    \item Teste deine Implementierung aus der Teilaufgabe \textit{f)} indem du deinen Code in der Methode \textit{subTaskC()} anpasst.
    \item Nun wollen wir natürlich dafür sorgen, dass unsere Nachfolger auch verstehen was in der von uns implementierten Methode von 
    Teilaufgabe \textit{f)} passiert. Schreibe Javadoc für diese Methode.
    \item Die Dokumentation von Methoden soll überraschendes Verhalten bei der Ausführung verhindern. Beinhaltet deine Dokumentation auch 
    das Verhalten, welches bei gewünschter 180° Drehung von Neo mittels der Methode \textit{turnCounterClockwise()} auftritt?
\end{enumerate}