\excercise{Wenn ich mich doch nur nach links drehen könnte}

\begin{enumerate}[label=\alph*)]
    \item Instanziiere den Task und den Verifier. Dabei soll wie in der \textit{Aufgabe 5b)} der SubTask jeweils angegeben werden.
    \item Lasse Neo im Uhrzeigersinn an der Wand entlang laufen, bis er wieder an seinem Startpunkt angekommen ist.
    Implementiere das Verhalten in der Methode \textit{subTaskB()} in der Klasse \textit{Sheet3Task6}.
    \item Lasse nun Neo gegen den Uhrzeigersinn an der Wand entlang laufen, bis er wieder an seinem Startpunkt angekommen ist.
    Implementiere das Verhalten in der Methode \textit{subTaskC()} in der Klasse \textit{Sheet3Task6}.
    Was fällt dir im Vergleich zu vorhin auf?
    \item Anstatt Neo \textit{manuell} drei Mal im Uhrzeigersinn drehen zu lassen, wollen wir nun die fehlende Methode \textit{turnCounterClockwise()}
    implementieren. Finde dafür zunächst die richtige Klasse und die dazugehörige Methode.
    \item Wir wollen uns nun die Methodensyntex etwas genauer anschauen. Betrachte und erkläre dafür den Aufbau des folgenden Programmauszugs:
    \begin{lstlisting}
//Class Main
public static void main(String[] args) {
    A a = new A();
    if(a.condition(5, 30))
        //do stuff
    //do some more stuff
}

//Class A
private int b = 9;
public boolean condition(int a, int b){
    if(a > b)
        return true;
    return power(a) > this.b;
}

private int power(int a){
    return a*a;
}
    \end{lstlisting}
    \item Implementiere nun die fehlende Methode \textit{turnCounterClockwise()}. Im Folgenden wird die Syntax beispielshaft vorgegeben:
    \begin{lstlisting}
//Class Neo
public void moveTwoBlocks() {
    this.move();
    this.move();
}
    \end{lstlisting}
    \item Teste deine Implementierung aus der Teilaufgabe \textit{f)} indem du deinen Code in der Methode \textit{subTaskC()} anpasst.
    \item Nun wollen wir natürlich dafür sorgen, dass unsere Nachfolger auch verstehen was in der von uns implementierten Methode von 
    Teilaufgabe \textit{f)} passiert. Schreibe Javadoc für diese Methode.
    \item Die Dokumentation von Methoden soll überraschendes Verhalten bei der Ausführung verhindern. Beinhaltet deine Dokumentation auch 
    das Verhalten, welches bei gewünschter 180° Drehung von Neo mittels der Methode \textit{turnCounterClockwise()} auftritt?
\end{enumerate}