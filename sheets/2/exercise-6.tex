%!TEX root = ./jvk-blatt2.tex

\excercise{Wenn ich mich doch nur nach links drehen könnte}

\begin{enumerate}
    \item Instanziiere den Task und den Verifier. 
        Dabei soll wie in der \textit{Aufgabe 5b)} der SubTask jeweils angegeben werden.
    \item Lasse Neo im Uhrzeigersinn an der Wand entlang laufen, bis er wieder an seinem Startpunkt angekommen ist.
        Implementiere das Verhalten an der markierten Stelle in der Klasse \lstinline{Sheet3Task6}.
    \item Lasse nun Neo gegen den Uhrzeigersinn an der Wand entlang laufen, bis er wieder an seinem Startpunkt angekommen ist.
        Welche der Varianten ist die schnellere?
\end{enumerate}


\begin{Infobox}[Operationensyntax]
    Wir wollen uns nun den Operationensyntax etwas genauer anschauen. 
    Dafür betrachten wir beispielhaft den folgenden Programmcode:

    \begin{lstlisting}[breaklines=true, numbers=none]
public void run(){
    System.out.println(bar(5, 7));
}

private int b = 9;
public boolean bar(int a, int b){
    if(a > b)
        return true;
    return power(a) > this.b;
}

public int power(int a){
    return a*a;
}
    \end{lstlisting}

    Bei der Programmausführung gehen wir davon aus dass das Kommando \lstinline{run} ausgeführt wird. 
    Wir unterteilen die Erklärung der Syntax in folgende Blöcke:

    \begin{enumerate}[label=\roman*)]
        \item Grundlegender Sytaxaufbau: 
            Der Aufbau einer \textit{einfachen} Operation folgt zunächst einmal dem Schema 

            \begin{lstlisting}
Modifier Datentyp Operationname (Par a, Par b, ...){
    //do stuff
    return ...;
}
            \end{lstlisting}

            Welche Operationen sind also in dem obigen Codebeispiel gegeben und wie wurden diese benannt?
        \item Modifier: 
            Hier benutzen wir in dem Vorkurs vorerst ausschließlich \lstinline{public}. 
            Mehr dazu kommt dann in der Vorlesung.
        \item Datentyp: 
            Dieser gibt an, welche Art von Rückgabe die Methode liefert. 
            Die Rückgabe selber wird durch das Keyword \lstinline{return} signalisiert.
        \item \lstinline{return}:
            Nachdem \lstinline{return} aufgerufen wurde, wird sofort zu der aufrufenden Operation zurückgesprungen.\\
            Beispiel: Wenn die Operation \lstinline{bar} mit Parametern \lstinline{a = 5, b = 2} aufgerufen wird, ist die IF-Condition erfüllt und es wird sofort \lstinline{true} zurückgegeben.
            Dadurch wird die darunterliegende Zeile nicht ausgeführt.
    \end{enumerate}

\end{Infobox}


\begin{enumerate} \setcounter{enumi}{3}
    \item Anstatt Neo \textit{manuell} drei Mal im Uhrzeigersinn drehen zu lassen, wollen wir nun das fehlende Kommando \lstinline{turnCounterClockwise()} implementieren. 
        Finde dafür zunächst die richtige Klasse und die dazugehörige Stelle im Code, in welcher das Kommando eingefügt werden soll (diese Stelle ist auch durch ein Kommentar vorgegeben).

        Implementiere anschließend das fehlende Kommando \lstinline{turnCounterClockwise()}. 
        Im Folgenden wird die Syntax des Kommandos beispielshaft vorgegeben:
        
        \begin{lstlisting}
public void turnCounterClockwise() {
    //TODO
}
        \end{lstlisting}

    \item Teste deine Implementierung aus der Teilaufgabe \textit{f)} indem du deinen Code in der Operation \lstinline{subTaskC()} anpasst.
\end{enumerate}


\begin{Infobox}[Javadoc]
    Hier wollen wir nun erstmalig Javadoc für eine unserer Operationen schreiben. 
    Dafür betrachten wir vorerst die Syntax an einem Beispiel:

    \begin{lstlisting}[numbers=none]
    /**
    * Computes the greatest integer value of the logarithm 
    * of a given natural number and a given base. 
    * 
    * The computed value corresponds to the logarithm of the
    * given base rounded downwards. In order to determine
    * a unique solution, a base greater than 1 is required.
    * Furthermore, the logarithm of 0 is not defined for any 
    * natural number. In this case an exception is thrown.
    * 
    * @param n the natural number of which the logarithm is computed
    * @param b the base of the computed logarithm
    * @throws IllegalArgumentException if n is not a natural
    * number greater than 0 or b is a base for which an unique
    * solution cannot be computed.
    * @return the logarithm of the given base rounded downwards
    */
    public static int roundedLogarithm(int n, int b){
        if(n <= 0 || b <= 1)
            throw new IllegalArgumentException();
        
        int logValue = -1;
        int power = 1;
        while(power <= n){
            power = power * b;
            logValue = logValue + 1;
        }

        return logValue;
    }
    \end{lstlisting}

    Dabei kann ein großteil der Berechnung innerhalb der Methode ignoriert werden und ist nur zur Vollständigkeit mit angegeben.
    
    Der generelle Aufbau der Methodendokumentation startet hier mit einer kurzen Zusammenfassung gefolgt von einer ausführlichen Beschreibung der Methodenfunktion. 
    Für den Leser ist hierbei die konkrete Umsetzung innerhalb der Methode eher uninteressant, weshalb man sich in der Dokumentation auf die Verwendung und das Verhalten der Methode beschränkt. 
    Es sollte hierbei insbesondere darauf eingegangen werden, welche Vorbedingungen von der Methode benötigt werden und mit welchem Ergebnis der Benutzer anschließend rechnen darf.

    Nun gibt es noch ein paar interessante Keywords, die im Folgenden noch kurz erklärt werden:

    \begin{enumerate}[label=\roman*)]
        \item \lstinline{@param}: 
            Gefolgt von einem Parametername (bspw. \lstinline{@param n}) wird dem Leser die Bedeutung des Parameters erklärt.
        \item \lstinline{@return}: 
            Beschreibt den zurückgegebene Wert unter Annahme, dass die Eingabe korrekt ist.
        \item \lstinline{@throws}: 
        Gefolgt von der geworfenen Exception (bspw. \lstinline{@throws IllegalArgumentException}) wird hier beschrieben wann diese Exception geworfen wird.
    \end{enumerate}
\end{Infobox}


\begin{enumerate}\setcounter{enumi}{5}
    \item Nun wollen wir natürlich dafür sorgen, dass unsere Nachfolger auch verstehen was in der von uns implementierten Operation von Teilaufgabe \textit{f)} passiert. 
        Schreibe Javadoc für diese Operation.
    \item Die Dokumentation von Operationen soll überraschendes Verhalten bei der Ausführung verhindern. 
        Wenn du Neo um 180° umdrehen willst kannst du dafür jetzt turnClockwise und deine neue Operation benutzen. 
        Verhalten sich die Beiden Operationen dabei gleich?
        Wenn nein: Kannst du das unterschiedliche Verhalten vorhersagen indem du nur die Dokumentation der beiden Operationen vergleichst?
        Falls nicht war deine Dokumentation noch nicht genau genug du solltest sie nochmal verbessern.
\end{enumerate}
