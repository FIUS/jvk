    \excercise{Datentypen}
    Wie wir oben gesehen haben gibt es verschieden Typen, die wir zum Beispiel für Argumente und Rückgabewerte verwenden können\footnote{Wir werden später noch andere Stellen sehen, an denen Typen vorkommen.}.

    Zunächst ist jede Klasse die wir oder jemand anderes definieren ein Typ. Variablen von diesem Typ können Objekte dieser Klasse oder einer ihrer Kindklassen enthalten. Eine häufig verwendete Klasse, die schon von Java definiert wurde, ist \lstinline{String}, welche Zeichenketten speichert.

    Zusätzlich gibt es noch primitive Datentypen. Diese sind in der folgenden Tabelle aufgelistet:
    \begin{table}[h!]
        \centering
        \begin{tabular}{c|c|c}
        Name & Bechreibung & Werte \\
        \hline\hline
        \texttt{byte} & 8-Bit-Ganzzahl & \(-128,\dots, -1, 0, 1, \dots, 127\) \\
        \texttt{short} & 16-Bit-Ganzzahl & \(-32768, \dots, -1, 0, 1, \dots, 32767\)\\
        \texttt{int} & 32-Bit-Ganzzahl & \(-2\,147\,483\,648, \dots, -1, 0, 1, \dots ,2\,147\,483\,647\)\\
        \texttt{long} & 64-Bit-Ganzzahl & \(-9\,223\,372\,036\,854\,775\,808, \dots, -1, 0, 1, \dots ,9\,223\,372\,036\,854\,775\,807\)\\
        \hline
        \texttt{float} & 32-Bit-Gleitkommazahl & \(1.3f\), \(0.7e+12f\), \(3.1415927\) \\
        \texttt{double} & 64-Bit-Gleitkommazahl & \(1.5739\), \(0.134e+123\), \(3.141592653589793\) \\
        \hline
        \texttt{boolean} & Wahrheitswerte & \texttt{true}, \texttt{false} \\
        \texttt{char} & Zeichen & \texttt{'a'}, \texttt{'B'}, \texttt{'5'}, \texttt{'('} \\
        \end{tabular}
        \caption{Primitive Datentypen in Java}
        \label{tab:primitive_types}
    \end{table}
    % TODO Datentypen Aufgaben erstellen
