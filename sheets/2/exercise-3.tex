%!TEX root = ./jvk-blatt2.tex

\excercise{What If there is a Wall?!}

In dieser Aufgabe wollen wir uns Conditions näher anschauen.

\begin{enumerate}                           
    \item Instanziiere die Simulation wie bekannt und mache dich mit dieser vertraut.
        Was passiert bei der Ausführung?
\end{enumerate}


\begin{Infobox}[IF-Condition]
    Nun betrachten wir ein beispielhaftes IF-Statement:

    \begin{lstlisting}[breaklines=true, numbers=none]
if(player.canMove()){
    player.move();
} else{
    System.out.println("Morpheus: There is a difference between knowing the path and walking the path")
}
    \end{lstlisting}

    Das Kommando \textit{canMove()} gibt einen Boolean Wert zurück (also \textit{true} oder \textit{false}).
    Falls sich also vor dem Spieler kein Hindernis befindet gibt die Abfrage canMove() "true" zurück, da es wahr ist, dass der spieler sich frei nach vorne bewegen kann.
    In diesem Fall ist die Bedingung des IF-Statements erfüllt. 
    Dadurch wird der erste Block ausgeführt, was hier dem Spieler erlaubt sich vorwärts zu bewegen. 

    In dem anderen Fall, dass sich doch ein Hindernis vor dem Spieler befindet, ist die Bedingung des IF-Statements nicht erfüllt und der zweite Block wird ausgeführt.
    In diesem Fall bekommen wir hier eine erleuchtende Weißheit von Morpheus.
\end{Infobox}


\begin{enumerate} \setcounter{enumi}{1}
    \item Betrachte die Methode \textit{movement} der Klasse \textit{Sheet2Task3}. 
        Füge das obige IF-Statement an einer geeigneten Stelle ein und führe die Simulation erneut aus.
    \item Das Hindernis wurde in der selben Klasse an der markierten Stelle in der Methode \textit{buildEnvironment} eingefügt.
        Kommentiere diese Zeile aus und führe die Simulation aus. Was kannst du beobachten?
    \item Jetzt soll die Wand wieder einkommentiert werden. Behandle also den \q{ELSE} Fall so, dass Neo in beiden Fällen die Telefonstation erreicht.

\end{enumerate}
