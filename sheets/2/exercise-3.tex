%!TEX root = ./jvk-blatt2.tex

\excercise{What If there is a Wall?!}

In dieser Aufgabe wollen wir uns Conditions näher anschauen.

\begin{enumerate}                           
    \item Instanziiere die Simulation wie bekannt und mache dich mit dieser vertraut.
        Was passiert bei der Ausführung?
\end{enumerate}


\begin{Infobox}[IF-Condition]
    Nun betrachten wir ein beispielhaftes IF-Statement:

    \begin{lstlisting}[breaklines=true, numbers=none]
if (player.canMove()) {
    player.move();
} else {
    System.out.println("Morpheus: There is a difference between knowing the path and walking the path")
}
    \end{lstlisting}

    Die Abfrage \lstinline{canMove()} gibt einen Boolean Wert zurück (also \lstinline{true} oder \lstinline{false}).
    Falls sich also vor dem Spieler kein Hindernis befindet gibt die Abfrage \lstinline{canMove()} \lstinline{true} zurück, da es wahr ist, dass der spieler sich frei nach vorne bewegen kann.
    In diesem Fall ist die Bedingung des IF-Statements erfüllt. 
    Dadurch wird der erste Block ausgeführt, was hier dem Spieler erlaubt sich vorwärts zu bewegen. 

    In dem anderen Fall, dass sich doch ein Hindernis vor dem Spieler befindet, ist die Bedingung des IF-Statements nicht erfüllt und der zweite Block wird ausgeführt.
    In diesem Fall bekommen wir hier eine erleuchtende Weißheit von Morpheus.

    Die Kommandos in den Zwei Code-Blöcken können nach bedarf ausgetauscht werden.
    In einem Code-Block dürfen sogar beliebig viele Kommandos, Abfragen oder wieder ganze IF-Statements stehen.
    Auch die genutzte Abfrage kann geändert werde, was in späteren Aufgaben noch erklärt wird.
\end{Infobox}


\begin{enumerate} \setcounter{enumi}{1}
    \item Betrachte die Operation \lstinline{movement} der Klasse \lstinline{Sheet2Task3}. 
        Füge das obige IF-Statement an einer geeigneten Stelle ein und führe die Simulation erneut aus.
        Die vorhandenen \lstinline{.move()} Kommandos nach dem IF-Statement sollten in den ersten Code-Block vom IF verschoben werden.
        Der \lstinline{else} Code-Block sollte erstmal leer bleiben.
    \item Starte das Fenster neu.
        Lösche die Wand in dem Spielfenster.
        Dazu musst du oben rechts das rote Minus auswählen und dann auf die Wand klicken.
        Starte dann die Simulation.
        Neo sollte sich jetzt anders verhalten obwohl du den code nicht geändert hast!
    \item Jetzt soll Neo die Telefonstation auch erreichen wenn die Wand nicht gelöscht wurde.
        Behandle also den \lstinline{else} Fall so, dass Neo in beiden Fällen die Telefonstation erreicht.
        Gib Neo dafür im \lstinline{else} Code-Block die notwendigen Kommandos.

        Überprüfe deinen Code indem du ihn einmal ausführst ohne die Wand zu löschen und einmal wenn du die Wand davor gelöscht hast.

\end{enumerate}
