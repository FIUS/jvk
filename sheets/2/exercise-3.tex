%!TEX root = ./jvk-blatt2.tex

\excercise{What If there is a Wall?!}

In dieser Aufgabe wollen wir uns Conditions (Bedingungen / Abfragen) näher anschauen.

\begin{enumerate}                           
    \item Instanziiere die Simulation wie bekannt für diese Aufgabe und mache dich mit dieser vertraut.
        Was passiert bei der Ausführung?
\end{enumerate}


\begin{Infobox}[IF-Condition]
    Nun betrachten wir ein beispielhaftes IF-Statement:

    \begin{lstlisting}[breaklines=true, numbers=none]
if (player.canMove()) {
    player.move();
} else {
    System.out.println("Morpheus: There is a difference between knowing the path and walking the path")
}
    \end{lstlisting}

    Die Abfrage \lstinline{canMove()} gibt einen Boolean Wert zurück (also \lstinline{true} oder \lstinline{false}, \texttt{Ja} oder \texttt{Nein}).\\
    Falls sich also vor dem Spieler kein Hindernis befindet gibt die Abfrage \lstinline{canMove()} \lstinline{true} zurück, da es wahr ist, dass der Spieler sich frei nach vorne bewegen kann.\\
    In diesem Fall ist die Bedingung des IF-Statements erfüllt.\\
    Dadurch wird der erste Block ausgeführt, was hier dem Spieler erlaubt sich vorwärts zu bewegen.\\
    Es wird also \lstinline{player.move()} ausgeführt.\\

    In dem anderen Fall, dass sich doch ein Hindernis vor dem Spieler befindet, ist die Bedingung des IF-Statements nicht erfüllt und der zweite Block wird ausgeführt.\\
    Morpheus schreibt uns also eine erleuchtende Weisheit in die Console.\\

    Die Kommandos in den jeweiligen Code-Blöcken können nach Bedarf ausgetauscht werden.
    In einem Code-Block dürfen sogar beliebig viele Kommandos, Abfragen (If-Statements) aber auch Schleifen (kommen in späteren Aufgaben) stehen.\\
    Auch die genutzte Abfrage kann entsprechend unserer Situation angepasst werden, was in späteren Aufgaben noch erklärt wird.
\end{Infobox}


\begin{enumerate} \setcounter{enumi}{1}
    \item Betrachte die Operation \lstinline{movement} der Klasse \lstinline{Sheet2Task3}.\\
        Füge das obige IF-Statement an einer geeigneten Stelle ein und führe die Simulation erneut aus.
        Die dann noch vorhandenen \lstinline{.move()} Kommandos welche nach dem IF-Statement sind sollten in den ersten Code-Block vom IF verschoben werden.
        Der \lstinline{else} Code-Block soll erstmal leer bleiben.
    \item Starte das Fenster neu.
        Lösche die Wand in dem Spielfenster.\\
        Dazu musst du oben rechts das rote Minus auswählen und dann auf die Wand klicken.
        Starte dann die Simulation.\\
        Neo sollte sich jetzt anders verhalten, obwohl du den Code nicht geändert hast!
    \item Jetzt wollen wir den Code so anpassen, dass Neo die Telefonstation auch erreicht wobei die Wand nicht mehr gelöscht werden soll.

        Du musst also einmal den Fall betrachten, dass keine Wand im Weg ist (der erste If-Block) und einmal den Fall, dass eine Wand im Weg ist (der Else-Block).\\
        Implementiere nun das beschriebene Verhalten.

        Überprüfe deinen Code indem du einmal die Simulation startest ohne die Wand zu löschen und einmal wenn du die Wand davor gelöscht hast.
        In beiden Fällen sollte Neo die Telefonzelle erreichen.

\end{enumerate}
