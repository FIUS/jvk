\excercise{What If there is a Wall?!}
In dieser Aufgabe wollen wir uns Conditions näher anschauen.
\begin{enumerate}[label=\alph*)]
    \item Wir wollen nun die Syntax einer If-Bedingung genauer betrachten. Überlege dir dazu, was der folgende Programmcode ausgibt:
    \begin{lstlisting}[breaklines=true]
if (true){
    System.out.println("Condition A is satisfied!");
}
    
// --------------------------------------------------------
    
if (42 > 42 || false){
    System.out.println("Condition B is satisfied!");
} else{
    System.out.println("Condition B is not satisfied!");
}
        
// --------------------------------------------------------
        
if ((5*2 > 3) && (false || 6 == 9)){
    System.out.println("Condition C is satisfied!");
} else if (!(2 > 3)){
    System.out.println("Condition C is not satisfied, but Condition D is satisfied!");
} else{
    System.out.println("Neither Condition C nor Condition D are satisfied!");
}
    \end{lstlisting}
                            
    \item Instanziiere die Simulation wie bekannt und mache dich mit dieser vertraut. Die Simulation soll zwei Mal ausgeführt werden:
    \begin{enumerate}
        \item[i)] Ohne Einfügen einer Wand zwischen Neo und der Telefonstation.
        \item[ii)] Mit Einfügen einer Wand zwischen Neu und der Telefonstation. 
    \end{enumerate}
    Was beobachtest du?
    \item Betrachte den markierten Teil der Methode \textit{movement} aus der Klasse \textit{Sheet2Task3}. Was sorgt dafür, dass es bei der Simulation 
    mit Hindernis zu keiner Exception kommt?
    \item Wähle ein fixes Hindernis auf dem Pfad zwischen Neo und der Telefonstation. Neo möchte trotz Hindernis natürlich dennoch die Telefonstation erreichen.
    Implementiere das gewünschte Verhalten in dem dafür vorgesehenen Bereich.
    \item Bonus: Verallgemeinere die Aufgabenstellung von Teilaufgabe \textit{d)} für ein beliebiges Hindernis auf dem direkten Pfad zwischen Neo und der Telefonstation.
\end{enumerate}
