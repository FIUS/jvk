% !TEX root = ./jvk-blatt2.tex

\excercise{Vor- und Nachbedingungen}

In dieser Aufgabe wollen wir eine neue Funktion von Totoro einführen.
Totoro kann Nüsse mit den Kommandos \lstinline{collectNut()} und \lstinline{dropNut()} aufheben und ablegen.

\begin{enumerate}
\item Instanziiere die Simulation wie bekannt (\lstinline{Sheet2Task3} und \lstinline{Sheet2Task3Verifier}) und mache dich mit dieser vertraut.
\item Auf dem Spielfeld siehst du viele Nüsse.
Wenn Totoro auf einem Feld mit Nuss steht, kann er mit dem Kommando \lstinline{collectNut()} eine Nuss aufheben.\\
Lass Totoro nun alle Nüsse auf dem Spielfeld aufheben.
\item Mit dem Kommando \lstinline{dropNut()} kannst du eine Nuss wieder auf dem Feld ablegen, auf welchem sich Totoro befindet.\\
Verwende das \lstinline{dropNut()} Kommando, um alle Nüsse die Totoro aufgesammelt hat auf das Feld (3,3) zu legen.\\
\textbf{Hinweis:} Auf dem Spielfeld liegen genau 6 Nüsse.
\end{enumerate}


\begin{Infobox}[Vor- und Nachbedingungen]
    Damit Operationen richtig funktionieren, müssen oft bestimme Bedingungen gelten, bevor man diese aufruft.
Zum Beispiel darf sich bei dem \lstinline{move()} Kommando aus der letzten Aufgabe vor dem Spieler kein Busch befinden.
Da diese Bedingungen gelten müssen, bevor man die Operation aufruft, nennt man sie Vorbedingungen.\\
Meistens findet man diese Vorbedingungen im Javadoc Kommentar (Dokumentation) der Operation. 
Diesen kann man sehen, wenn man mit der Maus über dem Funktionsnamen hovert.\\

So wie bei Vorbedingungen bestimmte Zustände gelten müssen, bevor man die Operation aufrufen darf, sind Nachbedingungen Zustände, die nach einer Operation gelten.\\
Nachdem Ausführen von \lstinline{move()} hat sich Totoro ein Feld nach vorne (in seine Blickrichtung) bewegt.\\
Das gilt allerdings nur, wenn vor dem Aufruf von \lstinline{move()} die Vorbedingungen erfüllt sind, wie zum Beispiel, dass sich kein Busch vor Totoro befindet.\\


\end{Infobox}

\newpage

\begin{Infobox}[JavaDoc]
    Im Zusammenhang mit Vor- und Nachbedingungen, ist wichtig zu erwähnen, dass diese, neben einer detaillierteren Beschreibung der Funktion, im Javadoc einer Operation zu finden sein sollten.
JavaDoc befindet sich als spezieller Kommentar in Quellcode vor einer Operation oder Klasse.
Wie andere Kommentare wird er nicht als Programmcode ausgeführt, sondern dient zur Dokumentation der Funktion der zugeordneten Elemente.

In den letzten Aufgaben haben wir bis jetzt immer recht \enquote{einfache} Methoden verwendet, deren Zweck du vielleicht ganz gut aus dem Quellcode oder Beispielen ableiten konntest.
Beim Schreiben von eigenen Methoden ist es daher sehr wichtig darauf zu achten, dass der Name der Funktion und die Werte, die diese ''empfangen'' kann, aussagekräftige und eindeutige Namen haben.
Vielleicht sind dir die Funktion von machen Methoden allerdings auch noch nicht ganz klar geworden oder der Name hat dir  zu wenig Kontext gegeben, um die Funktionsweise der Methodezu verstehen.
Genau hier wird die Dokumentation in Form von JavaDoc wichtig.

    \begin{lstlisting}[numbers=none]
    /**
     * Turn Totoro around.
     *
     * This operation turns Totoro around by calling
     * turnClockWise twice. This operation will fail 
     * if Totoro is not on a Playfield of a Simulation.
     *
     */
     public void turnAround() {
        this.turnClockWise();
        this.turnClockWise();
    }
    \end{lstlisting}

    Ein Javadoc Kommentar unterscheidet sich nur dahingehend von einem normalen Kommentar, dass dieser mit \lstinline{/**} statt \lstinline{/*} anfängt.\\

Der erste Satz ist eine kurze Beschreibung, was die Operation macht.
Gefolgt davon kommt eine ausführliche Beschreibung der Operation oder Klasse.

Der Kommentar sollte alles enthalten, was man jemals über die Operation wissen muss.

Im Javadoc Kommentar sollen dabei insbesondere Vorbedingungen beschrieben werden.
Wenn dies nicht passiert oder der JavaDoc ungenau ist, ist es sehr wahrscheinlich, dass daraus ein Bug entsteht.

IntelliJ kann diese speziellen JavaDoc-Kommentare erkennen und anzeigen, wenn man mit der Maus im Quelltext über einen Aufruf einer Operation fährt.
Außerdem kann man im Quellcode durch \lstinline{Ctrl + Linksklick} auf Operationen oder Klassen zu den Definitionen von diesen gelangen, um die zugehörige Javadoc
Letzteres kann man natürlich auch nutzen, um die Funktion einer Operation durch das Lesen des Quellcodes besser zu verstehen.
\end{Infobox}

\begin{enumerate}\setcounter{enumi}{3}
    \item Die Kommandos \lstinline{collectNut()} und \lstinline{dropNut()} funktionieren nicht immer.
    
        Welche Bedingungen müssen gelten, damit du die Kommandos korrekt benutzen kannst?
        Unter welchen Bedingungen kannst du sie nicht fehlerfrei benutzen?

        Überprüfe deine Vermutung  auf dem Spielfeld.
\end{enumerate}
