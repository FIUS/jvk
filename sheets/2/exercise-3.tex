% !TEX root = ./jvk-blatt2.tex

\excercise{Vor- und Nachbedingungen}

In dieser Aufgabe wollen wir eine neue Funktion von Neo einführen. 
Wer die Doku aufmerksam gelesen hat, kennt sie vielleicht schon:
Neo kann Goldmünzen aufheben und niederlegen. 
Dazu stehen die Kommandos \lstinline{collectCoin()} und \lstinline{dropCoin()} zur Verfügung.

\begin{enumerate}
    \item Instanziiere die Simulation wie bekannt (\lstinline{Sheet2Task3} und \lstinline{Sheet2Task3Verifier}) und mache dich mit dieser vertraut.
    \item Auf dem Spielfeld siehst du viele Goldmünzen.
        Wenn Neo auf einem Feld mit Münzen steht kann er mit dem Kommando \lstinline{collectCoin()} eine Münze aufheben.\\
        Lass Neo nun alle Goldmünzen auf dem Spielfeld aufheben.
    \item Mit dem Kommando \lstinline{dropCoin()} kannst du eine Münze wieder auf dem Spielfeld ablegen auf welchem sich Neo befindet.\\
        Verwende das \lstinline{dropCoin()} Kommando, um alle Münzen die Neo aufgesammelt hat auf das Feld (3,3) zu legen.
    \item Die Kommandos \lstinline{collectCoin()} und \lstinline{dropCoin()} funktionieren nicht immer.
        Welche Bedingungen müssen stimmen damit du die Kommandos korrekt benutzen kannst und wann funktionieren sie nicht?\\
        \textbf{Tipp}: Überprüfe deine Vermutung auch auf dem Spielfeld.\\
\end{enumerate}


\begin{Infobox}[Vor- und Nachbedingungen]
    Damit Operationen richtig funktionieren, müssen oft bestimme Bedingungen gelten, bevor man diese aufruft.\\
    Z.B. darf sich bei dem \lstinline{move()} Kommando, wie wir in der letzten Aufgabe gelernt haben, vor dem Spieler keine Wand befinden.\\
    Da diese Bedingungen gelten müssen bevor man die Operation aufruft, nennt man sie Vorbedingungen.\\
    Meistens findet man diese Vorbedingungen im Javadoc Kommentar (Dokumentation) der Operation. Welchen man sehen kann wenn man mit der Maus, wie in Blatt 1 Aufgabe 2 beschrieben, über dem Funktionsnamen hovert.\\

    So wie bei Vorbedingungen bestimmte Zustände gelten müssen bevor man die Operation aufrufen darf, sind Nachbedingungen Zustände die nach einer Operation gelten.\\
    Nach dem Ausführen von \lstinline{move()} hat sich Neo eine Position nach vorne (in seine Blickrichtung) bewegt.\\
    Das gilt allerdings, nur wenn vor dem Aufruf von \lstinline{move()} die Vorbedingungen erfüllt sind z.B. es darf sich keine Wand vor Neo befinden.\\


\end{Infobox}
