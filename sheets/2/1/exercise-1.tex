\excercise{Primitive Datentypen und Operationen}

\begin{enumerate}
    \item In dieser Aufgabe geht es darum mit Neo einen simplen Taschenrechner zu bauen. Neo soll die Anzahl an Münzen die sich auf seinem Feld befinden auf die Anzahl der Münzen die sich auf dem Feld rechts neben ihm befinden addieren, die Anzahl der Münzen auf den Feldern ist zufällig. Das Ergebnis dieser Addition soll Neo auf auf das Feld unter seinem aktuellen in Münzen ausgeben. Dazu könnt ihr die Funktion this.dropMultipleCoins(n) mit n als Anzahl der Münzen die abgelegt werden sollen verwenden. 

    \item Schreibt nun den Code aus der vorherigen Teilaufgabe so um, dass die Anzahl der Münzen auf den zwei Feldern multipliziert und nicht addiert werden.
    
    \item Welche weiteren Rechenoperationen kann dieser simple Taschenrechner noch ausführen und wieso kann er nicht alle gängigen Rechenoperationen ausführen?
\end{enumerate}
