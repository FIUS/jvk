  %!TEX root = ./jvk-blatt2.tex


\excercise{The Player Enters the Game}


\begin{enumerate}
    \item Instanziiere die Simulation wie bekannt und mache dich mit dieser vertraut. 
        Benutze dabei den Task aus der Klasse \lstinline{Sheet2Task1} und den Verifier aus \lstinline{Sheet2Task1Verifier}.
    \item Lade Neo in die Matrix.
        Dafür musst du zuerst ein objekt der Klasse \lstinline{Neo} erstellen und in einer Variable (z.B. \lstinline{player}) speichern.
        Dann kannst du \lstinline{player} wie in Aufgabenblatt 1 Aufgabe 5 die Münze mit dem \lstinline{PlayfieldModifier} auf dem Spielfed platzieren.

        Dein code sollte in etwa wie folgt aussehen.

        \begin{lstlisting}[firstnumber=14]
public void run(Simulation sim) {
    // Your Implementation here

    PlayfieldModifier pm = new PlayfieldModifier(sim.getPlayfield());
    Neo player = new Neo();
    pm.placeEntityAt(player, new Position(1,1));
        \end{lstlisting}

        Überprüfe im Task Status Tab ob dein code korrekt funktioniert.
    \item Damit sich Neo bewegt musst du ihm Kommandos geben.
        Du kannst Neo die folgenden Kommandos geben.

        \begin{lstlisting}[firstnumber=20]
    player.move();
    player.moveIfPossible();
    player.turnClockWise();
        \end{lstlisting}

        Teste alle 3 Kommandos.
        Damit Neo sich bewegen kann muss die Simulation mit dem Play Button gestartet werden!
    \item Neo kann sich nur nach rechts drehen.
        Wie kannst du Neo trotzdem dazu bringen sich nach rechts (oder nach hinten) umzudrehen?

        Hinweis: Du kannst Kommandos mehrfach hintereinander aufrufen.
    \item Gib Neo die folgenden Anweisungen:
        
        {
            \setenumerate{label=\roman*.}
            \begin{enumerate}
                \item Bewege dich 2 Schritte geradeaus
                \item Dann drehe dich nach rechts
                \item Gehe einen Schritt vor
                \item Drehe dich nach links
                \item Gehe vier Schritte geradeaus
                \item Drehe dich nach rechts
                \item Als letztes gehe einen Schritt geradeaus
            \end{enumerate}
        }

        Überprüfe im Task Status Tab ob du Neo die richtigen Kommandos gegeben hast.
\end{enumerate}

