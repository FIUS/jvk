  %!TEX root = ./jvk-blatt2.tex



\excercise{Totoro Enters the Game}
\label{ex1}


\begin{Infobox}[Benennung der Task Klassen]
    Wie du wahrscheinlich festgestellt hast, haben wir ein Namensschema für die Klassen.\\
    Für Blatt \textbf{X} und Aufgabe \textbf{Y} sollst du die Klassen \lstinline{SheetXTaskY} und \lstinline{SheetXTaskYVerifier} nutzen.\\
    Ebenso musst du für jede Aufgabe in der Main Klasse die \lstinline{SheetXTaskY} und \lstinline{SheetXTaskYVerifier} anpassen.
    Deinen Code musst du (fast) immer in \lstinline{SheetXTaskY} schreiben.
\end{Infobox}

\begin{enumerate}
    \item Instanziiere die Simulation wie bekannt und mache dich mit dieser vertraut. 
        Benutze dabei den Task \lstinline{Sheet2Task1} und den Verifier aus \lstinline{Sheet2Task1Verifier}.
    \item Lade Totoro in die Matrix.
        Dafür musst du zuerst ein Objekt der Klasse \lstinline{Totoro} erstellen und in einer Variable (z.B. \lstinline{totoro}) speichern.
        Dann kannst du \lstinline{totoro} und die Nuss wie bereits gelernt mit dem \lstinline{PlayfieldModifier} auf dem Spielfeld platzieren.

        Dein Code sollte in etwa wie folgt aussehen.

        \begin{lstlisting}[firstnumber=14]
public void run(Simulation sim) {
    // Your Implementation here

    PlayfieldModifier pm = new PlayfieldModifier(sim.getPlayfield());
    Totoro totoro = new Totoro();
    pm.placeEntityAt(totoro, new Position(1,1));

}
        \end{lstlisting}

        Überprüfe im \fbox{Task Status} Tab, ob dein Code korrekt funktioniert.
        \item Damit sich Totoro bewegt, musst du ihm eines der folgenden Kommandos geben und die Simulation mit dem Play Button starten. 
        Im Simulationsfenster musst du dann den Play Button links oben drücken. \\
        \textbf{Tipp:} Neben dem Play Button im Simulationsfenster ist ein Geschwindigkeitsschieberegler, mit dem du die Simulationsgeschwindigkeit anpassen kannst.

        \begin{lstlisting}[firstnumber=20]
    totoro.move();
    totoro.moveIfPossible();
    totoro.turnClockWise();
        \end{lstlisting}

        Teste alle 3 Kommandos.\\
        \item Mit dem gegebenen Kommando, kann sich Totoro nur nach rechts drehen.\\
        Finde nun einen Weg, wie du Totoro dazu bringen kannst, sich nach links (oder nach hinten) zu drehen?
        
        \textbf{Hinweis:} Du kannst Kommandos mehrfach hintereinander aufrufen.
\newpage
        \item Gib Totoro nun die folgenden Anweisungen:
        
        {
            \setenumerate{label=\roman*.}
\begin{enumerate}
\item Bewege dich 2 Schritte geradeaus
\item Drehe dich nach rechts
\item Gehe einen Schritt vor
\item Drehe dich nach links
\item Gehe vier Schritte geradeaus
\item Drehe dich nach rechts
\item Gehe als letztes einen Schritt geradeaus
\end{enumerate}
}

Überprüfe im \fbox{Task Status} Tab, ob du Totoro die richtigen Kommandos gegeben hast. 
Achte darauf, dass Totoro genau die in der Aufgabe genannten Kommandos ausführt.
\end{enumerate}

