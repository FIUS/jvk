  %!TEX root = ./jvk-blatt2.tex


\excercise{The Player Enters the Game}

\begin{Infobox}[Benennung der Task Klassen]
    Wie du wahrscheinlich festgestellt hast, haben wir ein Namensschema für die Klassen.\\
    Also für Blatt \textbf{X} und Aufgabe \textbf{Y} sollst du die Klassen \lstinline{SheetXTaskY} und \lstinline{SheetXTaskYVerifier} nutzen.\\
    Ebenso musst du für jede Aufgabe in der Main Klasse die \lstinline{SheetXTaskY} und \lstinline{SheetXTaskYVerifier} anpassen.
    Deinen Code musst du (fast) immer in \lstinline{SheetXTaskY} schreiben.
\end{Infobox}

\begin{enumerate}
    \item Instanziiere die Simulation wie bekannt und mache dich mit dieser vertraut. 
        Benutze dabei den Task \lstinline{Sheet2Task1} und den Verifier aus \lstinline{Sheet2Task1Verifier}.
    \item Lade Neo in die Matrix.
        Dafür musst du zuerst ein Objekt der Klasse \lstinline{Neo} erstellen und in einer Variable (z.B. \lstinline{player}) speichern.
        Dann kannst du den \lstinline{player} und die Münze wie bereits gelernt mit dem \lstinline{PlayfieldModifier} auf dem Spielfeld platzieren.

        Dein Code sollte in etwa wie folgt aussehen.

        \begin{lstlisting}[firstnumber=14]
public void run(Simulation sim) {
    // Your Implementation here

    PlayfieldModifier pm = new PlayfieldModifier(sim.getPlayfield());
    Neo player = new Neo();
    pm.placeEntityAt(player, new Position(1,1));
        \end{lstlisting}

        Überprüfe im \fbox{Task Status} Tab ob dein Code korrekt funktioniert.
    \item Damit sich Neo bewegt, musst du ihm eines der folgenden Kommandos geben und die Simulation mit dem Play Button starten. Im Simulationsfenster musst du dann den Play Button links oben drücken. \\
    Tipp: Neben dem Play Button im Simulationsfenster ist ein Geschwindigkeitsschieberegler, mit dem du die Simulationsgeschwindigkeit anpassen kannst.

        \begin{lstlisting}[firstnumber=20]
    player.move();
    player.moveIfPossible();
    player.turnClockWise();
        \end{lstlisting}

        Teste alle 3 Kommandos.\\
    \item Mit den gegebenen Kommando, kann sich Neo nur nach rechts drehen.\\
        Finde un einen Weg, wie du Neo dazu bringen kannst, sich nach links (oder nach hinten) umzudrehen?

        Hinweis: Du kannst Kommandos mehrfach hintereinander aufrufen.
    \item Gib Neo nun die folgenden Anweisungen:
        
        {
            \setenumerate{label=\roman*.}
            \begin{enumerate}
                \item Bewege dich 2 Schritte geradeaus
                \item Dann drehe dich nach rechts
                \item Gehe einen Schritt vor
                \item Drehe dich nach links
                \item Gehe vier Schritte geradeaus
                \item Drehe dich nach rechts
                \item Als letztes gehe einen Schritt geradeaus
            \end{enumerate}
        }

        Überprüfe im \fbox{Task Status} Tab ob du Neo die richtigen Kommandos gegeben hast, dazu musst du darauf achten, dass Neo nur diese Kommandos ausführt.
\end{enumerate}

