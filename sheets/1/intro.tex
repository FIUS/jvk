% !TeX root = ./jvk-blatt1.tex

\begin{center}
	Hallo und herzlich Willkommen zum diesjährigen Java Vorkurs!\\
\end{center}
Hier werden wir mit dir gemeinsam eine Grundlage in der objektorientierten Programmiersprache \textit{Java} aufbauen, so dass Du in dem Fach \textit{Programmierung und Softwareentwicklung} - kurz PSE - im kommenden Semester nicht bei null beginnen musst.\newline
Falls du bei der Bearbeitung des Blatts Fragen oder Probleme haben solltest, kannst du dich jederzeit gerne bei den Tutoren melden oder einfach mal deine Sitznachbar*innen fragen.\newline
Wir empfehlen die Aufgaben nicht nur \enquote{im Kopf} zu bearbeiten, sondern dir Notizen auf einem Blatt Papier, auf deinem Laptop oder einer Datei auf einem USB-Stick/ in der Cloud zu machen. Erfahrungsgemäß bringt dir das selber mehr und du kannst deinen Mitstudent*innen ggf. mit deinen notierten Antworten schnell weiterhelfen oder mit ihnen diskutieren, wenn es Differenzen geben sollte.\\\\
\textbf{Voraussetzungen:}
\begin{itemize}
	\item Eine halbwegs aktuelle Version des \textit{Java Development Kits}\\
	$\rightarrow$ \href{\jdkTutorial}{Hier} gehts zum Tutorial, welches bitte nur mit \textit{Kopfhörern} angeschaut wird
	\item Eine funktionierende \textit{IDE}, zum Beispiel \textit{Eclipse}. \newline \textit{Eclipse} Installation: \newline
	$\rightarrow$ Wenn du bei der Vorstellung des Installationsvorgangs durch die Tutoren nicht ganz mitgekommen sein solltest, schau dir (bitte mit Kopfhörern) unser \href{\eclipseTutorial}{Tutorial} an.\\
	$\rightarrow$ Falls das nicht möglich sein oder bei der Installation Probleme auftreten sollten, kannst du dich auch gerne melden. Unsere Tutoren helfen dir gerne weiter.
	\item Die aktuelle Version der \texttt{"\jvkpackage"}-Datei\\
	$\rightarrow$ Den Download-Link findest du \href{\jvkpackageurl}{hier}
\end{itemize}
\textbf{Lernziele:}
\begin{itemize}
	\item \ref{ex1}: Eine Einführung im Umgang mit der integrierten Entwicklungsumgebung Eclipse, den Import von Maven Projekten und dem Ausführen dieser. Anschließend sollen erste, kleinere Code-Edits durchgeführt werden.
	\item \ref{ex2}: \textbf{Nachschlag-Werk:} Eine Sammlung von Begriffserklärungen mit vielen Beispielen.\\
	$\rightarrow$ Auch für Aufgaben auf späteren Übungsblättern nützlich
	\item \ref{ex3}: Eine kleine Aufgabe, um dich in der UI unseres Games zurecht zu finden und um zu lernen mit den bereitgestellten Funktionen umzugehen.
\end{itemize}
\newpage
