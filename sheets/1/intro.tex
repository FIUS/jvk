% !TeX root = ./jvk-blatt1.tex


\begin{center}
	Hallo und herzlich willkommen zum diesjährigen Java Vorkurs!\\
\end{center}
Wie Du dir vielleicht schon denken kannst werden wir hier versuchen, gemeinsam eine Grundlage in der objektorientierten Programmiersprache \textbf{Java} aufzubauen, so dass Du in dem Fach \textit{Programmieren und Softwareentwicklung}, kurz PSE im kommenden Semester nicht ganz von 0 aus starten musst.\\\\
Falls Du bei der Bearbeitung des Blatts Fragen oder Probleme haben solltest, kannst Du Dich immer gerne bei den Tutoren melden oder einfach mal deinen Nachbar/ deine Nachbarin rechts oder links von Dir fragen.\\\\
Wir empfehlen die Aufgaben nicht nur \enquote{im Kopf} zu bearbeiten, sondern Dir Notizen auf einem Blatt Papier, auf deinem Laptop oder einer Datei auf einem USB-Stick/ in der Cloud zu machen. Erfahrungsgemäß bringt Dir das selber mehr und du kannst deinen Mitstudenten ggf. mit deinen notierten Antworten weiterhelfen oder mit ihnen diskutieren, wenn es Differenzen geben sollte.\\\\
\textbf{Voraussetzungen:}
\begin{itemize}
	\item Eine halbwegs aktuelle Version des \textit{Java Development Kits}
	\item Die \textit{Eclipse IDE} ist korrekt installiert\\
	$\rightarrow$ Wenn du Dir bei der Vorstellung der Installation durch die Tutoren nicht ganz mitgekommen sein solltest, schau dir unser \href{https://youtu.be/zxH3G1MTrVs}{Tutorial} an\\
	$\rightarrow$ Wenn bei der Installation Probleme auftreten sollten, kannst du dich auch gerne melden
	\item Die aktuelle Version der \texttt{\jvkpackage} Datei\\
	$\rightarrow$ Den Link zum downloaden findest du \href{\jvkpackageurl}{hier}
\end{itemize}
\textbf{Lernziele:}
\begin{itemize}
	\item Aufgabe 1: Eine Einführung in dem Umgang mit der integrierten Entwicklungsumgebung Eclipse, den Import von Maven Projekten und den Ausführen dieser. Anschließend sollen die ersten kleinen Code-Edits durchgeführt werden.
	\item Aufgabe 2:
	\item Aufgabe 3:
	\item Aufgabe 4:
\end{itemize}

%\lipsum[1]