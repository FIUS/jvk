% !TeX root = ../jvk-blatt1.tex

\excercise{Einrichtung von Eduroam}
\label{ex1}

In diesem Kapitel wollen wir dir zeigen wie man Eduroam einrichtet. Eduroam ist das W-Lan an der Uni und dementsprechend ist es für dein Studium und den Java Vorkurs sehr praktisch, wenn du es benutzen kannst.
 
\section*{Für Windows, Android und IOS}
Gehe zuerst auf die Website \href{\eduroamurl}{geteduroam.app} und wähle den schwarzen Button mit deinem Betriebssystem aus. Danach wird ein Programm heruntergeladen, mit dem du Eduroam einrichten kannst.\newline
Stelle nun sicher, das dein Laptop/Smartphone nicht über das ''normale'' Fenster zum W-Lan einrichten mit Eduroam verbunden ist. Falls dein Laptop/Smartphone schon über das ''normale'' Fenster zum W-Lan einrichten mit Eduroam verbunden ist, lösche diese Verbindung.\newline
Führe jetzt das heruntergeladene Programm aus und folge den Anweisungen, wähle als Institution die Universität Stuttgart und als Profil ''STUDENT''. Gib deine komplette st-Nummer (also stxxxxxx@stud.uni-stuttgart.de) und das Passwort zu deinem st-Account ein.\newline
\textit{Hinweis:} das Programm überprüft nicht, ob du dein Passwort richtig eingegeben hast.\newline
Nun solltest dein Laptop/Smartphone sich mit Eduroam verbinden.\newline

\section*{Für Mac}
Lade dir zunächst die .mobileconfig-Datei herunter, indem du auf \href{\eduroamurllinux}{uni-stuttgart.de/eduroam} das Profil ''STUDENT'' auswählst und danach die .mobileconfig-Datei herunterlädst.\newline
Stelle nun sicher, das dein Laptop nicht über das ''normale'' Fenster zum W-Lan einrichten mit Eduroam verbunden ist. Falls dein Laptop/Smartphone schon über das ''normale'' Fenster zum W-Lan einrichten mit Eduroam verbunden ist, lösche diese Verbindung.\newline
Führe jetzt das heruntergeladene Programm aus und folge den Anweisungen, wähle als Institution die Universität Stuttgart und als Profil ''STUDENT''. Gib deine komplette st-Nummer (also stxxxxxx@stud.uni-stuttgart.de) und das Passwort zu deinem st-Account ein.\newline
\textit{Hinweis:} das Programm überprüft nicht, ob du dein Passwort richtig eingegeben hast.\newline
Nun solltest dein Laptop sich mit Eduroam verbinden.\newline


\section*{Für Linux}
Lade dir zunächst das Python Script herunter, indem du auf \href{\eduroamurllinux}{uni-stuttgart.de/eduroam} das Profil ''STUDENT'' auswählst und danach das Python Script herunterlädst.\newline
Führe nun das Script im Terminal mit: ''\$ python3 [Dateipfad zum Python Script]'' aus. Folge jetzt den Anweisungen die in dem sich öffnenden Fenster erscheinen. Gib bei der Benutzererkennung deine komplette st-Nummer (also stxxxxxx@stud.uni-stuttgart.de).\newline
Nun sollte dein Laptop sich mit Eduroam verbinden.