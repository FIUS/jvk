% !TeX root = ../jvk-blatt1.tex

\excercise{Einrichtung von eduroam}
\label{ex1}

In diesem Kapitel wollen wir dir zeigen, wie man eduroam einrichtet. eduroam ist das WLAN an der Uni und dementsprechend ist es für dein Studium und den Java Vorkurs sehr praktisch, wenn du es benutzen kannst.
 
\section*{Für Windows, Android und IOS}
\begin{enumerate}[label=\arabic*.]
    \item Gehe zuerst auf die Website \href{\eduroamurl}{geteduroam.app} und wähle den schwarzen Button mit deinem Betriebssystem aus. Danach wird ein Programm heruntergeladen, mit dem du eduroam einrichten kannst.
    \item Stelle nun sicher, dass dein Laptop/Smartphone nicht über das ''normale'' Fenster zum WLAN einrichten mit eduroam verbunden ist. Falls dein Laptop/Smartphone schon über das ''normale'' Fenster zum WLAN einrichten mit eduroam verbunden ist, lösche diese Verbindung.
    \item Führe jetzt das heruntergeladene Programm aus und folge den Anweisungen. Wähle als Institution die Universität Stuttgart und als Profil ''STUDENT''. Gib deine komplette st-Nummer (also stxxxxxx@stud.uni-stuttgart.de) und das Passwort zu deinem st-Account ein.
\end{enumerate}

\textit{Hinweis:} das Programm überprüft nicht, ob du dein Passwort richtig eingegeben hast.

Nun sollte sich dein Laptop/Smartphone mit dem eduroam verbinden.



\section*{Für macOS}
\begin{enumerate}[label=\arabic*.]
    \item Lade dir zunächst die .mobileconfig-Datei herunter, indem du auf \href{\eduroamurllinux}{uni-stuttgart.de/eduroam} das Profil ''STUDENT'' auswählst und danach die .mobileconfig-Datei herunterlädst.\newline
    \item Stelle nun sicher, das dein Laptop nicht über das ''normale'' Fenster zum WLAN einrichten mit eduroam verbunden ist. Falls dein Laptop/Smartphone schon über das ''normale'' Fenster zum WLAN einrichten mit eduroam verbunden ist, lösche diese Verbindung.\newline
    \item Führe jetzt das heruntergeladene Programm aus und folge den Anweisungen. Wähle als Institution die Universität Stuttgart und als Profil ''STUDENT''. Gib deine komplette st-Nummer (also stxxxxxx@stud.uni-stuttgart.de) und das Passwort zu deinem st-Account ein.\newline
\end{enumerate}
\textit{Hinweis:} Das Programm überprüft nicht, ob du dein Passwort richtig eingegeben hast.\newline
Nun sollte sich dein Laptop/Smartphone mit dem eduroam verbinden.

\section*{Für Linux}
\begin{enumerate}[label=\arabic*.]
    \item Lade dir zunächst das Pythonskript herunter, indem du auf \href{\eduroamurllinux}{uni-stuttgart.de/eduroam} das Profil ''STUDENT'' auswählst und danach das Pythonskript herunterlädst.
    \item Führe nun das Skript im Terminal mit: ''\$ python3 [Dateipfad zum Pythonskript]'' aus. 
    Folge jetzt den Anweisungen, die in dem sich öffnenden Fenster erscheinen. 
    Gib bei der Benutzererkennung deine komplette st-Nummer (also stxxxxxx@stud.uni-stuttgart.de).
\end{enumerate}
Nun sollte sich dein Laptop/Smartphone mit dem eduroam verbinden.
