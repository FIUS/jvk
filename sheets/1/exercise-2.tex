% !TeX root = ./jvk-blatt1.tex
\lstset{
	basicstyle=\small,
	morecomment=[l]{*}
}
\excercise{Nachschlagwerk: Syntax und Codestil}

Jetzt folgt ein Nachschlagwerk auf das ihr immer wieder zurückgehen könnt bei den folgenden Aufgaben.
Ihr könnt es bereits mal durchlesen, es ist aber nicht schlimm, wenn ihr noch nicht alles versteht.

\subsection*{Syntax}
\noindent
In diesem Teil zeigen wir euch die Syntax von Java.
Die Syntax beschreibt, wie der Code aufgebaut und geschrieben werden muss, sodass er von einem Computer verstanden wird.\\
Es kommen jetzt viele Definitionen, ihr müsst euch aber keine Sorgen machen, wenn ihr nicht alles direkt versteht.
Nehmt diesen Teil als Referenz zum Nachschauen, wenn ihr euch später bei etwas nicht sicher seid.

\begin{Infobox}

	\subsubsection*{Kommentare:}
	Kommentare sind Teile des Codes, die vom Computer ignoriert werden.
	Sie werden genutzt, um anderen (aber auch euch selbst) kenntlich zu machen, was im Code passiert, was noch verändert werden sollte oder um allgemein Anmerkungen zu hinterlassen.
\end{Infobox}

\begin{lstlisting}[title=\textbf{Kommentar Beispiel}]
	// Dies ist ein einfacher Kommentar über eine Zeile.
	// Er wird mit // gestartet und nimmt die gesamte Zeile hinter sich ein.
	someCode.run(); //Kommentare sind auch nach Operationen möglich.
	// Kann man das auch vor Operationen in der selben Zeile schreiben ?
	/*
	* Dies ist ein Block Kommentar.
	* Diese können sich auch über mehrere Zeilen erstrecken.
	* Um die Lesbarkeit zu verbessern, werden die Zeilen dazwischen meist
	* mit einem * begonnen.
	*/
\end{lstlisting}
\lstset{
	basicstyle=\small
}
\begin{Infobox}

	\subsubsection*{Klassen:}
	Klassen sind Baupläne für Objekte.
	In diesen können Operationen definiert werden, welche dann den Objekten zur Verfügung stehen.\\
	Eine Klasse beginnt immer mit dem \lstinline{class} Schlüsselwort, gefolgt von ihrem Namen.\\
	Diesen könnt ihr fast frei wählen, nur spezielle Schlüsselwörter wie zum Beispiel \lstinline{class} sind verboten.
	Anschließend folgt ein Paar geschweifter Klammern \{ \}, in denen der Inhalt der Klasse geschrieben wird.\\
	In Java muss der Dateiname gleich dem Klassennamen sein. Z.B. Klasse \q{Hund} $\rightarrow$ Datei \q{Hund.java}.
\end{Infobox}

\newpage

\begin{lstlisting}[title=\textbf{Klassen Syntax}]
	// < > wird immer durch das ersetzt was in < > beschrieben wird
	public class <NameDerKlasse>{ // Beginn der Klasse
		/*
		* Inhalt der Klasse.
		* Um die Lesbarkeit zu verbessern wird
		* Code innerhalb geschweifter Klammern eingerückt.
		*/
		// Definition einer Operation (Siehe nächsten Eintrag)
		public <Rückgabetyp> <OperationsName>(<Typ> <ParameterName>, ...){
			// Inhalt der Operation
		}
		...
	}
\end{lstlisting}

\begin{lstlisting}[title=\textbf{Klassen Beispiel}]
	public class Hund{
		public void wuff(){
			// Code um zu bellen
		}
	}
\end{lstlisting}
\begin{Infobox}
	\subsubsection*{Operationen}
	\begin{itemize}
		\item Operationen: Sie sind Gruppierungen von Anweisungen (da wir diese vielleicht häufiger ausführen wollen).
			Dieser Code kann mithilfe des Operationsnamen über die Objekte einer Klasse aufgerufen werden.\\
			Eine Operation besitz einen Namen. Sie können auch Parameter und Rückgabewerte haben. (sie besitzen auch noch eine Sichtbarkeit, wir verwenden im Vorkurs immer \lstinline{public} um es zu vereinfachen)
		\item Rückgabewert: Das ist quasi das Ergbenis der Operation, diesen wollen wir zurückgeben um ihn weiterzubenutzen.
			Sollte eine Operation nichts zurückgeben so hat sie den Rückgabewert \lstinline{void}.
		\item Parameter: Parameter sind Variablen (Platzhalter für Werte, wie ''x'' in Mathe), die es ermöglichen einen Wert zu übergeben. Diese Variable ist nur innerhalb der Operation im Code verfügbar.\\
			Sie werden in den runden Klammern nach dem Namen der Operation aufgelistet.
			Wenn man eine Operation aufruft und ein Objekt dort übergibt, wird dieses Objekt Argument genannt.
		\item Operationrumpf: Befindet sich zwischen den geschweiften Klammern. Hier ist der eigentliche Code der Operation.
	\end{itemize}

\end{Infobox}

\begin{lstlisting}[title=\textbf{Operations Syntax}]
	public <RückgabeTyp> <OperationsName>(){
		// Inhalt der Operation
	}
	public <RückgabeTyp> <OperationsName>(<Typ> <ParameterName>, ...){
		// Inhalt der Operation
	}
\end{lstlisting}

\newpage

\begin{lstlisting}[title=\textbf{Operations Beispiel}]
	public void springen(){
		// Code um zu springen
	}
	
	public String getName(Hund hund){
		return hund.name;
	}
\end{lstlisting}

\begin{Infobox}

	\subsubsection*{Objekte erzeugen:}
	Klassen sind Baupläne für Objekte. Quasi wenn die Klasse Hund ist, dann ist Bello und Doggo Instanzen (Objekte) der Klasse Hund.\\
	Um ein Objekt zu erzeugen wird das \lstinline{new} Schlüsselwort verwendet.
	Man schreibt dann \lstinline{new } gefolgt von dem Klassennamen und ein paar Klammern ().
\end{Infobox}
\begin{lstlisting}[title=\textbf{Objekt erstellen Syntax}]
	new <KlassenName>(<MöglicheParameter>); // Ein Objekt wird erzeugt
\end{lstlisting}

\begin{lstlisting}[title=\textbf{Objekt erstellen Beispiel}]
	new Hund("Bello"); // Ein Hund mit dem Namen Bello wird erzeugt
\end{lstlisting}
\begin{Infobox}
	\subsubsection*{Variablen:}
	Variablen werden genutzt, um (temporär) Werte zu speichern.
	Sie sind in den Fällen nützlich, in denen ihr einen bestimmten Wert mehrfach im selben Code nutzen wollt.
	Außerdem auch wenn ein Wert beim Starten des Programms noch nicht fest steht oder sich noch ändert.\\

	Stellt euch Variablen wie Schubladen vor. Ihr könnt Sachen in die Schubladen reinlegen um sie später wieder zu benutzen. Um die Sachen aber wieder finden zu können geben wir den Schubladen Namen.\\

	Variablen besitzen immer einen Typ (Was man in die Schublade legen darf) und einen Namen (Name der Schublade).
	Mit einem = kann einer Variable ein Wert zugewiesen werden (Etwas in die Schublade legen).\\
	Über den Namen der Variable kann dann auf dessen Wert (Inhalt der Schublade) zugegriffen werden.\\

	Wenn man ein Objekt erzeugt, weist man es oft direkt einer Variable zu, damit man es später im Code noch benutzen kann.
\end{Infobox}

\newpage

\begin{lstlisting}[title=\textbf{Variablen Syntax}]
	<KlassenName> <Name>; // Eine Variable wird erzeugt, ihr ist noch kein Wert zugewiesen.
	
	<KlassenName> <Name> = <Wert>; // Eine Variable wird erzeugt und ihr wird ein Wert zugewiesen.
	
	<Name> = <Wert>; // Einer bestehenden Variable wird ein neuer Wert zugewiesen.
	
	<KlassenName> <Name> = new <KlassenName>(); // Einer neuen Variable wird eine neue Instanz eines Objekts zugewiesen.
	
	<KlassenName> <Name> = <NameEinerAnderenVariable>; // Einer neuen Variable wird der Wert einer anderen Variable zugewiesen. Das geht natürlich nur wenn die Varaiblen das selbe speichern können (den selben Typ haben).
\end{lstlisting}
\begin{lstlisting}[title=\textbf{Variable Beispiel}]
	String hundeName = "Bello";
	
	Hund meinHund = new Hund(hundeName);
\end{lstlisting}
\begin{Infobox}
	\subsubsection*{Kommandos und Abfragen:}
	\begin{itemize}
		\item Kommando: Sie ändern eventuell was an dem Objekt z.B. bennen den Hund um
		\item Abfragen: Sie ändern nichts an dem Objekt, geben dafür aber was zurück z.B. den Hundename
	\end{itemize}
	Um Operationen eines Objekts aufzurufen, wird zuerst der Objektname geschrieben, gefolgt von einem Punkt und dem Operationsname.
	Am Ende stehen paar Rundeklammern \q{()}, in denen Parameter stehen können.\\
	{\color{red} Wichtig: } Operationen können nur auf Objekten und nicht auf Klassen aufgerufen werden.
	Es gibt hier einen Sonderfall, diesen werdet ihr in PSE kennenlernen.
\end{Infobox}
\begin{lstlisting}[title=\textbf{Kommando/Abfrage Syntax}]
	<ObjektName>.<OperationsName>();
	
	<ObjektName>.<OperationsName>(<Argument>);
\end{lstlisting}

\begin{lstlisting}[title=\textbf{Kommando/Abfrage Beispiel}]
	// meinHund.setName(); soll dafür sorgen dass der Hund von Bello zu Doggo umbenannt wird. Er ändert also etwas an dem Objekt.
	meinHund.setName("Doggo");
	
	// meinHund.getName(); gibt den Namen des Hundes zurück, ändert also nichts am Obejekt selber sondern gibt Eigenschaften davon zurück.
	meinHund.getName();
\end{lstlisting}

\subsection*{Aufgabe 2.2 Namen und Codestyle}
Wie ihr wahrscheinlich schon gemerkt habt, kann Code schnell überfordern und verwirren.\\
Da man dieses Problem in der Informatik oft antrifft, vor allem wenn man mit Anderen zusammen arbeitet, gibt es Konventionen (Regeln an die wir uns alle halten wollen) wie der Code geschrieben werden sollte um die Lesbarkeit und das Verständnis zu verbessern.

\vspace{5mm}

\subsubsection*{Konvention Namen:}

\begin{Infobox}
	Die Java-Syntax verbietet bereits, dass:

	\begin{itemize}
		\item Schlüsselwörter(\lstinline{void}, \lstinline{class}, ...) als Name genutzt werden.
		\item Namen Leerzeichen beinhalten.
		\item Namen mit Zahlen beginnen.
	\end{itemize}
\end{Infobox}
\vspace{5mm}
\begin{Infobox}
	Formulierungen auf die man achten sollte:
	\begin{itemize}
		\item Variablennamen sollten selbsterklärend sein $\rightarrow$ \lstinline{favouriteDog} statt \lstinline{fD}
		\item Operationen sollten Verben im Namen haben $\rightarrow$ ''makeNoise'' statt ''laut''
		\item Abfragen sollten mit \lstinline{get} beginnen $\rightarrow$ ''getName''
		\item Kommandos, die einen Wert setzen, sollten mit \lstinline{set} beginnen $\rightarrow$ ''setName''
		\item Keine Umlaute (ä, ö, ß,...)

	\end{itemize}
\end{Infobox}

\vspace{5mm}

\noindent
Allgemein sollten alle Namen auf Englisch formuliert werden.
Variablen und Operationen werden immer mit einem Kleinbuchstaben angefangen.\\

\noindent
Falls ein Name aus mehreren Wörtern besteht, schreibt man diese trotzdem zusammen, aber den Anfangsbuchstaben aller neuen Wörter groß. Z.B. favouriteDog.\\
Diese Schreibweise nennt man auch \lstinline{camelCase}.\\

\noindent
Namen sollen selbsterklärend sein, damit ist erkennbar was sie tun bzw. wofür sie stehen.\\

\noindent
Für Klassen gelten fast die selben Richtlinien, nur ist hier auch das erst Wort groß.
Das nennt man dann \lstinline{PascalCase} oder \lstinline{UpperCamelCase}. Z.B. DogFood.

\vspace{5mm}

\begin{lstlisting}[title=\textbf{Beispiel gute Namensgebung}]
	class DogOwner {

		void talkAboutTheDog(Dog dog){
			String ownerName = "FIUS";
			// Talk about the Dog until stopped
		}
	}
\end{lstlisting}

\subsubsection*{Einrückungen:}
Um direkt auf einen Blick zu erkennen wo die Operation anfängt und wo sie endet, rücken wir den Code in einer Operation (übrigens auch in Klassen) um eine Stufe ein (Eine \fbox{TAB} Stufe).\\

\noindent
{\color{red} Wichtig: }Achtet unbedingt daruaf das alle Klammer die ihr öffnet auch wieder in der richtigen Reihenfolge geschlossen werden!

\newpage

\begin{lstlisting}[title=\textbf{Beispiel gute Namensgebung}]
	// Code außerhalb einer Klammerung, nicht eingerückt
	public class RandomClass{
		public void operation1(){
			// Operationsinhalt 2-mal eingerückt
		}
		// Klasseninhalt 1-mal eingerückt
		public void operation2(){
			// Operationsinhalt 2-mal eingerückt
		}
	}
\end{lstlisting}

\begin{lstlisting}[title=\textbf{Beispiel schlechte Einrückung}]
	// Code außerhalb einer Klammerung, nicht eingerückt
	public class RandomClass{
	public void operation1(){
// Operationsinhalt
	}
			public void operation2(){
		// Operationsinhalt
			}
}
\end{lstlisting}
