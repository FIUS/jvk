\excercise{Die Matrix}

\Todo{move to later... This needs to be adapted to new api completely!}
% Copy methods from ManualStartSimulation (ICGE2 repo)
% Needs working API for playfield manipulation
% Goal: Paint different geometric forms on the Playfield [learning goal: use more objects and methods]

Starte den \simulator.
Dafür musst du zunächst die Datei \texttt{Main.java} finden und öffnen.
Die Datei ist links im Package Explorer unter \texttt{src/main/java} $\to$ \texttt{de.unistuttgart.informatik.fius.jvk2019} zu finden.
Danach kann man das Programm mit einem Klick auf den Pfeil nach rechts in einem grünen Kreis in der Leiste oben starten.


    \subexcercise Beende das Programm ohne den "`\textit{Fenster schließen}"' Knopf zu drücken.
        Dafür kann man in Eclipse unten in der Konsole das rote Quadrat anklicken.
        Achtung: Es könnten schon mehrere Konsolen offen sein.
        Falls sich das Programm nicht beendet muss man die aktuelle Konsole erst mit dem \fbox{X} neben dem roten Quadrat schließen.
        Falls die Konsole nicht offen ist kann man sie im Menü \fbox{window} $\to$ \fbox{show view} $\to$ \fbox{console} öffnen.
    \subexcercise Starte das Programm im debug Modus.
        Der Knopf dafür sieht wie ein Käfer aus.
    \subexcercise Wähle im \simulator{} die Aufgabe \fbox{Task0 a)} aus.
        Starte die Simulation indem du auf den \fbox{Play} Knopf drückst.
        Probiere die anderen Knöpfe aus.
        Der \fbox{Stopp} Knopf setzt die Simulation auf start zurück.
    \subexcercise Lösche die letzte Mauer auf Neo's Weg.
        Was passiert?
    \subexcercise Stelle Neo eine Mauer in den Weg.
        Was passiert?

