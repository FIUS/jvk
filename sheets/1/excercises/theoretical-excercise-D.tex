% !TeX root = ../jvk-blatt1.tex
Du hast eine e-mail von deinem Team-leiter bekommen:
Hi,

Der Cheff hat mir eine E-mail von einem Kunden weitergeleitet und gemeint wir sollen das Projekt übernehmen. Annika und Jürgen sind ja noch im Urlaub und Sven setzt schonmal ein projekt auf. Kannst du dir das Blabla vom Kunden schon mal anschauen und die Klassen und Attribute rausschreiben!

Danke und grüße
David

--ForwardedMessage.eml\hline
\textbf{Subject:} Lagerhallen Rechner Sanftwahre 
\textbf{From:} Karl-Heinz-Mueller@storage.logistik.nil.eu.de
\textbf{Date:} 5.10 9:37
\textbf{To:} auftrag@comsulting.de
Sehr geehrt Damen und Herren,

ich bin leiter eine Lagerhalle bei Neckers-Ulm. Das ist die Zentralle Lagerhalle von Nil.de, für Ba-Wü. Kennen sie vielleicht ist direkt beim Ortseingang.

Wie auch immer, bei uns herrscht immer ein Chaos und ich möchte endlich auf meinen Computer sehen können, was da vor sich geht. Dann muss ich nicht mehr rumlaufen, um überall meine Augen zu haben.
Wir haben uns letztes Jahr so ein Funkkarten System gekauft. Da hat jeder hier so eine EC-karte und kann die an einen Kasten bei der Tür halten um sie auf zu schließen. Leider ohne die Computer dinge, die sollen ja auch von ihnen kommen. Also jedesmal wenn ein mitarbeiter seine Karte dranhalten. Soll der beim Hauptrechner sagen wer da gerade welche tür aufmachen möchte und der Antwortert dann geht. Natürlich nur wenn ich ihm vorhergesagt habe es geht, oder mein Sicherheitsbeauftragter. Ach ja ich möchte auch das man vorher irgendwie ein Passwort eingibt. Ich hasse die dinger zwar wie die Pest aber, dann kann das nicht jeder machen. 
Und die Mitarbeiter sollen sich auch anmelden können und sehen wo sie rein dürfen und wie Lange sie gearbeitet haben. Also wie lange zwischen reingehen durch die eingangstür und rausgehen vergangen ist. Ich kann das natürlich auch sehen.\\
Desweiter scannen natürlich die mitarbeiter mit so Handgeräten alle Waren ein wenn sie reinkommen und wenn sie rausgehen.

PS: Mein Anwalt hat gemeint, niemand darf die Arbeitszeit verändern dürfen, sondern nur Anfragen.

Mit Freundlichen Grüßen.
Karl-Hein M?:uller