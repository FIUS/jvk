% !TeX root = ./jvk-blatt1.tex

    \excercise{Objekt Orientierte Analyse}

        \Todo{Aufgabe ist viel zu kompliziert. Der text ist in der PSE klausur vielleicht ok, aber für den Vorkurs zu schwer}

        \subexcercise Du willst einen neuen Laptop kaufen. also informierst du dich online und im Elektrofachgeschäft. Nachdem du dir verschiedene Geräte angeschaut hast bist du der Meinung das ein Thinkbook++ genau richtig ist. Also gehst du in JupiterMarkt und sprichst mit einem Verkäufer. Jedoch ist das Gerät das du willst leider ausverkauft. Weil du nicht warten möchtest bestellst du das Gerät lieber in einem Onlineshop. Du entscheidest dich für Nil.com

        \begin{enumerate}
         \item[a)] Identifiziere (und schreibe auf) Klassen und Objekte die du in diesem Text finden kannst.
         \item[b)] Ist es sinnvoller JupiterMarkt und Nil.com durch die selbe Klasse zu representieren oder nicht? und warum
         \item[c)] Beschreibe 10 potenzielle Attribute für die Klasse Computer
        \end{enumerate}
        \subexcercise Du hast eine e-mail von deinem Team-leiter bekommen:\\

        Hi,
        Der Cheff hat mir eine E-mail von einem Kunden weitergeleitet und gemeint wir sollen das Projekt übernehmen. Annika und Jürgen sind ja noch im Urlaub und Sven setzt schonmal ein projekt auf. Kannst du dir das Blabla vom Kunden schon mal anschauen und die Klassen und Attribute rausschreiben!

        Danke und grüße
        David

        --ForwardedMessage.eml\\
        \textbf{Subject:} Lagerhallen Rechner Sanftwahre\\
        \textbf{From:} Karl-Heinz-Mueller@storage.logistik.nil.eu.de\\
        \textbf{Date:} 5.10 9:37\\
        \textbf{To:} auftrag@comsulting.de\\

        Sehr geehrt Damen und Herren,\medskip

        ich bin leiter eine Lagerhalle bei Neckars-Ulm. Das ist die Zentrale Lagerhalle von Nil.de, für Ba-Wü. Kennen sie vielleicht ist direkt beim Ortseingang.

        Wie auch immer, bei uns herrscht immer ein Chaos und ich möchte endlich auf meinen Computer sehen können, was da vor sich geht. Dann muss ich nicht mehr rumlaufen, um überall meine Augen zu haben.
        Wir haben uns letztes Jahr so ein Funkkarten System gekauft. Da hat jeder hier so eine EC-karte und kann die an einen Kasten bei der Tür halten um sie auf zu schließen. Leider ohne die Computer dinge, die sollen ja auch von ihnen kommen. Also jedesmal wenn ein mitarbeiter seine Karte dranhalten. Soll der beim Hauptrechner sagen wer da gerade welche tür aufmachen möchte und der Antwortet dann geht. Natürlich nur wenn ich ihm vorhergesagt habe es geht, oder mein Sicherheitsbeauftragter. Ach ja ich möchte auch das man vorher irgendwie ein Passwort eingibt. Ich hasse die dinger zwar wie die Pest aber, dann kann das nicht jeder machen.
        Und die Mitarbeiter sollen sich auch anmelden können und sehen wo sie rein dürfen und wie Lange sie gearbeitet haben. Also wie lange zwischen reingehen durch die eingangstür und rausgehen vergangen ist. Ich kann das natürlich auch sehen.\\
        Desweiter scannen natürlich die mitarbeiter mit so Handgeräten alle Waren ein wenn sie reinkommen und wenn sie rausgehen.
        \medskip

        PS: Mein Anwalt hat gemeint, niemand darf die Arbeitszeit verändern dürfen, sondern nur Anfragen.
        \medskip
        Mit Freundlichen Grüßen.\\Karl-Heinz\\ M?:uller

        \subexcercise
        Markiere im folgenden Text alle Klassen und Objekte.

        Es ist 15 Uhr und du freust dich auf deine Lieblingsshow, welche demnächst im TV auf Sender 7 läuft.
        Du läufst in die Küche, um dir Softdrinks und Snacks zu holen. Schnell merkst du jedoch, dass du beides nicht mehr vorrätig hast, und entscheidest dich, mit dem Auto zum Laden zu fahren und einzukaufen.

        Du schnappst dir deine Schlüssel und deinen Geldbeutel und läufst zu deinem Auto, einem blauen Golf 5, welches auf der Straße "Jupiterweg" steht.
        Am Lidl angekommen schnappst du dir schnell eine Tüte funny-frisch Chips und zwei Flaschen Coca-Cola und rennst fast schon zu den Kassen, damit du deine Lieblingsshow nicht verpasst. Dort merkst du, dass du kein Geld dabei hast, weshalb du mit deiner Kreditkarte bezahlst. Voller Vorfreude fährst du nun nach Hause.
