% !TeX root = ./jvk-blatt1.tex

\subsection*{\textit{Dog}-Klasse in Java}

Java ist eine sogenannte \textit{objektorientierte} Programmiersprache.
Das heißt, dass wir in Java hauptsächlich mit Objekten interagieren.
Wenn wir das also einen Hund in Java erstellen wollen, müssen wir zunächst die zugehörige Klasse erstellen.

\begin{figure}[H]
    \caption{Schemenhafte Java-Implementierung der \textit{Dog}-Klasse}
    \label{t1:dog-class-scheme}
    \begin{lstlisting}
class Dog {
    String name;
    String ownerName;
    Integer age;

    // ...

    void bark() {
        // Code to make bark
    }
    void fetchStick() {
        // Code to fetch a stick
    }
}
    \end{lstlisting}
\end{figure}

In \ref{t1:dog-class-scheme} finden sich die Attribute (Z. 2-4) und Methoden (Z.45,48) wieder, die wir uns oben ausgedacht haben.
Vor den Namen der Attribute muss man Datentypen angeben, in welchen diese gespeichert werden sollen.
Da Java Objektorientiert ist, sind diese Datentypen andere Klassen mit ihren eigenen Attributen und Funktionen.
Bei \texttt{String} handelt es sich dabei um eine Zeichenkette, bei \texttt{Integer} um eine Ganzzahl.

Vor den Methodennamen , die wir uns ausgedacht haben, (Z.45,48) befindet sich jeweils das \textit{Schlüsselwort} \texttt{void}, das eine Methode ohne \textit{Rückgabewert} anzeigt.
Was genau das bedeutet, wird später noch erklärt.

Momentan ist die Klasse \texttt{Dog} aus \ref{t1:dog-class-scheme} noch nicht vollständig, da die beiden Methoden noch keinen Code beinhalten.
Daher soll dieser nun ergänzt werden, damit wir die ersten \texttt{Dog}-Objekte erstellen können.

\begin{figure}[H]
    \caption{Vollständige Implementierung der \textit{Dog}-Klasse}
    \label{t1:dog-class}
    \begin{lstlisting}
class Dog {
    String name;
    String ownerName;
    Integer age;

    // Constructor: Necessary to create Dog-Objects
    Dog(String name, String ownerName, Integer age) {
        this.name = name;
        this.ownerName = ownerName;
        this.age = age;
    }

    void bark() {
        System.out.println("Woff!");
    }
    void fetchStick() {
        System.out.println("Running");
        System.out.println("Picking stick up");
        System.out.println("Returning stick");
    }
}
    \end{lstlisting}
\end{figure}

In \autoref{t1:dog-class} wurde den Methoden \texttt{bark} und \texttt{fetchStick} Inhalt gegeben.
Dafür wurde die Funktion \texttt{System.out.println} verwendet, die Zeichenketten auf der \textit{Konsole} ausgibt.
Zeichenketten werden in Java in \texttt{"Anführungszeichen"} angegeben und wurden in den beiden Methoden jeweils schon an die \texttt{System.out.println} Funktion übergeben, indem man sie in die runden Klammern \texttt{( )} hinter der Funktion geschrieben hat.
So würde beim Aufrufen der Methode \texttt{bark} auf einem \texttt{Dog}-Objekt der Text \texttt{Woff!} auf der Konsole erscheinen.

In Z.6-10 wurde der Klasse \texttt{Dog} außerdem ein sogenannter \textit{Konstruktor} hinzugefügt.
Dieser muss mit dem Namen der Klasse beginnen und sorgt dafür, dass wir später \texttt{Dog}-Objekte erstellen können.
Der Konstruktor erhält drei Werte als \textit{Argumente}, die er dann den Attributen der Klasse zugewiesen werden.
Diese drei Argumente muss man beim Aufrufen des Konstruktors der \texttt{Dog}-Klasse angeben, damit man einen \texttt{Dog}-Objekt erstellen kann.

\begin{figure}[H]
    \caption{Instanziierung der \texttt{Dog}-Klasse}
    \label{t1:dog-class-instantiation}
    \begin{lstlisting}
class Main {
    public static void main(String[] args) {
        Dog joey = new String("Jeoy", "Justin", 14);
    }
}
    \end{lstlisting}
\end{figure}

Durch das Aufrufen des Konstruktors, was in Java immer durch das \texttt{Schlüsselwort} \texttt{new} gekennzeichnet wird, wurde in \ref{t1:dog-class-instantiation} in Z.3 so ein \texttt{Dog}-Objekt erstellt.
Damit wir das neu erstellte Objekt auch verwalten können, wird es in einer Variable namens \texttt{joey} gespeichert.
Das Speichern in der Variable nennt man \texttt{Zuweisung} und ist gut, um auf dem erstellten Objekt Methoden aufzurufen, die Attribute zu ändern (bspw. wenn \texttt{joey} Geburtstag hat) oder das Objekt als Argument an andere Methoden oder Funktionen zu übergeben.
Zum Speichern wird der Zuweisungsoperator \texttt{=} verwendet, der auch schon im Konstruktor von \texttt{Dog} \ref{t1:dog-class} zum Einsatz kam um die Argumente abzuspeichern.

% TODO Aufgabenidee: Wir stellen eine Katzen-Klasse (mit einem extra Attribut) und Main-Klasse zur Verfügung und sie sollen eine Katze instanziieren

% TODO Nächste kurze Erklärung hieran

% \begin{figure}[H]
%     \caption{\textit{Dog}-Klasse mit \texttt{barkName}-Methode}
%     \label{t1:dog-class-mod1}
%     \begin{lstlisting}
% class Dog {
%     // ...
%     void fetchStick() {
%         // ...
%     }
%     void barkName() {
%         System.out.println("Woff! " + this.name + "! Woff!"); // TODO This is silly...
%     }
% }
%     \end{lstlisting}
% \end{figure}

% TODO Nächste Erklärung hieran (das könnte schon nach Blatt 2 Kapitel 1 gehören)

% \begin{figure}[H]
%     \caption{\textit{Dog}-Klasse mit \texttt{barkName}-Methode}
%     \label{t1:dog-class-mod1}
%     \begin{lstlisting}
% class Dog {
%     // ...
%     void markName() {
%         // ...
%     }
%     void greet(Dog otherDog) {
%         System.out.println("Woff " + otherDog.name + "!");
%     }
% }
%     \end{lstlisting}
% \end{figure}

% TODO Nächste Aufgabenidee: \texttt{greet}-Methode auf Klasse mit Implementierung aufrufen

% TODO Erklärung für Variablen von üb1 2023 verkürzt eingliedern

% Variablen:
% Variablen werden genutzt, um (tempor\"ar) Werte zu speichern. Sie sind in den F\"allen n\"utzlich,
% in denen du einen bestimmten Wert mehrfach im selben Code nutzen willst. Au\ss erdem auch
% wenn ein Wert beim Starten des Programms noch nicht fest steht oder sich noch \"andert.
% Stell dir Variablen wie Schubladen vor. Du kannst Sachen in die Schubladen reinlegen um
% sie sp\"ater wieder zu benutzen. Um die Sachen aber wieder finden zu k\"onnen geben wir den
% Schubladen Namen.
% Variablen besitzen immer einen Typ (was man in die Schublade legen darf / passt) und einen
% Namen (Name der Schublade). Mit einem = kann einer Variable ein Wert zugewiesen werden
% (Etwas in die Schublade legen).
% \"Uber den Namen der Variable kann dann auf dessen Wert (Inhalt der Schublade) zugegriffen
% werden.
% Einige wichtige Typen sind:
% • String: zum speichern von Text, z.B. ''Das ist ein String''
% • boolean: true oder false (wahr oder falsch), wird sp\"ater noch genauer erkl\"art
% • Integer: positive und negative ganze Zahlen
% • Double: positive und negative Zahlen, welche Nachkommastellen haben k\"onnen
% Wenn man ein Objekt erzeugt, weist man es oft direkt einer Variable zu, damit man es sp\"ater
% im Code noch benutzen kann.