% !TeX root = ./jvk-blatt1.tex

%\theory{Klassen und Objekte} # TODO Macht so etwas Sinn ???
\section*{Klassen und Objekte}

Bei einer Klasse handelt es sich um einen Bauplan oder eine Vorlage für eine bestimmte Art von Objekten.
Sie bestimmen, welche Eigenschaften diese Objekte der Klasse haben und und welche Funktionen sie ausführen können.

Zum Beispiel könnte eine Klasse \textit{Dog} die Eigenschaften \textit{name}, \textit{age} und \textit{ownerName} haben.
Eine Klasse beschreibt die generellen Eigenschaften ihre Objekte.
In unserem Beispiel muss jedes Objekt der \textit{Dog}-Klasse - also jeder konkrete Hund - diese drei Eigenschaften besitzen.

Die Funktionen eines Hundes könnten hingegen beispielsweise sein, dass er bellt, oder dass er ein Stöckchen holen geht.
In Klassen gedacht, könnte man also der Klasse \textit{Dog} die Funktionen \textit{bark} und \textit{fetchStick} hinzufügen, sodass jeder Hund diese Methoden erfüllt.
Diese Methoden sollen sich nur von Objekten der Klasse ausführen lassen.

Wie in jeder Disziplin gibt es beim Programmieren mit Objekten Fachbegriffe, damit es nicht zu Verwechslungen kommt.
Die Eigenschaften von Klassen nennt man beim Programmieren \textit{Attribute} oder \textit{Objektattribute}, die Funktionen der Objekte einer Klasse \textit{Methoden}.

\begin{table}
    \caption{Intuitive Begriffe - Java Fachbegriffe}
    \begin{tabular}{l|c|c}
        Intuitive Begriffe & Fachbegriffe\\\hline
        Eigenschaft eines Objekts & Attribut\\
        Funktion eines Objekts & Methode\\
    \end{tabular}
\end{table}

\excercise{Klassen und Objekte identifizieren}

Identifiziere drei Klassen aus deinem Alltag und überlege dir, welche Methoden und Attribute sie haben könnten.

\begin{figure}[H]
    \centering
    \begin{tikzpicture}[semithick]
        \draw (0,0)     rectangle (4,-0.8);
        \draw (0,-0.8)  rectangle (4,-3);
        \draw (0,-3)    rectangle (4,-5);

        \draw (5,0)     rectangle (9,-0.8);
        \draw (5,-0.8)  rectangle (9,-3);
        \draw (5,-3)    rectangle (9,-5);

        \draw (10,0)     rectangle (14,-0.8);
        \draw (10,-0.8)  rectangle (14,-3);
        \draw (10,-3)    rectangle (14,-5);
    \end{tikzpicture}
\end{figure}

Beispiel: % TODO Mit den Attributen und Methoden des Hund-Beispiels annotieren

\begin{figure}[H]
    \centering
    \begin{tikzpicture}[semithick]
        \draw (0,0)     rectangle (4,-0.8);
        \draw (0,-0.8)  rectangle (4,-3);
        \draw (0,-3)    rectangle (4,-5);
    \end{tikzpicture}
\end{figure}