% !TeX root = ./jvk-blatt1.tex
\newcommand{\jvkpackage}{MatrixJVK.zip }
\excercise{Programmstart}

 Öffne die Entwicklungsumgebung Eclipse und öffne das Maven-Projekt \jvkpackage und importiere es.
 Unter Umständen muss das \jvkpackage vorher noch entpackt werden.
 \begin{enumerate}
     \item Um ein Projekt zu importieren klicke zuerst auf \fbox{File} $\rightarrow$      \fbox{Import...}.
     \item Wähle in dem Ordner \fbox{Maven} $\to$ \fbox{Existing Maven Projects}.
     \item Drücke oben rechts auf \fbox{Browse...} und suche das Verzeichnis, wo die Datei \jvkpackage entpackt wurde.
     \item Zu guter Letzt noch auf \fbox{Finish} drücken.
     \item Nun siehst du das Projekt im \textit{Package Explorer}.
     \item Um euer Projekt zu vervollständigen solltet ihr im \textit{Package Explorer} das Projekt mit einem Rechtsklick auswählen und dann in diesem Kontextmenü \fbox{Maven} $\rightarrow$ \fbox{Update Project...} $\rightarrow$ \fbox{OK} ausführen.
 \end{enumerate}
