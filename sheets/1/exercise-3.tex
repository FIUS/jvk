
\newcommand{\solutionPackage}{\texttt{de.unistuttgart.informatik.fius.jvk2019.solutions}}
\excercise{Operationen selber aufrufen}

\Todo{rewrite für neue api}

Öffne die Klasse \texttt{Solution1} im Paket \solutionPackage.
    Eine Operation die Neo ausführen soll steht schon in der \texttt{solve} Operation der \texttt{Solution1} Klasse.
    Mit \texttt{this} kann man auf Daten des aktuellen Objekts zugreifen.
    In der \texttt{Solution1} Klasse bezieht sich this auf das aktuelle Objekt der \texttt{Solution1} Klasse.
    In Zeile 27 wird dem Attribut \texttt{player} das \texttt{Neo} Objekt zugewiesen.
    In Zeile 36 wird dann auf diesem \texttt{Neo} Objekt das Kommando \texttt{move()} aufgerufen.


        \subexcercise Bewege Neo auf die Telefonzelle indem du die fehlenden Kommandos in der \texttt{solve} Operation der \texttt{Solution1} Klasse ergänzt.
        \subexcercise Aus welcher Klasse in dem Klassendiagramm weiter oben kommt das Kommando \texttt{move()}?

