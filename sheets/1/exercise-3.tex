\excercise{Einschub: Objekte, Klassen, Klassendiagramm}

\Todo{Use this as task 2}
\Todo{Add how to Task for methods see task about methods}
\Todo{Add explanation for java new Class() syntax to create objects, maybe with task to write pseudo java code}
% Learning goal: Differentiate class/object, Map class object to java syntax from exercise 1,

In Java (und auch in anderen Programmiersprachen) gibt es Klassen und Objekte.
 Dabei gehört ein Objekt immer zu einer Klasse.
 Die Klasse beschreibt dabei generell welche Eigenschaften und Fähigkeiten ein Objekt haben kann.
 Eigenschaften können zum Beispiel Daten (wie der Name oder die Farbe des Objekts) sein.
 Fähigkeiten wiederum sind Operationen die das Objekt ausführen kann.
 Das Objekt, also eine konkrete Instanz der Klasse, hat dann zum Beispiel eine bestimmte Farbe (grün).
 Ein Anderes Objekt der selben Klasse kann eine eigene Farbe (rot) haben.

 Nimm eine Klasse (z.B. Auto, Telefon, Fernseher, ...) und überlege dir welche Eigenschaften (Daten) und Fähigkeiten (Operationen) diese Klasse hat.
 Überlege dir für drei Objekte deiner gewählten Klasse konkrete Werte.

 Entscheide welche Operationen das Objekt verändern, also Commands sind, und welche einen Wert zurückgeben ohne das Objekt zu verändern, also Queries sind.

 \begin{tikzpicture}[]
     \draw (0,0)     rectangle (4,-0.8);
     \draw (0,-0.8)  rectangle (4,-3);
     \draw (0,-3)    rectangle (4,-5);
     \draw (5,0)     rectangle (9,-0.8);
     \draw (5,-0.8)  rectangle (9,-3);
     \draw (10,0)    rectangle (14,-0.8);
     \draw (10,-0.8) rectangle (14,-3);
     \draw (5,-3.5)  rectangle (9,-4.3);
     \draw (5,-4.3)  rectangle (9,-6.5);
     \draw (10,-3.5) rectangle (14,-4.3);
     \draw (10,-4.3) rectangle (14,-6.5);
 \end{tikzpicture}
