
\excercise{DemoTask und UI}

\begin{enumerate}
    \item Nun wollen wir wie in Aufgabe 1 ein Spiel starten. 
        Dieses Mal nur mit dem \lstinline{DemoTask} und noch ohne den \lstinline{DemoTaskVerifier}:

    \begin{lstlisting}
Game myGame = new Game("Hello World", new DemoTask());
    \end{lstlisting}

    Du startest das Spiel in der Variable \lstinline{myGame} indem du die Operation \lstinline{run()} darauf aufrufst.

    \begin{lstlisting}
myGame.run();
    \end{lstlisting}

    Starte das Programm in Eclipse, mit dem kleinen Play-Button oben in der Werkzeugleiste und schau dich ein wenig im Fenster, das dann aufgeht um.

    \vspace{5mm}

    \item In der vorherigen Teilaufgabe hattest du bereits ''Hello World'' und ein DemoTask Objekt an Game übergeben.
        Nun wollen wir noch ein DemoTaskVerifier Objekt übergeben.\\
        Orientiere dich dazu unten am Bild. Starte nachdem du fertig bist das Spielfenster neu. \\

        Was verändert sich im Spielfenster? 
        Finde den \q{Task Status} Tab und drücke den Refresh Button.

    \begin{lstlisting}
Game myGame = new Game("Hello World", new DemoTask(), new DemoTaskVerifier());
    \end{lstlisting}

\end{enumerate}


\begin{Infobox}[Der Refresh Button]
    Wenn du überprüfen willst, ob du deine Aufgabe erledigt hast, musst du den \fbox{Task Status} Tab unter dem Spielfeld öffnen und dann auf \fbox{Refresh} klicken.\\

    Wichtig: Der Task Status aktualisiert sich nicht automatisch. Du musst also immer selber aktualisieren.
\end{Infobox}


\begin{enumerate}\setcounter{enumi}{2}

    \item Finde sowohl im Spiel als auch in Eclipse die Konsole. Dies ist ein Feld in dem Text ausgegeben wird:
    \begin{center}
        \includegraphics[width=\linewidth]{./figures/console.PNG}
    \end{center}

    In der Konsole siehst du, dass eine Münze (\texttt{Coin}) gespawnt (erzeugt) wurde.
    Finde die Koordinaten des Feldes, auf welchem die Münze gespawnt wurde.

    \item Nun suche nach der Stelle im Code in der Klasse \lstinline{DemoTask}, in dem die erste Münze erzeugt wird.\\

        Kleiner Tipp: 
        Wenn du \fbox{Strg} drückst während du auf einen Klassennamen oder einen Operationsnamen im Code klickst, öffnet Eclipse die entsprechende Java-Datei wo sich diese Operation oder Klasse befindet.\\
        Alternativ kannst du über den PackageExplorer in das Paket\\
        \texttt{de.unistuttagrt.informatik.fius.jvk.tasks} navigieren und dort die Datei \texttt{DemoTask.java} mit einem Doppelklick öffnen.
\end{enumerate}
