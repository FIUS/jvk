% !TeX root = ./jvk-blatt1.tex
\excercise{Syntax}

\Todo{Erklärung zu Methoden (ähnlich zu aufgabe 2)}
\Todo{Syntax von Methodenaufrufen wie er für Aufgabe 1 notwendig ist}

\Todo{rewrite für neue api}

\Todo{Die (aktuelle) Aufgabe muss später kommen. Die Studis müssen ja erstmal das Framework lernen.}

% Task Idee für methoden aufrufen:
% a) game.getGameWindow().setTitle("New Title!");
% b) game.getSimulation().runTask(new DemoTask());
% c) DemoTask händisch ausführen in main (also neue Variable vom typ Task und dann task.run(game.getSimulation()) in main aufrufen)
% d) (optional) Gehe in DemoTask und Spawne noch mehr coins (einfach copy paste + positionen ändern)
% Learning goal: Students can call methods by copying from an example (same for for constructors and variable declarations)


Öffne die Klasse \texttt{Solution1} im Paket \solutionPackage.
    Eine Operation die Neo ausführen soll steht schon in der \texttt{solve} Operation der \texttt{Solution1} Klasse.
    Mit \texttt{this} kann man auf Daten des aktuellen Objekts zugreifen.
    In der \texttt{Solution1} Klasse bezieht sich this auf das aktuelle Objekt der \texttt{Solution1} Klasse.
    In Zeile 27 wird dem Attribut \texttt{player} das \texttt{Neo} Objekt zugewiesen.
    In Zeile 36 wird dann auf diesem \texttt{Neo} Objekt das Kommando \texttt{move()} aufgerufen.


        \subexcercise Bewege Neo auf die Telefonzelle indem du die fehlenden Kommandos in der \texttt{solve} Operation der \texttt{Solution1} Klasse ergänzt.
        \subexcercise Aus welcher Klasse in dem Klassendiagramm weiter oben kommt das Kommando \texttt{move()}?

