% !TeX root = ./jvk-blatt1.tex
\lstset{
	basicstyle=\small,
	morecomment=[l]{*}
}
\excercise{Syntax und Codestil}
\begin{enumerate}
	\item \textbf{Schreibt} zu jedem Teil selbst zwei Beispiele in einem Editor.
	\item \teamsaufgabe{Falls ein Link bereits aufgebraucht ist, verwendet einen anderen (es ist 3 mal das selbe Quiz)}

	Macht das Quiz erst, nachdem ihr Aufgabe a) gemacht habt.

	\url{https://kahoot.it/challenge/02297621?challenge-id=9fbcfde6-f12d-4f90-a71a-78962e44d1e0_1633208690696}
	
	\url{https://kahoot.it/challenge/06631442?challenge-id=9fbcfde6-f12d-4f90-a71a-78962e44d1e0_1633208706556}
	
	\url{https://kahoot.it/challenge/03438017?challenge-id=9fbcfde6-f12d-4f90-a71a-78962e44d1e0_1633208720874}
\end{enumerate}
\subsection*{Syntax}
\noindent
In diesem Teil zeigen wir euch die Syntax von Java.
Die Syntax beschreibt, wie der Code geschrieben werden muss, sodass er von einem Computer verstanden wird.\\
Es kommen jetzt viele Definitionen, ihr müsst euch aber keine Sorgen machen, wenn ihr nicht alles direkt versteht.
Nehmt diesen Teil als Referenz zum Nachschauen, wenn ihr euch später bei etwas nicht sicher seid.

\begin{Infobox}
	
	\subsubsection*{Kommentare:}
	Kommentare sind Teile des Codes, die vom Computer ignoriert werden.
	Sie werden genutzt, um anderen (aber auch euch selbst) kenntlich zu machen, was im Code passiert, was noch verändert werden sollte oder um allgemein Anmerkungen zu hinterlassen.
\end{Infobox}

\begin{lstlisting}[title=\textbf{Kommentar Beispiel}]
	// Dies ist ein einfacher Kommentar über eine Zeile.
	// Er wird mit // gestartet und nimmt die gesamte Zeile hinter sich ein.
	someCode.run(); //Kommentare sind auch nach Operationen möglich.
	/*
	* Dies ist ein Block Kommentar. 
	* Diese können sich auch über mehrere Zeilen erstrecken.
	* Um die Lesbarkeit zu verbessern, werden die Zeilen dazwischen meist 
	* mit einem * begonnen.
	*/
\end{lstlisting}
\lstset{
	basicstyle=\small
}
\begin{Infobox}
	
	\subsubsection*{Klassen:}
	Klassen sind Baupläne für Objekte.
	In diesen können Operationen definiert werden, welche dann den Objekten zur Verfügung stehen.\\
	Eine Klasse beginnt immer mit dem \lstinline{class} Schlüsselwort, gefolgt von ihrem Namen.
	Diesen könnt ihr fast frei wählen, nur spezielle Schlüsselwörter wie zum Beispiel \lstinline{class} sind verboten.
	Anschließend folgt ein Paar geschweifter Klammern \{ \}, in denen der Inhalt der Klasse geschrieben wird.
	In Java muss der Dateiname gleich dem Klassennamen sein. Z.B. Klasse \q{Hund} $\rightarrow$ Datei \q{Hund.java}.
\end{Infobox}
\begin{lstlisting}[title=\textbf{Klassen Syntax}]
	// < > wird immer durch das ersetzt was in < > beschrieben wird
	public class <NameDerKlasse>{ // Beginn der Klasse
		/*  
		* Inhalt der Klasse. Um die Lesbarkeit zu verbessern wird
		* Code innerhalb geschweifter Klammern eingerückt.
		*/
		// Definition einer Operation (Siehe nächsten Eintrag)
		public <Rückgabetyp> <OperationsName>(<Typ> <ParameterName>, ...){
			// Inhalt der Operation
		}
		...
	}
\end{lstlisting}

\newpage

\begin{lstlisting}[title=\textbf{Klassen Beispiel}]
	public class Hund{
		public void wuff(){
			// Code um zu bellen
		}
	}
\end{lstlisting}
\begin{Infobox}
	\subsubsection*{Operationen}
	Eine Operation besitzt einen Namen, einen Rückgabewert und kann Parameter besitzen (sie besitzen auch noch eine Sichtbarkeit, wir verwenden im Vorkurs immer \lstinline{public} um es zu vereinfachen).\\
	Operationen beinhalten Code, der mithilfe des Operationsnamen über die Objekte einer Klasse aufgerufen werden kann.
	Der Rückgabewert gibt an, ob und was die Operation zurückgeben wird.
	Sollte eine Operation nichts zurückgeben so hat sie den Rückgabewert \lstinline{void}.\\
	Parameter sind Variablen, die es ermöglichen einen Wert zu übergeben. Diese Variable ist nur innerhalb der Operation im Code gültig.
	Parameter werden in den Klammern nach dem Namen der Operation angegeben.
	Danach kommt in geschweiften Klammern der Inhalt der Operation.
	
	Wenn man eine Operation aufruft und ein Objekt dort übergibt, wird dieses Objekt Argument genannt.
\end{Infobox}
\vspace{5mm}

\begin{Infobox}
	Bei der Definition einer Operation werden die Variablen, die festlegen, was man übergeben, kann \textbf{Parameter} genannt. Die Objekte, die man beim Aufrufen einer Operation übergibt, werden \textbf{Argumente} genannt. 
\end{Infobox}

\begin{lstlisting}[title=\textbf{Operations Syntax}]
	public <RückgabeTyp> <OperationsName>(){
		// Inhalt der Operation
	}
	public <RückgabeTyp> <OperationsName>(<Typ> <ParameterName>, ...){
		// Inhalt der Operation
	}
\end{lstlisting}

\begin{lstlisting}[title=\textbf{Operations Beispiel}]
	public void springen(){
		// Code um zu springen
	}
	
	public Hund sucheEinenHundAus(String wunschFarbe){
		// Code um einen Hund auszusuchen
	}
\end{lstlisting}

\begin{Infobox}
	
	\subsubsection*{Objekte erzeugen:}
	Objekte sind Instanzen von Klassen.
	Um ein Objekt zu erzeugen wird das \lstinline{new} Schlüsselwort verwendet.
	Man schreibt dann \lstinline{new } gefolgt von dem Klassennamen und ein paar Klammern ().
\end{Infobox}
\begin{lstlisting}[title=\textbf{Objekt erstellen Syntax}]
	new <KlassenName>(<MöglicheParameter>);// Ein Objekt wird erzeugt
\end{lstlisting}

\begin{lstlisting}[title=\textbf{Objekt erstellen Beispiel}]
	new Hund("Border Collie");
\end{lstlisting}
\begin{Infobox}
	\subsubsection*{Variablen:}
	Variablen werden genutzt, um Werte zu speichern.
	Sie sind in den Fällen nützlich, in denen ihr einen bestimmten Wert mehrfach im selben Code nutzen wollt.
	Außerdem auch wenn ein Wert beim Starten des Programms noch nicht fest steht oder sich noch ändert.
	Variablen besitzen immer einen Typ und einen Namen.
	Mit einem = kann einer Variable ein Wert zugewiesen werden.
	Wenn man ein Objekt erzeugt, weist man es oft direkt einer Variable zu, damit man es später im Code noch benutzen kann.
\end{Infobox}
\begin{lstlisting}[title=\textbf{Variablen Syntax}]
	<KlassenName> <Name>; // Eine Variable wird erzeugt, ihr ist noch kein Wert zugewiesen.
	
	<KlassenName> <Name> = <Wert>; // Eine Variable wird erzeugt und ihr wird ein Wert zugewiesen.
	
	<Name> = <Wert>; // Einer bestehenden Variable wird ein neuer Wert zugewiesen.
	
	<KlassenName> <Name> = new <KlassenName>(); // Einer neuen Variable wird eine neue Instanz eines Objekts zugewiesen.
	
	<KlassenName> <Name> = <NameEinerAnderenVariable>; // Einer neuen Variable wird der Wert einer anderen Variable zugewiesen. 
\end{lstlisting}
\begin{lstlisting}[title=\textbf{Variable Beispiel}]
	String etwasText = "Dies ist nun der Wert der Variable";
	
	Hund meinHund = new Hund("Deutscher Schäferhund");
\end{lstlisting}
\begin{Infobox}
	\subsubsection*{Kommandos und Abfragen:}
	Kommandos und Abfragen sind zwei Varianten von Operationen einer Klasse. Kommandos beschreiben alle Operationen, die etwas an dem Objekt, auf dem sie aufgerufen werden, verändern.
	Abfragen sind alle Operationen, die nichts verändern, dafür aber eine Eigenschaft als Wert zurückgeben.\\
	Um Operationen eines Objekts aufzurufen, wird zuerst der Objektname geschrieben, gefolgt von einem Punkt und dem Operationsname.
	Am Ende stehen paar Rundeklammern \q{( )}, in denen Parameter stehen können.\\
	{\color{red} Wichtig: } Operationen können nur auf Objekten und nicht auf Klassen aufgerufen werden.
	Es  gibt hier einen Sonderfall, diesen werdet ihr in PSE kennenlernen.
\end{Infobox}
\begin{lstlisting}[title=\textbf{Kommando/Abfrage Syntax}]
	<ObjektName>.<OperationsName>();
	
	<ObjektName>.<OperationsName>(<Argument>);
\end{lstlisting}
\newpage
\begin{lstlisting}[title=\textbf{Kommando/Abfrage Beispiel}]
	// meinHund.sitz(); soll dafür sorgen dass der Hund sich hinsetzt, es wird also etwas am Objekt geändert und somit ist diese Operation ein Kommando.
	meinHund.sitz();
	
	// meinHund.getAlter() soll wiedergeben wie alt der Hund ist, es wird also etwas abgefragt wonach es sich um eine Abfrage handelt.
	meinHund.getAlter();
\end{lstlisting}

\subsection*{Aufgabe 3.2 Namen und Codestyle}
Wie ihr wahrscheinlich schon gemerkt habt, kann Code schnell überfordern und verwirren.
Da man dieses Problem in der Informatik oft antrifft wenn man mit Anderen zusammen arbeitet, gibt es Konventionen wie der Code geschrieben werden sollte, um die Lesbarkeit und das Verständnis zu verbessern.

\vspace{5mm}

\subsubsection*{Namen:}

\begin{Infobox}
	Die Java-Syntax verbietet bereits, dass:
	
	\begin{itemize}
		\item Schlüsselwörter(\lstinline{void}, \lstinline{class}, ...) als Name genutzt werden.
		\item Namen Leerzeichen beinhalten.
		\item Namen mit Zahlen beginnen.
	\end{itemize}
\end{Infobox}
\vspace{5mm}
\begin{Infobox}
	Formulierungen auf die man achten sollte:
	\begin{itemize}
		\item Variablennamen sollten sprechend sein $\rightarrow$ \lstinline{bestDog} statt \lstinline{bD}
		\item Operationen sollten Verben im Namen haben $\rightarrow$ ''makeNoise'' statt ''laut''
		\item Abfragen sollten mit \lstinline{get} beginnen
		\item Kommandos, die einen Wert setzen, sollten mit \lstinline{set} beginnen
		\item Keine Umlaute
		
	\end{itemize}
\end{Infobox}

\noindent
Allgemein sollten alle Namen auf Englisch formuliert werden.
Variablen und Operationen werden immer mit einem Kleinbuchstaben angefangen.
Falls ein Name aus mehreren Wörtern besteht, schreibt man die Anfangsbuchstaben ab dem zweiten Wort groß.
Diese Schreibweise nennt man auch \lstinline{camelCase}.
Sie sollten jeweils so benannt sein, dass erkennbar ist, was sie tun bzw. wofür sie stehen.\\
Für Klassen gelten fast die selben Richtlinien, nur ist hier auch das erst Wort groß.
Das nennt man dann \lstinline{PascalCase} oder \lstinline{UpperCamelCase}.
\begin{lstlisting}[title=\textbf{Beispiel gute Namensgebung}]
	class DogOwner{
		
		void talkAboutTheDog(Dog dog){
			String ownerName = "FIUS";
			// Talk about the Dog until stopped
		}
	}
\end{lstlisting}

\subsubsection*{Einrückungen:}
Um dafür zu sorgen, dass erkennbar ist, ob man sich gerade in einer Operation oder in einer Klasse befindet, wird allgemein jeder Code-Block innerhalb von geschweiften Klammern um eine Stufe weiter eingerückt als der Block, in dem die Klammern stehen.
Wichtig ist, dass ihr immer alle Klammern wieder schließt, die ihr öffnet.
Dabei müsst ihr auch die Reihenfolge der schließenden Klammern beachten!

\vspace{5mm}

\begin{lstlisting}[title=\textbf{Beispiel gute Namensgebung}]
	// Code außerhalb einer Klammerung, nicht eingerückt
	public class RandomClass{
		public void operation1(){
			// Operationsinhalt 2-mal eingerückt
		}
		// Klasseninhalt 1-mal eingerückt
		public void operation2(){
			// Operationsinhalt 2-mal eingerückt
		}
	}
	
\end{lstlisting}
