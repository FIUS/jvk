\excercise{Exceptions – Wenn etwas schief geht}

\Todo{test if this still works}
% sehr einfaches Beispiel von doku lesen aufgabe (gegen wand laufen -> exception) und das fixen (entwede mit turn oder indem man die wand löscht oder indem man überprüft ob man laufen kann)
% learning goal: exceptions im log erkennen, nachvollziehen warum der spezielle Fehler fliegt und strategien zum beheben von fehlern lernen

Öffne die Klasse zu \fbox{Task0 a)} (siehe Aufgabe davor).
Füge nach dem \texttt{// do AB1 task 5 \& 6 here} eine neue Zeile \texttt{this.player.turnClockWise();} ein.
Starte das Programm neu und führe \fbox{Task0 a)} aus.

In der Simulation-Konsole sollte eine \texttt{IllegalMoveException} Exception geflogen sein.
Eine Entity darf nicht in eine Wand reinlaufen.
Wenn sie es dennoch versucht wird eine Exception geworfen.

a) gegen wand laufen Beispiel (selber schreiben oder vorbereitet) ausführen => move wird nicht ausgeführt. Aufgabe: Finde den Stacktrace/ die exception im log/der konsole (Teilaufgabe finde wo die exception für move dokumentiert ist)

\Todo{Stacktrace aufbau sehr grob erklären (exception name/typ, ursprungsort der exception[erste Zeile, und ist anklickbar], rest vom stacktrace [weg der exception, hier uninteressant]).}

\Todo{Klarmachen, dass exception nicht aus real world logik kommt sonder aus "selber" programmierter in game logik}

b) Exception wieder loswerden. Egal welche strategie zählt. Aber ein hinweis, dass einfach auskommentieren keine gute strategie ist, weill man ja eigentlich schon laufen will.
