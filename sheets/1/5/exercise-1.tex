% !TeX root = ./jvk-blatt1.tex

\excercise{Programmstart}
\label{ex1}

\begin{Infobox}[Klassendateien]
    Ein Java Programm setzt sich aus vielen Klassen zusammen, die einzelne Funktionen eines Programms beinhalten.
    Klassen sind dabei jeweils in einer \texttt{.java}-Datei enthalten.
\end{Infobox}

\begin{enumerate}
    \item
    \begin{itemize}
        \item
        Um das Programm starten zu können, musst Du die \texttt{Main}-Klasse ausführen.
        Öffne dazu bitte die \texttt{Main}-Klasse im Packet Explorer von IntelliJ.
        Die Klassendatei befindet sich genauer im \texttt{src/main/java/de/unistuttgart/informatik/fius/jvk} Ordner des Projekts.
        % TODO Hier ein Bild aus dem IntelliJ Package Explorer einfügen

        \item
        Um die \texttt{Main}-Klasse auszuführen, drücke bitte den grünen Play Button $\vartriangleright$ in IntelliJ.
        \item Wenn du bei der Installation alles richtig gemacht hast, sollte sich (spätestens jetzt) die Konsole der IDE öffnen.
        Auf dieser sollte der Text \texttt{Hallo FIUS Java Vorkurs!} erscheinen.
        % TODO Dieses sout in die Main-Klasse einfügen des Codes einfügen

        Wenn hingegen ein roter Text auf der Konsole erscheinen sollte, ist beim Setup etwas schief gelaufen.
        Schaue in diesem Fall nochmal genau nach, ob du einen Schritt bei der Einrichtung vergessen hast und melde dich dann bei einem Helfer.
    \end{itemize}

    \item
    Finde die folgende Zeile in der \lstinline{Main} Klasse:
    \begin{lstlisting}
        // implement task 1 (from sheet 1) here
    \end{lstlisting}

    Um den \simlator starten zu können, musst du den Code aus \autoref{s1c1:program-start-code} hier einfügen.
    Aktuell musst Du den Code noch nicht verstehen, es geht lediglich darum das Programm einmal gestartet zu haben.

    \begin{figure}
        \caption{Code zum Starten des \simulator s in der \texttt{Main}-Klasse}
        \label{s1c1:program-start-code}
        \begin{lstlisting}
        // implement task 1 (from sheet 1) here
        Game demoGame = new Game("FIUS Demo Simulator", new DemoTask(), new DemoTaskVerifier());
        demoGame.run();
        \end{lstlisting}
    \end{figure}

    Wenn Du das Programm jetzt erneut wie in der letzten Teilaufgabe ausführst, sollte sich ein Fenster mit dem \simulator~öffnen.
    Probiere die Elemente im Simulator, die Du in der letzten Teilaufgabe kennengelernt hast, aus.

    Betätige schließlich die (rote) Stop Taste $\square$ in IntelliJ, um das Programm zu stoppen.

% TODO In die Demo-Task ein paar mehr Objekte einfügen, sodass die folgenden Teilaufgaben möglich werden
%\item Lösche die letzte Mauer auf Neo’s Weg. Was passiert?
%\item Stelle Neo eine Mauer in den Weg. Was passiert?
\end{enumerate}

\begin{Infobox}[Optionale Aufgaben] % TODO find a better place for this
    Aufgaben die mit \optional markiert, stellen Herausforderungen dar und sind müssen daher nicht bearbeitet werden.
    Sie setzen meistens Vorkenntnisse mit dem Programmieren voraus und sind deshalb auch oft deutlich schwerer als die normalen Aufgaben.

    Wenn du also an einer optionalen Aufgabe festhängst, dann solltest du mit der nächsten normalen Aufgabe weitermachen.
\end{Infobox}


% \begin{enumerate} \setcounter{enumi}{3}
%\item \optional Versuche nun den Code aus \ref{s1c1:program-start-code} so zu verändern, dass dein Name im Fenstertitel steht.
% \item \optional Finde eine Möglichkeit, den Fenstertitel nach der \lstinline{demoGame.run();} Zeile zu ändern.
% Dafür benötigst du den folgenden Code, den du aber noch auf deinen Namen anpassen musst: \lstinline{demoGame.getGameWindow().setWindowTitle("");}
% \item \optional Versuche, drei Fenster gleichzeitig zu starten.
% \end{enumerate}
